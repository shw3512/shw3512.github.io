%
%
%   Exercise 1
%
%

% True/False (5 questions total)

\section{Exercise \ref{sec : Math112 Spring2022 MockExam2 Q1}}
\label{sec : Math112 Spring2022 MockExam2 Q1}

(10 pt) True/False. For each of the following statements, circle whether it is true or false. No justification is necessary.

\vspace{0.25in}



\noindent{}For parts \ref{itm : Exam2Q1a}--\ref{itm : Exam2Q1b}, let $f(x)$ be a function, and let $F_{1}(x)$ and $F_{2}(x)$ be antiderivatives of $f(x)$.



\begin{enumerate}[label=(\alph*)]
\item\label{itm : Exam2Q1a} (2 pt) The function $2 F_{1}(x)$ is an antiderivative of $2 f(x)$.
\begin{center}
\begin{tabular}{c c c}
true	&	\hspace{1in}	&	false
\end{tabular}
\end{center}
\end{enumerate}

\spaceSolution{0.6in}{% Begin solution.
True. By definition of antiderivative, the hypothesis that $F_{1}(x)$ is an antiderivative of $f(x)$ means that $F_{1}'(x) = f(x)$. Therefore, by linearity of the derivative,
\begin{align*}
(2 F_{1}(x))'
=
2 F_{1}'(x)
=
2 f(x)
\end{align*}
This equation says that $2 F_{1} (x)$ is an antiderivative of $2 f(x)$.}% End solution.



\begin{enumerate}[resume,label=(\alph*)]
\item\label{itm : Exam2Q1b} (2 pt) The function $F_{1}(x) + F_{2}(x)$ is an antiderivative of $2 f(x)$.
\begin{center}
\begin{tabular}{c c c}
true	&	\hspace{1in}	&	false
\end{tabular}
\end{center}
\end{enumerate}

\spaceSolution{0.6in}{% Begin solution.
True. By definition of antiderivative, $F_{1}'(x) = f(x)$ and $F_{2}'(x) = f(x)$. Therefore, by linearity of the derivative,
\begin{align*}
\left(F_{1}(x) + F_{2}(x)\right)'
=
F_{1}'(x) + F_{2}'(x)
=
f(x) + f(x)
=
2 f(x)
\end{align*}
This equation says that $F_{1}(x) + F_{2}(x)$ is an antiderivative of $2 f(x)$.}% End solution.



\noindent{}For parts \ref{itm : Exam2Q1c}--\ref{itm : Exam2Q1d}, let $f(x)$ and $g(x)$ be functions such that
\begin{align*}
\int_{0}^{2} f(x) \spaceIntd \intd x
=
1
&&
\int_{0}^{2} g(x) \spaceIntd \intd x
=
-4
\end{align*}

\begin{enumerate}[resume,label=(\alph*)]
\item\label{itm : Exam2Q1c} (2 pt) $\displaystyle\int_{0}^{2} \left[3 f(x) + \frac{1}{2} g(x)\right] \spaceIntd \intd x = 1$
\begin{center}
\begin{tabular}{c c c}
true	&	\hspace{1in}	&	false
\end{tabular}
\end{center}
\end{enumerate}

\spaceSolution{0.6in}{% Begin solution.
True. The definite integral is linear. Therefore
\begin{align*}
\int_{0}^{2} \left[3 f(x) + \frac{1}{2} g(x)\right] \spaceIntd \intd x
=
3 \int_{0}^{2} f(x) \spaceIntd \intd x + \frac{1}{2} \int_{0}^{2} g(x) \spaceIntd \intd x
=
3 (1) + \frac{1}{2} (-4)
=
1
\end{align*}}% End solution.



\begin{enumerate}[resume,label=(\alph*)]
\item\label{itm : Exam2Q1d} (2 pt) $\displaystyle\int_{0}^{1} g(x) \spaceIntd \intd x + \int_{1}^{2} g(x) \spaceIntd \intd x = -2$
\begin{center}
\begin{tabular}{c c c}
true	&	\hspace{1in}	&	false
\end{tabular}
\end{center}
\end{enumerate}

\spaceSolution{0.6in}{% Begin solution.
False. Note that
\begin{align*}
\int_{0}^{1} g(x) \spaceIntd \intd x + \int_{1}^{2} g(x) \spaceIntd \intd x
=
\int_{0}^{2} g(x) \spaceIntd \intd x
=
-4
\end{align*}}% End solution.



\begin{enumerate}[resume,label=(\alph*)]
\item\label{itm : Exam4P2Q1e} (2 pt) If the definite integral of a function is zero, then the function must be the zero function.
\begin{center}
\begin{tabular}{c c c}
true	&	\hspace{1in}	&	false
\end{tabular}
\end{center}
\end{enumerate}

\spaceSolution{0.6in}{% Begin solution.
False. For example, consider the function $f : \reals \rightarrow \reals$ given by $f(x) = \sin x$. This is not the zero function, and we can check that $\int_{0}^{2 \pi} \sin x \spaceIntd \intd x = 0$. The positive and negative areas exactly cancel.}% End solution.





%
%
%   Exercise 2
%
%

% Evaluating limits (include one exercise on function-limit interchange)

\newpage

\section{Exercise \ref{sec : Math112 Spring2022 MockExam2 Q2}}
\label{sec : Math112 Spring2022 MockExam2 Q2}

(16 pt) Evaluate each of the following limits. Briefly but clearly justify your work.



\begin{enumerate}[label=(\alph*)]
\item\label{itm : Exam2Q2a} (4 pt) Use the Taylor series
\begin{align}
\ln(1 + x)
=
x - \frac{1}{2} x^{2} + \frac{1}{3} x^{3} - \ldots%
\label{eq : Exam2Q2c Taylor Series ln(1 + x)}
\end{align}
where the terms in ``$\ldots$'' all involve $x$ to the power $4$ or higher, to evaluate
\begin{align*}
\lim_{x \rightarrow 0} \frac{x^{2} + x - \ln(1 + x)}{x^{2}}
\end{align*}
\end{enumerate}

\spaceSolution{2.5in}{% Begin solution.
Substituting the Taylor series \eqref{eq : Exam2Q2c Taylor Series ln(1 + x)} into the limit, the simplifying, we get
\begin{align*}
\lim_{x \rightarrow 0} \frac{x^{2} + x - \ln(1 + x)}{x^{2}}
&=
\lim_{x \rightarrow 0} \frac{x^{2} + x - \left(x - \frac{1}{2} x^{2} + \frac{1}{3} x^{3} - \ldots\right)}{x^{2}}
\\
&=
\lim_{x \rightarrow 0} \frac{\frac{3}{2} x^{2} - \frac{1}{3} x^{3} + \ldots}{x^{2}}
\\
&=
\lim_{x \rightarrow 0} \frac{\frac{3}{2} - \frac{1}{3} x + \ldots}{1}
\\
&=
\frac{3}{2}
\end{align*}
where in evaluating the final limit we use the fact that, after canceling the factor $x^{2}$ in the third equality, the terms in ``$\ldots$'' all involve $x$ to the power $2$ or higher.}% End solution.



\begin{enumerate}[resume,label=(\alph*)]
\item\label{itm : Exam2Q2b} (4 pt) Use l'H\^{o}pital's rule to evaluate
\begin{align*}
\lim_{x \rightarrow 0} \frac{x^{2} + x - \ln(1 + x)}{x^{2}}
\end{align*}
(Note that this is the same limit as in part \ref{itm : Exam2Q2a}.)
\end{enumerate}

\spaceSolution{3in}{% Begin solution.
At each step, we (i) check that direct evaluation (abbreviated ``D.E.'' below) of the limit gives $\frac{0}{0}$, so l'H\^{o}pital's rule applies; and (ii) apply l'H\^{o}pital's rule (differentiate numerator and denominator). The first result from direct evaluation that is not an indeterminate form is the limit.
\begin{align*}
&\lim_{x \rightarrow 0} \frac{x^{2} + x - \ln(1 + x)}{x^{2}}
\xrightarrow{\text{D.E.}}
\frac{0^{2} + 0 - \ln(1)}{0^{2}}
=
\frac{0}{0}
\\
&=
\lim_{x \rightarrow 0} \frac{2 x + 1 - \frac{1}{1 + x}}{2 x}
\xrightarrow{\text{D.E.}}
\frac{0 + 1 - 1}{0}
=
\frac{0}{0}
\\
&=
\lim_{x \rightarrow 0} \frac{2 + \frac{1}{(1 + x)^{2}}}{2}
=
\frac{2 + 1}{2}
=
\frac{3}{2}
\end{align*}}% End solution.



\newpage

\begin{enumerate}[resume,label=(\alph*)]
\item\label{itm : Exam2Q2c} (4 pt) $\displaystyle\lim_{x \rightarrow 2} \frac{1 - \sqrt{5 - 2 x}}{x - 2}$
\end{enumerate}

\spaceSolution{3.5in}{% Begin solution.
Direct evaluation of the limit gives the indeterminate form $\frac{0}{0}$. Thus l'H\^{o}pital's rule applies, and we may use it to find the limit. However, let us instead multiply numerator and denominator by the conjugate of the numerator (equivalent to multiplying by $1$):
\begin{align*}
\lim_{x \rightarrow 2} \frac{1 - \sqrt{5 - 2 x}}{x - 2}
&=
\lim_{x \rightarrow 2} \frac{1 - \sqrt{5 - 2 x}}{x - 2} \cdot \frac{1 + \sqrt{5 - 2 x}}{1 + \sqrt{5 - 2 x}}
\\
&=
\lim_{x \rightarrow 2} \frac{1 - (\sqrt{5 - 2 x})^{2}}{(x - 2) (1 + \sqrt{5 - 2 x})}
\\
&=
\lim_{x \rightarrow 2} \frac{2 x - 4}{(x - 2) (1 + \sqrt{5 - 2 x})}
\end{align*}
Factoring the numerator and canceling the common factor $x - 2$ allows us to compute the limit by direct evaluation:
\begin{align*}
&=
\lim_{x \rightarrow 2} \frac{2 (x - 2)}{(x - 2) (1 + \sqrt{5 - 2 x})}
\\
&=
\lim_{x \rightarrow 2} \frac{2}{1 + \sqrt{5 - 2 x}}
\\
&=
\frac{2}{1 + \sqrt{1}}
=
1
\end{align*}}% End solution.



\begin{enumerate}[resume,label=(\alph*)]
\item\label{itm : Exam2Q2d} (4 pt) $\displaystyle\lim_{x \rightarrow +\infty} x^{3} e^{-x}$
\end{enumerate}

\spaceSolution{3.5in}{% Begin solution.
Direct evaluation of the limit gives $+\infty \cdot 0$, an indeterminate form. It is not an indeterminate form to which l'H\^{o}pital's rule applies. We massage the given function into an equivalent form by moving the exponential factor to the denominator:
\begin{align*}
\lim_{x \rightarrow +\infty} x^{3} e^{-x}
=
\lim_{x \rightarrow +\infty} \frac{x^{3}}{e^{x}}
\end{align*}
Direct evaluation of this limit gives $\frac{+\infty}{+\infty}$, an indeterminate form to which l'H\^{o}pital's rule applies. Applying it iteratively (three times total), we get
\begin{align*}
&=
\lim_{x \rightarrow +\infty} \frac{3 x^{2}}{e^{x}}
\xrightarrow{\text{D.E.}}
\frac{+\infty}{+\infty}
\\
&=
\lim_{x \rightarrow +\infty} \frac{6 x}{e^{x}}
\xrightarrow{\text{D.E.}}
\frac{+\infty}{+\infty}
\\
&=
\lim_{x \rightarrow +\infty} \frac{6}{e^{x}}
\xrightarrow{\text{D.E.}}
\frac{1}{+\infty}
=
0
\end{align*}
(Recall that a limit of $\frac{1}{+\infty}$ means that we're making the denominator larger and larger---arbitrarily large---while the numerator stays constant, so the fraction as a whole becomes smaller and smaller---arbitrarily small, that is, close to $0$.) Thus the limit is $0$.}% End solution.





%
%
%   Exercise 3
%
%

% Finding antiderivatives

\newpage

\section{Exercise \ref{sec : Math112 Spring2022 MockExam2 Q3}}
\label{sec : Math112 Spring2022 MockExam2 Q3}

(12 pt) Evaluate the indefinite integrals. (That is, find the most-general antiderivative $F(x)$ of the integrand $f(x)$ in the following integrals $\int f(x) \spaceIntd \intd x$.)



\begin{enumerate}[label=(\alph*)]
\item\label{itm : Exam2Q3a} (4 pt) $\displaystyle\int 4 x^{3} - 3 x^{2} + 1 \spaceIntd \intd x$
\end{enumerate}

\spaceSolution{2.25in}{% Begin solution.
For each indefinite integral in this exercise, we (1) use linearity of the integral to ``break'' the integral into separate terms and factor out constants, (2) guess and check to find an antiderivative for each smaller integral, and (3) include ``$+ C$'' to get the most-general antiderivative, where $C$ is an arbitrary real number (that is, a constant). We check our result by verifying that differentiating our most-general antiderivative (what we've been calling $F(x)$) gives the original integrand (the function in the original integral, which we've been calling $f(x)$).

We compute
\begin{align*}
\int 4 x^{3} - 3 x^{2} + 1 \spaceIntd \intd x
&=
4 \int x^{3} \spaceIntd \intd x - 3 \int x^{2} \spaceIntd \intd x + \int 1 \spaceIntd \intd x
\\
&=
4 \left[\frac{1}{4} x^{4}\right] - 3 \left[\frac{1}{3} x^{3}\right] + x + C
\\
&=
x^{4} - x^{3} + x + C
\end{align*}
We check
\begin{align*}
\left(x^{4} - x^{3} + x + C\right)'
=
4 x^{3} - 3 x^{2} + 1
=
f(x)
\end{align*}}% End solution.



\begin{enumerate}[resume,label=(\alph*)]
\item\label{itm : Exam2Q3b} (4 pt) $\displaystyle\int 6 e^{3 x} - 25 \cos(5 x) \spaceIntd \intd x$
\end{enumerate}

\spaceSolution{2.25in}{% Begin solution.
We compute
\begin{align*}
\int 6 e^{3 x} - 25 \cos(5 x) \spaceIntd \intd x
&=
6 \int e^{3 x} \spaceIntd \intd x - 25 \int \cos(5 x) \spaceIntd \intd x
\\
&=
6 \left[\frac{1}{3} e^{3 x}\right] - 25 \left[\frac{1}{5} \sin(5 x)\right] + C
\\
&=
2 e^{3 x} - 5 \sin(5 x) + C
\end{align*}
We check
\begin{align*}
\left(2 e^{3 x} - 5 \sin(5 x) + C\right)'
=
2 e^{3 x} \cdot 3 - 5 \cos(5 x) \cdot 5
=
6 e^{3 x} - 25 \cos(5 x)
=
f(x)
\end{align*}}% End solution.



\begin{enumerate}[resume,label=(\alph*)]
\item\label{itm : Exam2Q3c} (4 pt) $\displaystyle\int \frac{1 - x \sin x}{x} \spaceIntd \intd x$
\end{enumerate}

\spaceSolution{2.25in}{% Begin solution.
We compute
\begin{align*}
\int \frac{1 - x \sin x}{x} \spaceIntd \intd x
&=
\int \frac{1}{x} - \sin x \spaceIntd \intd x
\\
&=
\int \frac{1}{x} \spaceIntd \intd x - \int \sin x \spaceIntd \intd x
\\
&=
\ln x - (-\cos x) + C
\\
&=
\ln x + \cos x + C
\end{align*}
(More precisely, $\int \frac{1}{x} \spaceIntd \intd x = \ln\abs{x} + C$. We'll ignore the subtlety about the absolute value for now.) We check
\begin{align*}
\left(\ln x + \cos x + C\right)'
=
\frac{1}{x} - \sin x
=
\frac{1}{x} - \frac{x \sin x}{x}
=
\frac{1 - x \sin x}{x}
=
f(x)
\end{align*}}% End solution.





%
%
%   Exercise 4
%
%

% Lower and upper sums, implicit fundamental theorem

\newpage

\section{Exercise \ref{sec : Math112 Spring2022 MockExam2 Q4}}
\label{sec : Math112 Spring2022 MockExam2 Q4}

\begin{wrapfigure}{r}{0.35\textwidth}
\centering
\includegraphics[width=0.33\textwidth]{\filePathGraphics Exam2Q4_Graph.png}
\end{wrapfigure}
(10 pt) Consider the piecewise function $f : \reals \rightarrow \reals$ given by
\begin{align*}
f(x)
=
\begin{dcases*}
2				&	if $x \leq 0$		\\
2 - \sqrt{4 x - x^{2}}	&	if $-2 \leq x \leq 0$	\\
2 - x				&	if $x \geq 2$
\end{dcases*}
\end{align*}
A graph of $f$ is shown at right.



\begin{enumerate}[label=(\alph*)]
\item\label{itm : Exam2Q4a} (2 pt) Consider the rule of assignment for $f$ on the interval $[0,2]$, namely,
\begin{align*}
y
=
2 - \sqrt{4 x - x^{2}}
\end{align*}
Use algebra to massage this equation into the form $(x - a)^{2} + (y - b)^{2} = r^{2}$, thus showing that the graph of $f$ on the interval $[0,2]$ is part of a circle of radius $r = 2$.
\end{enumerate}

\spaceSolution{1.25in}{% Begin solution.
First, we need to get rid of the square root, which we'll do by squaring. So let's get the square root by itself
\begin{align*}
y - 2
=
-\sqrt{4 x - x^{2}}
\end{align*}
and then square both sides
\begin{align*}
(y - 2)^{2}
=
\left(-\sqrt{4 x - x^{2}}\right)^{2}
=
4 x - x^{2}
\end{align*}
To put the right side in the form $(x - a)^{2}$, we complete the square:
\begin{align*}
4 x - x^{2}
=
-(x^{2} - 4 x)
=
-(x^{2} - 4 x + 4 - 4)
=
-(x^{2} - 4 x + 4) + 4
=
-(x - 2)^{2} + 4
\end{align*}
We substitute this into the previous expression, then move the $x$ and $y$ expressions to the same side, obtaining
\begin{align*}
(x - 2)^{2} + (y - 2)^{2}
=
4
=
2^{2}
\end{align*}
This is the equation of a circle with center $(a,b) = (2,2)$ and radius $r = 2$.}% End solution.



%\begin{enumerate}[resume,label=(\alph*)]
%\item\label{itm : Exam2Q4b} (4 pt) Draw and compute a lower- and upper-sum estimate (call them $L_{3}$ and $U_{3}$, respectively) for $\int_{-2}^{4} f(x) \spaceIntd \intd x$ by partitioning $[-2,4]$ into three subintervals, each of width $2$.
%\end{enumerate}
%\begin{center}
%\includegraphics[scale=0.5]{\filePathGraphics Exam2Q4_Graph.png}
%\hspace{0.5in}
%\includegraphics[scale=0.5]{\filePathGraphics Exam2Q4_Graph.png}
%\\
%Lower sum ($L_{3}$)
%\hspace{1.75in}
%Upper sum ($U_{3}$)
%\end{center}
%
%\spaceSolution{2in}% End solution.
%
%
%
%\newpage
%
%\begin{enumerate}[resume,label=(\alph*)]
%\item\label{itm : Exam2Q4c} (4 pt) Draw and compute a lower- and upper-sum estimate (call them $L_{6}$ and $U_{6}$, respectively) for $\int_{-2}^{4} f(x) \spaceIntd \intd x$ by partitioning $[-2,4]$ into six subintervals, each of width $1$. (Use the rule of assignment for $f(x)$ to help find rectangle heights. For reference, $\sqrt{3} \approx 1.73$.)
%\end{enumerate}
%\begin{center}
%\includegraphics[scale=0.5]{\filePathGraphics Exam2Q4_Graph.png}
%\hspace{0.5in}
%\includegraphics[scale=0.5]{\filePathGraphics Exam2Q4_Graph.png}
%\\
%Lower sum ($L_{6}$)
%\hspace{1.5in}
%Upper sum ($U_{6}$)
%\end{center}
%
%\spaceSolution{1.75in}% End solution.



\begin{enumerate}[resume,label=(\alph*)]
\item\label{itm : Exam2Q4d} (4 pt) Use geometry to compute the exact value of $\int_{-2}^{4} f(x) \spaceIntd \intd x$. You may leave your answer in terms of $\pi$; if you prefer, you may use that $\pi \approx 3.14$. (Hint: Use the conclusion from part \ref{itm : Exam2Q4a} to help find the area on the interval $[0,2]$. The area of a full circle of radius $r$ is $\pi r^{2}$. Remember, area is signed ($+/-$)!)
\end{enumerate}

\spaceSolution{1.75in}{% Begin solution.
We break the definite integral on the interval $[-2,4]$ into three pieces, based on the geometry of the graph of $f(x)$:
\begin{align}
\int_{-2}^{4} f(x) \spaceIntd \intd x
=
\int_{-2}^{0} f(x) \spaceIntd \intd x + \int_{0}^{2} f(x) \spaceIntd \intd x + \int_{2}^{4} f(x) \spaceIntd \intd x%
\label{eq : Exam2Q4d Intermediate}
\end{align}
Let's call these areas $A_{[-2,0]}$, $A_{[0,2]}$, and $A_{[2,4]}$, respectively.

On the interval $[-2,0]$, the area $A_{[-2,0]}$ between the graph of $f(x)$ and the $x$-axis is a rectangle (in fact, square) of base $2$ and height $2$. This area is positive because it lies above the $x$-axis. So
\begin{align*}
A_{[-2,0]}
=
(2) (2)
=
4
\end{align*}

On the interval $[0,2]$, the area $A_{[0,2]}$ between the graph of $f(x)$ and the $x$-axis is a square minus a quarter circle, where the sides of the square and the radius of the circle are $2$. This area is positive because it lies above the $x$-axis. So
\begin{align*}
A_{[0,2]}
=
(2) (2) - \frac{1}{4} \pi (2)^{2}
=
4 - \pi
\end{align*}

On the interval $[2,4]$, the area $A_{[2,4]}$ between the graph of $f(x)$ and the $x$-axis is a triangle (in fact, a right triangle) of base $2$ and height $2$. This area is negative because it lies below the $x$-axis. So
\begin{align*}
A_{[2,4]}
=
-\frac{1}{2} (2) (2)
=
-2
\end{align*}

Substituting these results into \eqref{eq : Exam2Q4d Intermediate}, we conclude
\begin{align*}
\int_{-2}^{4} f(x) \spaceIntd \intd x
&=
A_{[-2,0]} + A_{[0,2]} + A_{[2,4]}
\\
&=
[4] + [4 - \pi] + [-2]
\\
&=
6 - \pi
\end{align*}}% End solution.



\begin{enumerate}[resume,label=(\alph*)]
\item\label{itm : Exam2Q4e} (4 pt) Is the definite integral $\int_{0}^{4} f(x) \spaceIntd \intd x$ positive, negative, or zero? Justify. (Hint: You need not give an exact value, though you may. It is enough to justify your answer using the geometry of the graph of $f(x)$.)
\end{enumerate}

\spaceSolution{1in}{% Begin solution.
The area left over when we subtract the quarter circle from its enclosing square---what we called $A_{[0,2]}$ in part \ref{itm : Exam2Q4d}---fits inside the right triangle of base $2$ and height $2$---what we called $A_{[2,4]}$---leaving a gap of area along the hypotenuse of the triangle. Because $A_{[2,4]}$ is negative area and $A_{[0,2]}$ is positive area, the overlapping parts cancel, and the gap of area that remains (which belongs to $A_{[2,4]}$) is negative. Thus we expect $\int_{0}^{4} f(x) \spaceIntd \intd x$ to be negative.

Note that, if we use our values from part \ref{itm : Exam2Q4d} to compute the exact value of this definite integral, we get
\begin{align*}
\int_{0}^{4} f(x) \spaceIntd \intd x
=
A_{[0,2]} + A_{[2,4]}
=
[4 - \pi] + [-2]
=
2 - \pi
<
0
\end{align*}
(Note that $2 - \pi = -(\pi - 2) \approx -1.1416$.)}% End solution.





%
%
%   Exercise 5
%
%

% Implicit fundamental theorem, average value

\newpage

\section{Exercise \ref{sec : Math112 Spring2022 MockExam2 Q5}}
\label{sec : Math112 Spring2022 MockExam2 Q5}

(18 pt) Consider the function $f : \reals \rightarrow \reals$ given by
\begin{align}
f(x)
=
4 x^{3} - 12 x + 8%
\label{eq : Exam2Q5 Rule Of Assignment}
\end{align}



\begin{enumerate}[label=(\alph*)]
\item\label{itm : Exam2Q5a} (4 pt) Find an antiderivative $F(x)$ of $f(x)$. Verify that it is indeed an antiderivative.
\end{enumerate}

\spaceSolution{3in}{% Begin solution.
We compute
\begin{align*}
\int 4 x^{3} - 12 x + 8 \spaceIntd \intd x
&=
4 \int x^{3} \spaceIntd \intd x - 12 \int x \spaceIntd \intd x + 8 \int 1 \spaceIntd \intd x
\\
&=
4 \left[\frac{1}{4} x^{4}\right] - 12 \left[\frac{1}{2} x^{2}\right] + 8 x + C
\\
&=
x^{4} - 6 x^{2} + 8 x + C
\end{align*}
We check
\begin{align*}
\left(x^{4} - 6 x^{2} + 8 x + C\right)'
=
4 x^{3} - 12 x + 8
=
f(x)
\end{align*}
We're asked to find an antiderivative, not necessarily the most-general antiderivative, of $f(x)$. So, if desired, we may set $C$ equal to a particularly convenient value, for example, $C = 0$. In this case, we get a particular antiderivative
\begin{align*}
F(x)
=
x^{4} - 6 x^{2} + 8 x
\end{align*}
We'll use this as $F(x)$ in the remainder of this exercise.}% End solution.



\begin{enumerate}[resume,label=(\alph*)]
\item\label{itm : Exam2Q5b} (2 pt) Using your antiderivative $F(x)$ from part \ref{itm : Exam2Q5a}, show that
\begin{align*}
\int_{-1}^{2} f(x) \spaceIntd \intd x
=
F(2) - F(-1)
=
21
\end{align*}
Take the first equality as given. (It is the fundamental theorem of calculus.) Just show that $F(2) - F(-1) = 21$.
\end{enumerate}

\spaceSolution{3in}{% Begin solution.
We compute
\begin{align*}
F(2)
&=
2^{4} - 6 \cdot 2^{2} + 8 \cdot 2
=
16 - 24 + 16
=
8
\\
F(-1)
&=
(-1)^{4} - 6 \cdot (-1)^{2} + 8 \cdot (-1)
=
1 - 6 - 8
=
-13
\end{align*}
Thus
\begin{align*}
F(2) - F(-1)
=
8 - (-13)
=
21
\end{align*}
as we were asked to show.}% End solution.



\newpage

\begin{enumerate}[resume,label=(\alph*)]
\item\label{itm : Exam2Q5c} (4 pt) On the graphs of $f$ below, draw a lower- and upper-sum approximation to the definite integral $\int_{-1}^{2} f(x) \spaceIntd \intd x$. Partition the interval $[-1,2]$ into three subintervals, each of width $1$.
\end{enumerate}
\begin{center}
\includegraphics[scale=0.5]{\filePathGraphics Exam2Q5_Graph.png}% Activate for exam.
%\includegraphics[scale=0.5]{\filePathGraphics Exam2Q5_LowerSum.png}% Activate for solutions.
\hspace{0.5in}
\includegraphics[scale=0.5]{\filePathGraphics Exam2Q5_Graph.png}% Activate for exam.
%\includegraphics[scale=0.5]{\filePathGraphics Exam2Q5_UpperSum.png}% Activate for solutions.
\\
Lower sum
\hspace{2.25in}
Upper sum
\end{center}

\spaceSolution{0in}{% Begin solution.
}% End solution.



\begin{enumerate}[resume,label=(\alph*)]
\item\label{itm : Exam2Q5d} (4 pt) Compute the upper- and lower-sum approximations you sketched in part \ref{itm : Exam2Q5c}. (Use the rule of assignment for $f(x)$, given in Equation \eqref{eq : Exam2Q5 Rule Of Assignment} at the start of this exercise, to help find the heights of your rectangles.) Show that these approximations bound your value of the definite integral computed in part \ref{itm : Exam2Q5b}.
\end{enumerate}

\spaceSolution{2.5in}{% Begin solution.
For the lower-sum approximation, we compute
\begin{align*}
L_{3}
&=
1 \cdot f(0) + 1 \cdot f(1) + 1 \cdot f(1)
\\
&=
1 \cdot 8 + 1 \cdot 0 + 1 \cdot 0
\\
&=
8
\end{align*}
For the upper-sum approximation, we compute
\begin{align*}
U_{3}
&=
1 \cdot f(-1) + 1 \cdot f(0) + 1 \cdot f(2)
\\
&=
1 \cdot (-4 - (-12) + 8) + 1 \cdot 8 + 1 \cdot (32 - 24 + 8)
\\
&=
16 + 8 + 16
\\
&=
40
\end{align*}
We verify that
\begin{align*}
L_{3}
=
8
\leq
\int_{-1}^{2} f(x) \spaceIntd \intd x
=
21
\leq
U_{3}
=
40
\end{align*}}% End solution.



\begin{enumerate}[resume,label=(\alph*)]
\item\label{itm : Exam2Q5e} (4 pt) Your friend computes lower- and upper-sum approximations for $\int_{-1}^{2} f(x) \spaceIntd \intd x$ with $300$ subintervals, each of width $\frac{1}{100}$. Your friend reports the results are $L_{300} \approx -1.2723$ and $U_{300} \approx 20.8603$. Explain to your friend why neither result can be correct. (Be kind---your friend just did a lot of work!)
\end{enumerate}

\spaceSolution{2in}{% Begin solution.
The lower sum cannot be correct, because the function $f(x)$ that we are integrating is nonnegative (that is, greater than or equal to $0$) everywhere on the interval $[-1,2]$ over which we are integrating. Thus the smallest possible value of any lower-sum approximation is $0$.

The upper sum cannot be correct, because the actual area under the graph of $f(x)$ on the interval $[-1,2]$ is given by $\int_{-1}^{2} f(x) \spaceIntd \intd x$, which we showed in part \ref{itm : Exam2Q5b} equals $21$; and any upper-sum approximation must be greater than or equal to the actual area.}% End solution.