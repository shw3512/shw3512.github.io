%
%
%   LQuiz 01 : 2022-01-11 (T)
%
%

%
%   Section 1 : Precalculus
%

\section{Precalculus}

\subsection{Exercise \ref{sec : Precalculus Q1}}
\label{sec : Precalculus Q1}

% Reference : SHW

(4 pt) Let
\begin{align*}
f(x)
&=
\frac{3 x + 8}{x^{2} + 4 x + 2}
&
g(x)
&=
x - 2
\end{align*}
(Assume the domain and codomain of $f$ and $g$ are the largest possible subsets of the real numbers, $\reals$, for which the above rules of assignment make sense.) Find the composition $(f \circ g)(x)$, presented as simply as possible, and state its domain (i.e. the allowed values of $x$).

\spaceSolution{3in}{% Begin solution.
We compute
\begin{align*}
(f \circ g)(x)
=
f\left(g(x)\right)
=
f\left(x - 2\right)
=
\frac{3 (x - 2) + 8}{(x - 2)^{2} + 4 (x - 2) + 2}
=
\frac{3 x + 2}{x^{2} - 2}
\end{align*}
This final expression for $f \circ g$ is not defined if and only if the denominator equals $0$, i.e.
\begin{align*}
x^{2} - 2
&=
0
&
&\Leftrightarrow
&
x
&=
\pm{}\sqrt{2}
\end{align*}
Hence the domain of $f \circ g$ is all real numbers except $\pm{}\sqrt{2}$. We can also write this as
\begin{align*}
\reals \setminus \{\pm{}\sqrt{2}\}
&&
\text{or}
&&
\left(-\infty,-\sqrt{2}\right) \cup \left(-\sqrt{2},\sqrt{2}\right) \cup \left(\sqrt{2},+\infty\right)
\end{align*}}% End solution.



\subsection{Exercise \ref{sec : Precalculus Q2}}
\label{sec : Precalculus Q2}

% Reference : SHW

(4 pt) Show that the following expression simplifies to a single trigonometric function. State any disallowed values of $\theta$.
\begin{align*}
(\sec \theta - \cos \theta) \csc \theta
\end{align*}

\spaceSolution{3in}{% Begin solution.
Let's begin by distributing, then rewriting $\csc \theta$ and $\sec \theta$ in terms of $\cos \theta$ and $\sin \theta$:
\begin{align*}
\left(\sec \theta - \cos \theta\right) \csc \theta
=
\sec \theta \csc \theta - \cos \theta \csc \theta
=
\frac{1}{\cos \theta \sin \theta} - \frac{\cos \theta}{\sin \theta}
\end{align*}
To head further in the direction of a single trigonometric function, let's write the difference as a single fraction:
\begin{align*}
\ldots
=
\frac{1}{\cos \theta \sin \theta} - \frac{\cos \theta}{\cos \theta} \frac{\cos \theta}{\sin \theta}
=
\frac{1 - \cos^{2} \theta}{\cos \theta \sin \theta}
\end{align*}
Recall the identity $\cos^{2} \theta + \sin^{2} \theta = 1$. Using this, we can rewrite the expression as
\begin{align*}
\ldots
=
\frac{\sin^{2} \theta}{\cos \theta \sin \theta}
=
\frac{\sin \theta}{\cos \theta}
=
\tan \theta
\end{align*}
We can only cancel a factor of $\sin \theta$ from numerator and denominator when $\sin \theta \neq 0$, i.e. when $\theta$ is not an integer multiple of $\pi$. Otherwise, the denominator is $0$; thus these values of $\theta$ are not allowed. Moreover, $\tan \theta$ is not defined when $\theta$ is an odd-integer multiple of $\frac{\pi}{2}$. These are the only disallowed values of $\theta$.}% End solution.



\newpage



\subsection{Exercise \ref{sec : Precalculus Q3}}
\label{sec : Precalculus Q3}

% Reference : OpenStax, Calculus, V1, Exercise 1.5.288

(4 pt) Solve the following equation exactly. If this is not possible, or no solution exists, state so.
\begin{align*}
\ln \sqrt{x + 3}
=
2
\end{align*}

\spaceSolution{3in}{% Begin solution.
Let's begin by exponentiating both sides and simplifying:
\begin{align*}
\sqrt{x + 3}
=
e^{\ln \sqrt{x + 3}}
=
e^{2}
\end{align*}
Because the exponential function% Begin footnote.
\footnote{The inverse of the exponential function --- the natural logarithm function --- is also injective.} % End footnote.
is injective (i.e. one-to-one), this transformation does not change the solutions to the original equation. Now we may square both sides and solve for $x$:
\begin{align*}
x
=
\left(e^{2}\right)^{2} - 3
=
e^{4} - 3
\end{align*}
The ``square'' function is not injective, so it is possible that we added spurious solutions at that step.% Begin footnote.
\footnote{Consider the equation $x = 1$. Square both sides, then solve. What do you find? Is the result you found false?} % End footnote.
However, we can check directly that the solution $x = e^{4} - 3$ indeed solves the original equation.}% End solution.



\subsection{Exercise \ref{sec : Precalculus Q4}}
\label{sec : Precalculus Q4}

% Reference : SHW

(4 pt) Compute the following.
\begin{align*}
\sum_{m = 1}^{19} (5 m + 2) - \sum_{n = -2}^{4} n^{2}
\end{align*}

\spaceSolution{3in}{% Begin solution.
Let's tackle the two sums separately. For the first sum, we compute
\begin{align*}
\sum_{m = 1}^{19} (5 m + 2)
=
5 \sum_{m = 1}^{19} m + \sum_{m = 1}^{19} 2
=
5 \frac{19 (19 + 1)}{2} + 19 (2)
=
5 (190) + 38
=
950 + 38
\end{align*}
For the second sum, there is a formula when the summation index starts at $1$ (or $0$),% Begin footnote.
\footnote{The formula, which we can prove by induction, is $\sum_{n = 1}^{N} n^{2} = \frac{1}{6} N (N + 1) (2 N + 1)$. (Can you prove this?)} % End footnote.
and we could partition the sum into two subsums to use it. However, because this sum has only 7 relatively small terms, I find it easier to compute them directly:
\begin{align*}
\sum_{n = -2}^{4} n^{2}
=
(-2)^{2} + (-1)^{2} + \ldots + 3^{2} + 4^{2}
=
4 + 1 + \ldots + 9 + 16
=
35
\end{align*}
Putting our two results together, we conclude that
\begin{align*}
\sum_{m = 1}^{19} (5 m + 2) - \sum_{n = -2}^{4} n^{2}
=
(950 + 38) - (35)
=
953
\end{align*}}% End solution.



\newpage



%
%   Section 2 : Differential calculus
%

\section{Differential calculus}

\subsection{Exercise \ref{sec : Differential Calculus Q1}}
\label{sec : Differential Calculus Q1}

% Reference : Math 111, Fall 2021, Final, Q3

(4 pt) Let $f : \reals \rightarrow \reals$ be the function defined by
\begin{align*}
f(x)
=
x^{3} + 3 x^{2} - 9 x + 13
\end{align*}
Find all points $(x,f(x))$ where the tangent line to the graph of $f$ is horizontal.

\spaceSolution{3in}{% Begin solution.
Horizontal tangents to the graph of $f$ occurs at values of $x$ where $f'(x) = 0$. We compute
\begin{align*}
0
\seteq
f'(x)
=
3 x^{2} + 6 x - 9
=
3 (x^{2} + 2 x - 3)
=
3 (x + 3) (x - 1)
\end{align*}
This equation has two solutions: $x = -3$ and $x = 1$. For each solution, we evaluate $f$ at that value of $x$ to find the corresponding point $(x,f(x))$ on the graph of $f$. This gives us the two points
\begin{align*}
(-3,40)
&&
\text{and}
&&
(1,8)
\end{align*}
(Can you determine whether each point is a local minimum or a local maximum of $f$? Do you need to use calculus to do this?)}% End solution.



\subsection{Exercise \ref{sec : Differential Calculus Q2}}
\label{sec : Differential Calculus Q2}

% Reference : Math 111, Fall 2021, Final, Q5 and Q6

(4 pt) Compute the first derivative of the following function.
\begin{align*}
v(t)
=
t^{2} \ln(2 t^{3}) + \arctan(5 t) - \sin\left(\frac{\pi}{12}\right)
\end{align*}
(Assume the domain and codomain of $v$ are the largest possible subsets of the real numbers, $\reals$, for which the above rule of assignment make sense.) 

\spaceSolution{3in}{% Begin solution.
The derivative is linear, so we can differential $v$ term-by-term.

The first term of $v$ is a product of (nontrivial) functions of $t$, so we must use the product rule:
\begin{align*}
\frac{\intd}{\intd t}\left(t^{2} \ln(2 t^{3})\right)
&=
\frac{\intd}{\intd t}\left(t^{2}\right) \ln(2 t^{3}) + t^{2} \frac{\intd}{\intd t}\left(\ln(2 t^{3})\right)
\\
&=
(2 t) \ln(2 t^{3}) + t^{2} \left(\frac{1}{2 t^{3}} (6 t^{2})\right)
\\
&=
2 t \ln(2 t^{3}) + 3 t
\end{align*}
where in the second equality we use the chain rule on the second term.

The second term of $v$ also requires the chain rule (why?). Frankly, I don't remember the derivative of $y = \arctan x$, so I rewrite it as $\tan y = x$ and use implicit differentiation and elementary trigonometry, obtaining $y' = \frac{1}{x^{2} + 1}$. Thus
\begin{align*}
\frac{\intd}{\intd t}\left(\arctan(5 t)\right)
=
\frac{1}{(5 t)^{2} + 1} (5)
=
\frac{5}{25 t^{2} + 1}
\end{align*}

The third term of $v$ is a constant (no variable $t$!), so its derivative is $0$.

Combining these results, we conclude that
\begin{align*}
v'(t)
=
2 t \ln(2 t^{3}) + 3 t + \frac{5}{25 t^{2} + 1}
\end{align*}}% End solution.



\newpage



\subsection{Exercise \ref{sec : Differential Calculus Q3}}
\label{sec : Differential Calculus Q3}

% Reference : Math 111, Fall 2021, Exam 2, Q 3

(4 pt) Let $f : [2,+\infty) \rightarrow \reals$ be the function defined by
\begin{align*}
f(x)
=
\sqrt{2 x - 4}
\end{align*}
Using the limit definition of the derivative (!), compute $f'(x)$.

\spaceSolution{3in}{% Begin solution.
Fix $x \in [2,+\infty)$. Using the limit definition of the derivative (i.e. the limit of a difference quotient), we compute
\begin{align*}
f'(x)
=
\lim_{h \downarrow 0} \frac{f(x + h) - f(x)}{(x + h) - x}
=
\lim_{h \downarrow 0} \frac{\sqrt{2 x + 2 h - 4} - \sqrt{2 x - 4}}{h}
\end{align*}
Direct evaluation of this limit gives $\frac{0}{0}$, an indeterminate form. Let's rationalize the numerator to see if that helps:% Begin footnote.
\footnote{Can we apply l'H\^{o}pital's rule here? What do we need to ``know'' to do the computations l'H\^{o}pital's rule requires?}% End footnote.
\begin{align*}
\lim_{h \downarrow 0} \frac{\sqrt{2 x + 2 h - 4} - \sqrt{2 x - 4}}{h}
&=
\lim_{h \downarrow 0} \frac{\sqrt{2 x + 2 h - 4} - \sqrt{2 x - 4}}{h} \cdot \frac{\sqrt{2 x + 2 h - 4} + \sqrt{2 x - 4}}{\sqrt{2 x + 2 h - 4} + \sqrt{2 x - 4}}
\\
&=
\lim_{h \downarrow 0} \frac{2 h}{h \left(\sqrt{2 x + 2 h - 4} + \sqrt{2 x - 4}\right)}
\\
&=
\lim_{h \downarrow 0} \frac{2}{\left(\sqrt{2 x + 2 h - 4} + \sqrt{2 x - 4}\right)}
=
\frac{2}{2 \sqrt{2 x - 4}}
=
\frac{1}{\sqrt{2 x - 4}}
\end{align*}
(Can you show we get the same result if we compute $f'(x)$ using the ``rules'' for derivatives?)}% End solution.



\subsection{Exercise \ref{sec : Differential Calculus Q4}}
\label{sec : Differential Calculus Q4}

% Reference : Math 111, Fall 2021, Final, Q10

(4 pt) Compute the following limit.
\begin{align*}
\lim_{x \rightarrow +\infty} \frac{3 e^{x} + 5}{5 e^{x} + x + 1}
\end{align*}

\spaceSolution{3in}{% Begin solution.
Direct evaluation of the limit gives $\frac{+\infty}{+\infty}$, an indeterminate form. Divide both numerator and denominator by $e^{x}$, or equivalently, multiply both numerator and denominator by $e^{-x}$:
\begin{align*}
\lim_{x \rightarrow +\infty} \frac{3 e^{x} + 5}{5 e^{x} + x + 1} \cdot \frac{e^{-x}}{e^{-x}}
=
\lim_{x \rightarrow +\infty} \frac{3 + 5 e^{-x}}{5 + x e^{-x} + e^{-x}}
=
\frac{\displaystyle\lim_{x \rightarrow +\infty} \left(3 + 5 e^{-x}\right)}{\displaystyle\lim_{x \rightarrow +\infty} \left(5 + x e^{-x} + e^{-x}\right)}
\end{align*}
where the final equality uses limit laws (which? what is the exact statement?). In the numerator, direct evaluation gives
\begin{align*}
\lim_{x \rightarrow +\infty} \left(3 + 5 e^{-x}\right)
=
3 + 5 e^{-\infty}
=
3 + 5 (0)
=
3
\end{align*}
In the denominator, another use of limit laws (which?) and one application of l'H\^{o}pital's rule give
\begin{align*}
\lim_{x \rightarrow +\infty} \left(5 + x e^{-x} + e^{-x}\right)
=
5 + \lim_{x \rightarrow +\infty} \frac{1}{e^{x}} + 0
=
5 + 0 + 0
=
5
\end{align*}
which we note is nonzero. (Why might we note that?) We conclude that
\begin{align*}
\lim_{x \rightarrow +\infty} \frac{3 e^{x} + 5}{5 e^{x} + x + 1}
=
\frac{3}{5}
\end{align*}}% End solution.



\newpage



%
%   Section 3 : Integral calculus
%

\section{Integral calculus}

\subsection{Exercise \ref{sec : Integral Calculus Q1}}
\label{sec : Integral Calculus Q1}

% Reference : OpenStax, Calculus, Volume 1, Exercise 4.10.500

(4 pt) Solve the following initial value problem (i.e. find $f$ satisfying the following conditions).
\begin{align*}
f'(x)
&=
x^{2} + \sqrt{x}
&
&\text{such that}
&
f(0)
&=
2
\end{align*}

\spaceSolution{3in}{% Begin solution.
First we integrate $f'(x)$ to get a family of functions:
\begin{align*}
f(x)
=
\int f'(x) \spaceIntd \intd x
=
\int x^{2} + \sqrt{x} \spaceIntd \intd x
=
\frac{1}{3} x^{3} + \frac{2}{3} x^{\frac{3}{2}} + C
\end{align*}
where $C \in \reals$. This is the general solution, ignoring the initial value (aka boundary condition) $f(0) = 2$. To obtain the particular solution, we apply the initial value:
\begin{align*}
2
\seteq
f(0)
=
\frac{1}{3} 0^{3} + \frac{2}{3} 0^{\frac{3}{2}} + C
=
C
\end{align*}
Thus, the particular solution to the initial value problem is
\begin{align*}
f(x)
=
\frac{1}{3} x^{3} + \frac{2}{3} x^{\frac{3}{2}} + 2
\end{align*}
(Can you confirm that this function satisfies all parts of the initial value problem? What about uniqueness: Is this solution unique?)}% End solution.



\subsection{Exercise \ref{sec : Integral Calculus Q2}}
\label{sec : Integral Calculus Q2}

% Reference : OpenStax, Calculus, Volume 2, Exercise 1.2.112, modified

(4 pt) Let $f : [-2,2] \rightarrow \reals$ be the function defined by
\begin{align*}
f(x)
=
2 - \sqrt{4 - x^{2}}
\end{align*}
Find the average value of $f$ on the interval $[0,2]$. Then find a value $x_{0}$ of $x$ on this interval such that $f(x_{0})$ equals this average value. (Is such an $x_{0}$ guaranteed to exist? Why or why not? Is it unique?)

\spaceSolution{3in}{% Begin solution.
The average value (aka mean value) of $f$ on the interval $[0,2]$ is given by
\begin{align*}
\averageValue(f,[0,2])
=
\frac{1}{2 - 0} \int_{0}^{2} f(x) \spaceIntd \intd x
\end{align*}
The integral is messy to compute algebraically...but a breeze to compute geometrically! (Why? Sketch the graph of $f$...) Elementary euclidean geometry gives us
\begin{align*}
\averageValue(f,[0,2])
=
\frac{1}{2} \left(2 \cdot 2 - \frac{1}{4} \pi \cdot 2^{2}\right)
=
2 - \frac{\pi}{2}
\end{align*}
We find the requested $x_{0}$ --- guaranteed to exist by the mean value theorem --- by solving
\begin{align*}
2 - \frac{\pi}{2}
=
\averageValue(f,[0,2])
\seteq
f(x_{0})
=
2 - \sqrt{4 - x_{0}^{2}}
&&
\Rightarrow
&&
x_{0}
=
\pm{}\sqrt{4 - \frac{\pi^{2}}{4}}
\end{align*}
Only the positive square root lies in the interval $[0,2]$ (can you show this? without a calculator?), so $x_{0} = \sqrt{4 - \frac{\pi^{2}}{4}}$.}% End solution.



\newpage



\subsection{Exercise \ref{sec : Integral Calculus Q3}}
\label{sec : Integral Calculus Q3}

% Reference : OpenStax, Calculus, Volume 2, Exercise 1.3.158, modified

(4 pt) Compute the following derivative.
\begin{align*}
\frac{\intd}{\intd x} \int_{0}^{\ln x} e^{2 t} \spaceIntd \intd t
\end{align*}

\spaceSolution{3in}{% Begin solution.
By the fundamental theorem of calculus,% Begin footnote.
\footnote{The integrand, $e^{2 t}$, is continuous on all of $\reals$, so the fundamental theorem of calculus applies, at least on a subset of $(0,+\infty)$, the lower bound $0$ coming from the lower limit of integration of the original integral.}% End footnote.
\begin{align*}
\frac{\intd}{\intd x} \int_{0}^{\ln x} e^{2 t} \spaceIntd \intd t
=
e^{2 (\ln x)} \frac{\intd}{\intd x} \ln x - e^{2 (0)} \frac{\intd}{\intd x} 0
=
e^{\ln(x^{2})} \frac{1}{x} - 0
=
x
\end{align*}
(Can you show we get the same result, if we first compute the definite integral (as a function of $x$), then compute its derivative?)

Note that the original integral is defined only for $x \in (0,+\infty)$, so our final answer has the same constraint. That is, a more precise answer is
\begin{align*}
\frac{\intd}{\intd x} \int_{0}^{\ln x} e^{2 t} \spaceIntd \intd t
=
x,
\quad
x \in (0,+\infty)
\end{align*}
(Can you explain this constraint, geometrically?)}% End solution.



\subsection{Exercise \ref{sec : Integral Calculus Q4}}
\label{sec : Integral Calculus Q4}

% Reference : OpenStax, Calculus, Volume 2, Exercise 3.1.56

(4 pt) Compute the following indefinite integral.
\begin{align*}
\int x^{2} \sin x \spaceIntd \intd x
\end{align*}

\spaceSolution{3in}{% Begin solution.
One application of integration by parts (with $u = x^{2}$ and $\intd v = \sin x \spaceIntd \intd x$) gives
\begin{align}
\int x^{2} \sin x \spaceIntd \intd x
=
-x^{2} \cos x - \int -\cos x (2 x) \spaceIntd \intd x
=
-x^{2} \cos x + 2 \int x \cos x \spaceIntd \intd x%
\label{eq : LQuiz01 Integration By Parts Intermediate Solution}
\end{align}
Another integration by parts, on the second integral (with $u = x$ and $\intd v = \cos x \spaceIntd \intd x$), gives
\begin{align*}
\int x \cos x \spaceIntd \intd x
=
x \sin x - \int \sin x \spaceIntd \intd x
=
x \sin x + \cos x + C_{0}
\end{align*}
Substituting this result into \eqref{eq : LQuiz01 Integration By Parts Intermediate Solution}, we obtain
\begin{align*}
\int x^{2} \sin x \spaceIntd \intd x
=
-x^{2} \cos x + 2 x \sin x + 2 \cos x + C
\end{align*}
(Can you show that differentiating our answer indeed gives the original integrand?)}% End solution.



\newpage