%
%
%   LQuiz 05 : 2022-01-25 (T)
%
%

\section{Exercise}

% Reference : SHW

Consider the function
\begin{align*}
f : \reals \rightarrow \reals
&&
\text{given by}
&&
f(x)
=
x^{3} + 3 x^{2} - 9 x + 4
\end{align*}
\begin{enumerate}[label=(\alph*)]
\item\label{itm : LQ05 Linearization} (3 pt) Compute the linearization of (aka linear approximation to) $f$ at $x = 2$.
\end{enumerate}

\spaceSolution{4in}{% Begin solution.
By definition, the linearization of $f$ at $x = 2$ is the function $L : \reals \rightarrow \reals$ given by
\begin{align}
L(x)
=
f(2) + f'(2) (x - 2)%
\label{eq : LQ05 Linearization 0}
\end{align}
We compute
\begin{align*}
f(2)
=
6
&&
f'(x)
=
3 x^{2} + 6 x - 9
&&
f'(2)
=
15
\end{align*}
Substituting these results into \eqref{eq : LQ05 Linearization 0}, we conclude that the rule of assignment for $L$ is
\begin{align}
L(x)
=
6 + 15 (x - 2)
=
15 x - 24%
\label{eq : LQ05 Linearization 1}
\end{align}}% End solution.

\begin{enumerate}[resume,label=(\alph*)]
\item\label{itm : LQ05 Approximation Error} (1 pt) Use your linearization from part \ref{itm : LQ05 Linearization} to approximate the value $f(2.1)$. Given that the actual value is $f(2.1) = 7.591$, find the error in this approximation.
\end{enumerate}

\spaceSolution{2in}{% Begin solution.
Using \eqref{eq : LQ05 Linearization 1},% Begin footnote.
\footnote{It may be easier to compute with the first expression in \eqref{eq : LQ05 Linearization 1}, rather than the second, ``simplified'' one.} % End footnote.
we compute
\begin{align*}
L(2.1)
=
7.5
\end{align*}
The error $\varepsilon$ in this approximation is% Begin footnote.
\footnote{The negative sign simply indicates that our estimate $L(2.1)$ is below the actual value $f(2.1)$. Taking the absolute value gives the magnitude (size) of the error, without its direction (too small or too large).}% End footnote.
\begin{align*}
\varepsilon
=
L(2.1) - f(2.1)
=
7.5 - 7.591
=
-0.091
\end{align*}
(Question: Can we predict, with accuracy, the direction of the approximation error? I claim that in this exercise, we can use certain features of $f$ to deduce with certainty that our approximation will be too small, i.e. that $L(2.1) < f(2.1)$. How?)}% End solution.