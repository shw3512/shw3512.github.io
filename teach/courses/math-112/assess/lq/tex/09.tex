%
%
%   LQuiz 09 : 2022-02-17 (R)
%
%

\section{Exercise}

% Reference : SHW

(4 pt) Compute the following limits. Justify your steps.

\begin{enumerate}[label=(\alph*)]
\item\label{itm : LQ09a} (2 pt) $\displaystyle\lim_{x \rightarrow 0} \frac{e^{3 x^{2}} - 1}{x^{2}}$
\end{enumerate}

\spaceSolution{3in}{% Begin solution.
Direct evaluation of the limit gives
\begin{align*}
\frac{e^{3 (0)^{2}} - 1}{(0)^{2}}
=
\frac{e^{0} - 1}{0}
=
\frac{0}{0}
\end{align*}
an indeterminate form to which l'H\^{o}pital's rule applies. Using it, we compute
\begin{align*}
\lim_{x \rightarrow 0} \frac{e^{3 x^{2}} - 1}{x^{2}}
=
\lim_{x \rightarrow 0} \frac{6 x e^{3 x^{2}} - 0}{2 x}
=
\lim_{x \rightarrow 0} \frac{3 e^{3 x^{2}} - 0}{1}
=
3
\end{align*}}% End solution.



\begin{enumerate}[resume,label=(\alph*)]
\item\label{itm : LQ09b} (2 pt) $\displaystyle\lim_{x \rightarrow 0} \frac{\sqrt{1 + x^{2}} - \sqrt{1 - x^{2}}}{x^{2}}$
\end{enumerate}

\spaceSolution{3in}{% Begin solution.
Direct evaluation of the limit gives
\begin{align*}
\frac{\sqrt{1 + (0)^{2}} - \sqrt{1 - (0)^{2}}}{(0)^{2}}
=
\frac{0}{0}
\end{align*}
an indeterminate form to which l'H\^{o}pital's rule applies. L'H\^{o}pital's rule will work. However, it requires differentiating the numerator, which looks messy. Is there another approach?

Let's try multiplying numerator and denominator by the conjugate of the numerator:% Begin footnote.
\footnote{Why might we think conjugates will work here? Multiplying an expression by its conjugate squares the two terms and puts a minus sign between them. The squares will cancel the square roots, and the minus sign will cancel the constants, leaving us with (some coefficient times) $x^{2}$. This, in turn, will cancel the $x^{2}$ in the denominator. That should be nice...}% End footnote.
\begin{align*}
\lim_{x \rightarrow 0} \frac{\sqrt{1 + x^{2}} - \sqrt{1 - x^{2}}}{x^{2}}
&=
\lim_{x \rightarrow 0} \frac{\sqrt{1 + x^{2}} - \sqrt{1 - x^{2}}}{x^{2}} \cdot \frac{\sqrt{1 + x^{2}} + \sqrt{1 - x^{2}}}{\sqrt{1 + x^{2}} + \sqrt{1 - x^{2}}}
\\
&=
\lim_{x \rightarrow 0} \frac{(1 + x^{2}) - (1 - x^{2})}{x^{2} \left(\sqrt{1 + x^{2}} + \sqrt{1 - x^{2}}\right)}
\\
&=
\lim_{x \rightarrow 0} \frac{2 x^{2}}{x^{2} \left(\sqrt{1 + x^{2}} + \sqrt{1 - x^{2}}\right)}
\\
&=
\lim_{x \rightarrow 0} \frac{2}{\left(\sqrt{1 + x^{2}} + \sqrt{1 - x^{2}}\right)}
\\
&=
\frac{2}{1 + 1}
=
1
\end{align*}

Try applying l'H\^{o}pital's rule here. (It's messy, but not that messy.) You will get the same result. Compare the effort.}% End solution.