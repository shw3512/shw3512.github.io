%
%
%   LQuiz 05B06B : 2022-02-13 (N)
%
%

\section{Exercise}

% Reference : SHW

(4 pt) Consider the function
\begin{align*}
f : \reals \rightarrow \reals
&&
\text{given by}
&&
f(x)
=
e^{-2 x} + x^{3} + 4 x + 3
\end{align*}
\begin{enumerate}[label=(\alph*)]
\item\label{itm : LQ05B06B Linearization} (3 pt) Compute the linearization of (aka linear approximation to) $f$ at $x = 0$.
\end{enumerate}

\spaceSolution{4in}{% Begin solution.
By definition, the linearization of $f$ at $x = 0$ is the function $L : \reals \rightarrow \reals$ given by
\begin{align}
L(x)
=
f(0) + f'(0) (x - 0)%
\label{eq : LQ05B06B Linearization 0}
\end{align}
We compute
\begin{align*}
f(0)
&=
e^{-2 (0)} + 0 + 0 + 3
=
4
\\
f'(x)
&=
-2 e^{-2 x} + 3 x^{2} + 4
\\
f'(0)
&=
-2 e^{-2 (0)} + 0 + 4
=
2
\end{align*}
Substituting these results into Equation \eqref{eq : LQ05B06B Linearization 0}, we find that the rule of assignment for $L$ is
\begin{align*}
L(x)
=
4 + 2 x
\end{align*}}% End solution.



\begin{enumerate}[resume,label=(\alph*)]
\item\label{itm : LQ05B06B Approximation Error} (1 pt) Use your linearization from part \ref{itm : LQ05B06B Linearization} to approximate the value $f(1)$. Find the error in this approximation. (Where relevant, you may leave computations in terms of $e$.)
\end{enumerate}

\spaceSolution{2in}{% Begin solution.
We compute
\begin{align*}
L(1)
=
4 + 2 (1)
=
6
&&
\text{and}
&&
f(1)
=
e^{-2 (1)} + (1)^{3} + 4 (1) + 3
=
8 + e^{-2}
\end{align*}
The error $\varepsilon$ in the approximation $L(1)$ to $f(1)$ is
\begin{align*}
\varepsilon
=
L(1) - f(1)
=
6 - (8 + e^{-2})
=
-2 - e^{-2}
\end{align*}
This error is negative, indicating that our estimate $L(1)$ is less than the actual value $f(1)$. That is, our estimate is too small.}% End solution.