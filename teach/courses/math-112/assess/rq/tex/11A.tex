%
%
%   ReQuiz 11A : 2022--04--08 (F)
%
%

\section{Exercise}

% Reference : SHW

(4 pt) Take as given the following ``infinite polynomial'' expression for $\ln(x + 1)$:
\begin{align}
\ln(x + 1)
&=
x - \frac{1}{2} x^{2} + \frac{1}{3} x^{3} - \frac{1}{4} x^{4} + \ldots%
\nonumber
\\
&=
x - \frac{1}{2} x^{2} + \frac{1}{3} x^{3} - O(x^{4})%
\label{eq : RQ11A lns x Taylor Series}
\end{align}
(Recall that $O(x^{n})$ means ``terms involving $x$ to powers $n$ and higher''.) Consider the limit
\begin{align}
\lim_{x \rightarrow 0} \frac{\frac{1}{3} x^{3}}{\ln(x + 1) - x + \frac{1}{2} x^{2}}%
\label{eq : RQ11A Limit}
\end{align}

\begin{enumerate}[label=(\alph*)]
\item\label{itm : RQ11Aa} (2 pt) Compute the limit in \eqref{eq : RQ11A Limit} by substituting \eqref{eq : RQ11A lns x Taylor Series} for $\ln(x + 1)$, simplifying, and evaluating.
\end{enumerate}

\spaceSolution{3in}{% Begin solution.
.}% End solution.



\begin{enumerate}[resume,label=(\alph*)]
\item\label{itm : RQ11Ab} (2 pt) Compute the limit in \eqref{eq : RQ11A Limit} by iteratively applying l'H\^{o}pital's rule. (Briefly show it applies each time you use it!) Confirm you get the same result you got in part \ref{itm : RQ11Aa}.
\end{enumerate}

\spaceSolution{3in}{% Begin solution.
}% End solution.