\section{How to Read Math Effectively}

Effective reading is engaged.
\begin{itemize}
\item \textbf{Read focused.} Proactively manage distractions. If the situation allows, turn off your cell phone and (!) put it out of sight. Commit to what's before you.
\item \textbf{Read prepared.} Look up the authors. Read the \href{https://openstax.org/books/calculus-volume-2/pages/preface}{preface}. Ask yourself, ``What do the authors think that I already know? Do I know it? What do the authors want me to learn? (Do I want to learn it?)''
\item \textbf{Read slowly.} The goal is not to get to the end of the assigned reading. The goal is to engage the concepts, to structure your thought in new and useful ways. This takes time, and often repetition. Give yourself both. You're worth it.
\item \textbf{Read deliberately.} Before you read, ask yourself, ``What do I want to learn from this section?'' (The assigned exercises may help you formulate an answer.) After you read, ask yourself, ``What were the big ideas? key results? new techniques? What do I want to clarify? to know more about?''
\item \textbf{Read actively.} When you get to an example, work it out on your own. Refer to the textbook (i) for hints if you get stuck and stay stuck (don't give up immediately!), and (ii) to check your results along the way. When you get to the end of a proof, write it out on your own. When you can do the examples and proofs without outside aid, when you can explain them to other people, you have grown.
\item \textbf{Read forward.} If you don't understand something new, keep reading a bit. What's confusing you may be explained in the next few paragraphs.
\item \textbf{Read backward.} If you don't understand something old, go back and reread about it. What confused you previously may be clearer from your new vantage point.
\item \textbf{Read sideways.} Use outside references (e.g., other textbooks, \href{https://www.wikipedia.org/}{Wikipedia}, \href{http://tutorial.math.lamar.edu/}{Paul's online math notes}) to address questions that arise and to provide alternative views.
\item \textbf{Write.} Your textbook, your rules. I write all over mine. Explanatory footnotes, computations in the margins, critiques, corrections, comics --- make it yours. %(Yes, this probably drives down the book's resale value, but it makes reading way more lively, engaging, and personal. Plus, if you become famous, your snarky notes and silly doodles will enhance the book's value.)
Do not just passively read and accept --- actively \href{https://www.youtube.com/watch?v=tpeLSMKNFO4}{think for yourself}.
\end{itemize}