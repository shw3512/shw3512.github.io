% https://ga.rice.edu/syllabus/
% https://registrar.rice.edu/students/syllabus/
% http://cte.rice.edu/syllabus/



\begin{center}
{\Large Math 112 : Calculus : Integration and its applications}\\
{\scriptsize (draft version : last updated : \Year--\Month--\Day)}
\end{center}





\section{Disclaimer}

The information contained in this syllabus, other than the absence policy, is subject to change with reasonable advance notice.



%
%
%	Course information
%
%

\section{Course Information}



\subsection{Course Logistics}

\begin{tabular}{r c l}
Class time 	&	:	&	Tuesday \& Thursday, 09h25 -- 10h40	\\
Class room	&	:	&	\href{https://goo.gl/maps/CfwNovWqwqMdsBUV6}{George R. Brown Hall} (GRB) W211	\\
Office hours	&	:	&	Tuesday, 12h00 -- 13h30, Herman Brown Hall (HBH) 208	\\
Recitation		&	:	&	Sunday, 13h30 -- 15h00, Humanities Building (HUM) 118	\\
TA sessions	&	:	&	Monday, 18h00 -- 20h00, Maxfield 252		\\
			&		&	Tuesday, 18h00 -- 20h00, Brockman 101		\\
			&		&	Wednesday, 19h00 -- 20h00, Maxfield 252	\\
			&		&	Thursday, 18h00 -- 20h00, Brockman 101
\end{tabular}



\subsection{Instructor Information}


\begin{tabular}{r c l}
Instructor	&	:	&	Stephen Wolff	\\
Office 	&	:	&	HBH 208	\\
E-mail	&	:	&	\href{mailto:Stephen.Wolff@rice.edu?subject=[Math\%20112]}{Stephen.Wolff@rice.edu}
\end{tabular}



\subsection{Textbook}

This course is based on the textbook
\begin{quote}
OpenStax : \fontBookTitle{Calculus}, \href{https://openstax.org/details/books/calculus-volume-1}{Volume 1} and \href{https://openstax.org/details/books/calculus-volume-2}{Volume 2}, by Herman, Strang, et al.
\end{quote}
The books are available free of charge at the links above (URLs below% Begin footnote.
\footnote{The URLs for Volume 1 and Volume 2, respectively, are
\begin{enumerate}
\item \url{https://openstax.org/details/books/calculus-volume-1}
\item \url{https://openstax.org/details/books/calculus-volume-2}
\end{enumerate}}% End footnote.
).

We will wrestle with the following topics (``V$v$.$c$'' denotes Volume $v$, Chapter $c$):
\begin{multicols}{2}
\begin{itemize}
\item V1.2. Limits
\item V1.3. Derivatives
\item V1.4. Applications of derivatives
\item V2.1. Integration
\item V2.2. Applications of integration
\item V2.3. Techniques of integration
%\item V2.5. Sequences and series
\end{itemize}
\end{multicols}





\subsection{Absence Policy}

Class attendance is strongly encouraged. We will take and accept responsibility for our actions and decisions.





%
%
%	Grading policy
%
%

%\newpage
\subsection{Grading Policy}

Our goal in this course will be mastery of the concepts we engage. We will promote and assess mastery using the following tools:
\begin{enumerate}
\item \textbf{Homework (H).} There will be homework every day. All resources are allowed. Homework is due promptly by 09h30 the next class. Late homework will not be accepted. Homework will be graded 0--2 using a ``reasonable attempt'' rubric. Your lowest 3 homework scores will be dropped.
\item \textbf{Quizzes (Q).} There will be an in-class quiz every day. No external resources are allowed. Quizzes come in two flavors: short quiz (SQ) and long quiz (LQ). SQs are about 3 minutes on key concepts. SQs will be graded 0--2 and count as bonus. LQs are about 10 minutes on a (new) exercise. LQs will be graded 0--4. Some quizzes you will be able to discuss with your colleagues before submitting. Others will be solo. You may ``re-quiz'' LQs throughout the semester. Only your highest grade on each LQ sequence will count toward your course grade.
\item \textbf{Exams (E).} There will be three midterm exams (ME) and one final exam (FE). No external resources are allowed. Each exam will be cumulative. The final exam will have three sections, with content corresponding to that of the three MEs. For each section of the FE, if your score on that section is higher than your score on the corresponding ME, then your FE score (for that section) will replace your ME score. See page \pageref{tab : Calendar} for exam dates.
\item \textbf{Communication (C).} There will be a team project (TP) and an oral exam (OE). For the team project, you will solve past exam questions, exchange feedback with peers on your solutions, and typeset your revised solutions. The final week of classes, each student will have a 15-minute oral exam (a scary term for ``math chat'').
\end{enumerate}

\noindent{}\textbf{Course grade.}
Your course grade is allocated among the above tools as follows:
\begin{center}
{} \hfill{} H : 10\% \hfill{} LQ : 20\% \hfill{} each ME : 10\% \hfill{} FE : 30\% \hfill{} TP : 8\% \hfill{} OE : 2\% \hfill{} SQ : 2\% bonus \hfill{} {}
\end{center}
Example. Suppose that prior to the final exam, you submit all homework, earn 80\% on long quizzes and 100\% on short quizzes, score 80\% on each midterm exam, and earn 80\% in communication (team project and oral exam). If you score 70\% on each section of the final exam, then your course grade is
\begin{align*}
\underbrace{100\% \times 10 \%}_{H} + \underbrace{80\% \times 20 \%}_{LQ} + \underbrace{3 \times \left(80\% \times 10 \%\right)}_{ME} + \underbrace{70\% \times 30 \%}_{FE} + \underbrace{80\% \times 10 \%}_{C} + \underbrace{100\% \times 2 \%}_{SQ}
=
81 \%
\end{align*}
If you score 90\% on each section of the final exam, then your course grade is
\begin{align*}
\underbrace{100\% \times 10 \%}_{H} + \underbrace{80\% \times 20 \%}_{LQ} + \underbrace{3 \times \left(90\% \times 10 \%\right)}_{ME} + \underbrace{90\% \times 30 \%}_{FE} + \underbrace{80\% \times 10 \%}_{C} + \underbrace{100\% \times 2 \%}_{SQ}
=
90 \%
\end{align*}





%
%
%	Course objectives
%
%

%\newpage

\section{Course Objectives and Expected Learning Outcomes}

Our goal, by the end of this course, is to be able to accurately and confidently
\begin{itemize}
\item Assess what we know and what we don't yet know.
\item Communicate math clearly and effectively with others.
\item Perform fundamental math operations (e.g., simplifying and evaluating algebraic expressions, using functions, graphing in $\reals^{2}$, using logic).
\item Use and evaluate limits.
\item Use and evaluate derivatives.
\item Perform implicit differentiation and apply it to problems involving related rates.
\item Use l'H\^{o}pital's rule, and know when not to use it.
\item Link algebraic, geometric, and physical views of derivatives and integrals.
\item Define and use antiderivatives (aka indefinite integrals).
\item Define and use definite integrals.
\item State and use the fundamental theorem of calculus.
\item Use integration techniques, including substitution and integration by parts.%, and trig substitution.%, and partial-fraction decomposition.
\item Use and evaluate improper integrals.
\item Use Taylor series, and understand why they are useful.
%\item Use the logic of the divergence and integral tests.
\end{itemize}





%
%
%	Resources
%
%

\section{Resources}

Potentially helpful resources:
\begin{itemize}
\item Past exams are hosted on the \href{https://canvas.rice.edu/courses/9592/files}{Calculus Resources page} on Canvas. See the page ``\href{https://math.rice.edu/exam-help}{Math Exam Help}'' for log-in instructions. (You will need a valid Rice NetID.)
\item Solutions to selected exercises from the OpenStax textbooks are hosted on the OpenStax website, under the Student Resources tab for each book. (Here are links to the tabs for \href{https://openstax.org/details/books/calculus-volume-1?Student\%20resources}{Volume 1} and \href{https://openstax.org/details/books/calculus-volume-2?Student\%20resources}{Volume 2}.)
\end{itemize}





\newpage

%
%
%	Disabilities
%
%

\section{Students with Disabilities}

Any student with a documented disability that requires accommodation is encouraged to contact both the course instructor (\href{mailto:Stephen.Wolff@rice.edu?subject=[Math\%20112]}{Stephen.Wolff@rice.edu}) and the \href{https://drc.rice.edu/}{Rice Disability Resource Center} (\href{mailto:adarice@rice.edu}{adarice@rice.edu}; Allen Center, Room 111).





%
%
%	Honor code
%
%

\section{Rice Honor Code}
\label{sec : Honor Code}

As a student at Rice University, you pledge to uphold the Rice Honor Code, which you can find in the \href{http://honor.rice.edu/honor-system-handbook/}{Honor System Handbook}.

On homework, all resources are allowed. In particular, you are strongly encouraged to work with one another. The purpose of homework is to help you to learn, practice, and internalize concepts.

On quizzes and exams, no external resources are allowed, unless the instructor explicitly indicates otherwise. The purpose of quizzes and exams is to help you to see what you can do so far and identify what you want to work on.





%
%
%	Inclusivity
%
%

%\section{Inclusivity Statement}

%\fontNeedsEdit{required wording?}





%
%
%	Religious accommodations
%
%

\section{Religious Accommodations}

Every reasonable effort will be made to allow members of the university community to observe their religious holidays without jeopardizing the fulfillment of their academic obligations. Absence from classes or examinations for religious reasons does not relieve students from responsibility for any part of the course work required during the period of absence. It is the obligation of students to provide faculty with reasonable notice of the dates of religious holidays on which they will be absent. 





%
%
%	Title IX
%
%

\section{Title IX Statement}

Rice University cares about your wellbeing and safety. Rice encourages any student who has experienced an incident of harassment; pregnancy or gender discrimination; or relationship, sexual, or other forms interpersonal violence to seek support through the SAFE Office. Please be aware, when seeking support on campus, that most employees (including myself, as an instructor) are required by Title IX to disclose all incidents of non-consensual interpersonal behaviors to Title IX professionals on campus, who can act to support students and meet their needs. For more information, please visit \href{https://safe.rice.edu/}{safe.rice.edu} or e-mail \href{mailto:titleixsupport@rice.edu}{titleixsupport@rice.edu}.





%
%
%	Calendar
%
%

%\newpage

\section{Calendar}
\label{sec : Calendar}

Below is a preliminary schedule of topics. Section numbers refer to pdf versions of OpenStax : \fontBookTitle{Calculus}, \href{https://openstax.org/details/books/calculus-volume-1}{Volume 1} (V1) or \href{https://openstax.org/details/books/calculus-volume-2}{Volume 2} (V2).% Exercises are \emph{assigned} on the date of the line on which they appear and are \emph{due} the next class.
\begin{landscape}
\thispagestyle{empty}
\begin{table}
\caption{Math 112 : Spring 2022 : Preliminary schedule. (``V$v$:$c$.$s$'' denotes Volume $v$, Chapter $c$, Section $s$.)}
\label{tab : Calendar}
\centering{% Begin centering.
\begingroup%
%\resizebox{\textwidth}{!}{%
%\scriptsize
\footnotesize
\begin{tabular}{*2{c}*4{l}}
\hline\hline
Week			&	Day	&	Date		&	Topics						&	Sections	&	Special HW	\\
\hline
\multirow{2}{*}{01}	&	T	&	11 Jan	&	Intro. Diagnostic quiz. Review.		&	V1:1.1--5	&		\\
				&	R	&	13 Jan	&	Limits and continuity.				&	V1:2.2--4	&		\\
\hline
\multirow{2}{*}{02}	&	T	&	18 Jan	&	Derivatives as rates of change.		&	V1:3.2--4	&		\\
				&	R	&	20 Jan	&	Differentiation. Linearization.		&	V1:3.3,4.2	&		\\
\hline
\multirow{2}{*}{03}	&	T	&	25 Jan	&	Implicit differentiation.			&	V1:3.8	&		\\
				&	R	&	27 Jan	&	Implicit differentiation.			&	V1:3.8	&		\\
\hline
\multirow{2}{*}{04}	&	T	&	01 Feb	&	Implicit differentiation. Related rates.	&	V1:4.1	&	Mock exam 1	\\
				&	R	&	03 Feb	&	Related rates. Optimization.		&	V1:4.1,7	&		\\
\hline
\multirow{2}{*}{05}	&	\fontExam{T}	&	\fontExam{08 Feb}	&	\fontExam{Midterm exam 1.}	&	$\leq$V1:4.7	&		\\
				&	\multicolumn{2}{c}{\fontHoliday{10--11 Feb}}	&	\multicolumn{2}{l}{\fontHoliday{Spring recess --- no class.}}	&	\\
\hline
\multirow{2}{*}{06}	&	T	&	15 Feb	&	Limits. L'H\^{o}pital's rule.			&	V1:2.3,4.8	&		\\
				&	R	&	17 Feb	&	L'H\^{o}pital's rule.				&	V1:4.8	&		\\
\hline
\multirow{2}{*}{07}	&	T	&	22 Feb	&	L'H\^{o}pital's rule.				&	V1:4.8	&	Exam 1 corrections	\\
				&	R	&	24 Feb	&	Antiderivatives.					&	V1:4.10	&		\\
\hline
\multirow{2}{*}{08}	&	T	&	01 Mar	&	Antiderivatives.					&	V1:4.10	&		\\
				&	R	&	03 Mar	&	Approximating area.				&	V2:1.1	&	Mock exam 2	\\
\hline
\multirow{2}{*}{09}	&	T	&	08 Mar	&	Definite integral. Review.			&	V2:1.2	&		\\
				&	\fontExam{R}	&	\fontExam{10 Mar}	&	\fontExam{Midterm exam 2.}	&	$\leq$V2:1.2	&		\\
\hline
\fontHoliday{10}	&	\multicolumn{2}{c}{\fontHoliday{12--20 Mar}}	&	\multicolumn{2}{l}{\fontHoliday{Spring break --- no class.}}	&		\\
\hline
\multirow{2}{*}{11}	&	T	&	22 Mar	&	Fundamental theorem of calculus.	&	V2:1.3	&	Exam 2 corrections	\\
				&	R	&	24 Mar	&	Integration formulas.				&	V2:1.4	&		\\
\hline
\multirow{2}{*}{12}	&	T	&	29 Mar	&	Substitution.					&	V2:1.5	&		\\
				&	R	&	31 Mar	&	Area between curves.			&	V2:2.1	&		\\
\hline
\multirow{2}{*}{13}	&	T	&	05 Apr	&	Volumes of rotation.				&	V2:2.2--3	&		\\
				&	R	&	07 Apr	&	Integration by parts.				&	V2:3.1	&	Mock exam 3	\\
\hline
\multirow{2}{*}{14}	&	T	&	12 Apr	&	Improper integrals. Review.		&	V2:3.7	&		\\
				&	\fontExam{R}	&	\fontExam{14 Apr}	&	\fontExam{Midterm exam 3.}	&	all	&		\\
\hline
\multirow{4}{*}{15}	&	\fontExam{?}	&	\fontExam{TBD}	&	\fontExam{Oral exams.}	&		&		\\
				&	T	&	19 Apr	&	Review.						&		&	Exam 3 corrections	\\
				&	R	&	21 Apr	&	Review.						&		&		\\
				&	F	&	22 Apr	&	Due : Team project.				&		&		\\
\hline
16		&	\fontExam{S}	&	\fontExam{30 Apr}	&	\fontExam{9h00 : DCH 1042 : Final exam.}	&	all	&	\\
\hline
\end{tabular}
\endgroup
}% End centering.
\end{table}
\end{landscape}