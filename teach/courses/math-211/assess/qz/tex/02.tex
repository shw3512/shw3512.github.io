%
% Quiz 03 : 2019-07-10 (W)
% Blanchard et al. Chapter 1 Review Exercise 39 : p 139, solution p 760
%
\section{Exercise}


(2 pt) Consider the 1st-order linear ODE
\begin{align}
\frac{\intd y}{\intd t}
=
2 t y + 3 t e^{t^{2}}.%
\label{eq : Quiz 02 ODE}
\end{align}
Confirm that the function% Begin footnote.
\footnote{This notation says that $y$ is a function with domain (input) all real numbers, codomain (output) real numbers, and rule of assignment given by
\begin{align*}
y(t)
=
\left(\frac{3}{2} t^{2} + 1\right) e^{t^{2}}.
\end{align*}}% End footnote.
\begin{align*}
y
:
\reals
&\rightarrow
\reals
\\
t
&\mapsto
\left(\frac{3}{2} t^{2} + 1\right) e^{t^{2}}
\end{align*}
is a solution to \eqref{eq : Quiz 02 ODE}, and that the solution satisfies the initial condition $y(0) = 1$.

\spaceSolution{3in}{% Begin solution.
Let's do the easy part first: Check the initial condition. We evaluate
\begin{align*}
y(0)
=
\left(\frac{3}{2} 0^{2} + 1\right) e^{0^{2}}
=
(0 + 1) 1
=
1,
\end{align*}
as required by the initial condition. Next, we check that $y(t)$ is a solution to \eqref{eq : Quiz 02 ODE}, by plugging it into both sides of the ODE. On the left side, we compute the derivative using the product rule and chain rule:
\begin{align*}
\frac{\intd y}{\intd t}
=
(3 t) e^{t^{2}} + \left(\frac{3}{2} t^{2} + 1\right) (2 t) e^{t^{2}}.
\end{align*}
On the right side, we plug in $y(t)$:
\begin{align*}
2 t \left(\frac{3}{2} t^{2} + 1\right) e^{t^{2}} + 3 t e^{t^{2}}.
\end{align*}
These expressions are the same,% Begin footnote.
\footnote{Both expressions simplify to
\begin{align*}
t e^{t^{2}} \left(3 t^{2} + 5\right).
\end{align*}} % End footnote.
i.e. $y(t)$ satisfies equation \eqref{eq : Quiz 02 ODE}, so by definition it is a solution.}% End solution.