%
% Quiz 05 : 2019-07-12 (F)
% For a discussion of the function f(t,y), see https://math.stackexchange.com/questions/1391544/differentiable-but-not-continuously-differentiable
%
\section{Exercise}

(2 pt) Consider the first-order IVP
\begin{align*}
y'
=
t \sqrt[3]{y} + \cos t^{2},
&&
y(1)
=
0.
\end{align*}
\begin{enumerate}[label=(\alph*)]
\item (1 pt) What, if anything, can we conclude about existence of solutions to this IVP? Justify briefly.
\end{enumerate}

\spaceSolution{3in}{% Begin spaceSolution.
The expression for $y'$ is continuous, so Picard's theorem implies that there exists a solution (i.e. at least one solution) to this IVP on some $t$-interval around $t = 1$.}% End spaceSolution.



\begin{enumerate}[resume,label=(\alph*)]
\item (1 pt) What, if anything, can we conclude about uniqueness of solutions to this IVP? Justify briefly.
\end{enumerate}

\spaceSolution{3in}{% Begin spaceSolution.
We compute
\begin{align*}
\frac{\partial (y')}{\partial y}
=
\frac{\partial}{\partial y}\left(t \sqrt[3]{y} + \cos t^{2}\right)
=
\frac{t}{3 \sqrt[3]{y^{2}}}.
\end{align*}
This function is not defined at $y = 0$, therefore it cannot be continuous at the point $(t,y) = (1,0)$. The uniqueness statement of Picard's theorem requires that this partial derivative be continuous, so we can't use Picard's theorem to conclude anything about uniqueness.}% End spaceSolution.