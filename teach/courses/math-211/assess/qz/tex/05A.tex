\section{Exercise}

% For a discussion of the function f(t,y), see https://math.stackexchange.com/questions/1391544/differentiable-but-not-continuously-differentiable

(2 pt) Consider the first-order IVP
\begin{align*}
y'
=
f(t,y),
&&
y(1)
=
0,
\end{align*}
where
\begin{align*}
f(t,y)
=
\begin{dcases*}
y^{2} \sin\left(\frac{1}{y}\right)	&	if $y \neq 0$,	\\
0						&	if $y = 0$.
\end{dcases*}%
.
\end{align*}
\begin{enumerate}[label=(\alph*)]
\item (1 pt) A friend tells you that $f(t,y)$ is differentiable at all $(t,y)$. Assuming that she's right, what, if anything, can we conclude about existence of solutions to this IVP? Justify briefly.
\end{enumerate}

\spaceSolution{2in}{% Begin spaceSolution.
If a function is differentiable, then it is continuous. If our friend is right, then $f(t,y)$ is continuous, so Picard's theorem implies that there exists a solution to this IVP on some $t$-interval around $t = 1$.}% End spaceSolution.



\begin{enumerate}[resume,label=(\alph*)]
\item (1 pt) Your friend tells you that $\frac{\partial f}{\partial y}$ is not continuous at $(1,0)$. Assuming that she's right, what, if anything, can we conclude about uniqueness of solutions to this IVP? Justify briefly.
\end{enumerate}

\spaceSolution{2in}{% Begin spaceSolution.
The uniqueness statement in Picard's theorem requires that $\frac{\partial f}{\partial y}$ be continuous. Thus, if our friend is right, we can't use Picard's theorem to conclude anything about uniqueness.}% End spaceSolution.



\begin{enumerate}[resume,label=(\alph*)]
\item (\Heart{} pt) Can you justify or refute your friend's assertions about $f$?
\end{enumerate}

\spaceSolution{2in}{% Begin spaceSolution.
For $y \neq 0$, we compute
\begin{align*}
\frac{\partial f}{\partial y}
=
\frac{\intd f}{\intd y}
=
2 y \sin\left(\frac{1}{y}\right) + y^{2} \cos\left(\frac{1}{y}\right) \left(-y^{-2}\right)
=
2 y \sin\left(\frac{1}{y}\right) - \cos\left(\frac{1}{y}\right).
\end{align*}
This function is well defined for all $y \neq 0$, so $f$ is well defined}% End spaceSolution.