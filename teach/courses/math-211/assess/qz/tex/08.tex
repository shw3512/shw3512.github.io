%
% Quiz 08 : 2019-07-17 (W)
%
\section{Exercise}

(2 pt) Consider the 1st-order autonomous ODE
\begin{align}
\frac{\intd y}{\intd t}
=
-(y + 1) (y - 1)^{2} (y - 3).%
\label{eq : Quiz 08 ODE}
\end{align}
Identify the equilibrium solutions, and classify the stability of each. (Mind the leading negative sign in this ODE!)

\spaceSolution{5in}{%Begin solution.
By definition, equilibrium solutions are constant functions $y(t) \equiv c$ that solve the given ODE. For the ODE \eqref{eq : Quiz 08 ODE}, the equilibrium solutions are $y \equiv -1$, $y \equiv 1$, and $y \equiv 3$. By using the ODE \eqref{eq : Quiz 08 ODE} to compute the sign of the slope (e.g., for each factor, then multiplying all the results to get the slope $\frac{\intd y}{\intd t}$; or computing the value of \eqref{eq : Quiz 08 ODE} all at once using $y$-values in the open intervals cut out by the equilibrium solutions), and being careful with the coefficient $-1$ in \eqref{eq : Quiz 08 ODE}, we get the phase line\fontNeedsEdit{ (insert graphic of phase line)}
\begin{align*}
-\infty
&&
(-)
&&
-1
&&
(+)
&&
1
&&
(+)
&&
3
&&
(-)
&&
+\infty.
\end{align*}
We conclude that the equilibrium solutions and their stability are
\begin{itemize}
\item $y \equiv -1$ : unstable
\item $y \equiv 1$ : semistable
\item $y \equiv 3$ : stable
\end{itemize}}% End solution.