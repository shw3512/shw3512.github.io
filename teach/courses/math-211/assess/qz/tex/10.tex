%
% Quiz 10 : 2019-07-19 (F)
%
\section{Exercise}

(5 pt) Consider the matrix
\begin{align*}
A
=
\begin{bmatrix}
0	&	0	&	1	\\
1	&	2	&	0	\\
-2	&	-2	&	1
\end{bmatrix}%
.
\end{align*}
\begin{enumerate}[label=(\alph*)]
\item\label{itm : Quiz10 Part a} (2 pt) Show that $\det A = 2$. \fontHint{Use expansion by minors along row 1. (Why?)}
\end{enumerate}

\spaceSolution{1.5in}{% Begin solution.
Using expansion by minors along row 1 (which we choose because row 1 contains many zeros, which will simplify our work), we compute
\begin{align*}
\det A
=
0 + 0 + (-1)^{1 + 3} (1) \det
\begin{bmatrix}
1	&	2	\\
-2	&	-2
\end{bmatrix}
=
(1) (1) (1 (-2) - 2 (-2))
=
2,
\end{align*}
where we use the fact that the determinant of a $2 \times 2$ matrix is
\begin{align*}
\det
\begin{bmatrix}
a	&	b	\\
c	&	d
\end{bmatrix}
=
a d - b c.
\end{align*}
Note: Expansion by minors along any row or column will yield this same result. To illustrate, consider expansion by minors along column 2. We get
\begin{align*}
\det A
&=
0 + (-1)^{2 + 2} (2) \det
\begin{bmatrix}
0	&	1	\\
-2	&	1
\end{bmatrix}
+ (-1)^{3 + 2} (-2) \det
\begin{bmatrix}
0	&	1	\\
1	&	0
\end{bmatrix}
\\
&=
0 + (1) (2) (0 (1) - 1 (-2)) + (-1) (-2) (0 (0) - 1 (1))
=
2 (2) + 2 (-1)
=
2.
\end{align*}}% End solution.



\begin{enumerate}[resume,label=(\alph*)]
\item\label{itm : Quiz10 Part b} (2 pt) Apply the row reduction algorithm to $\left[\begin{array}{c;{2pt/2pt}c}\fontMatrix{A}&\fontMatrix{I}_{3}\end{array}\right]$ to show that
\begin{align*}
A^{-1}
=
\frac{1}{2}
\begin{bmatrix}
2	&	-2	&	-2	\\
-1	&	2	&	1	\\
2	&	0	&	0
\end{bmatrix}%
.
\end{align*}
\fontHint{Recall that one of the three elementary row operations lets us swap any two rows.}
\end{enumerate}

\spaceSolution{3in}{% Begin solution.
We compute
\begin{align*}
\left[
\begin{array}{c;{2pt/2pt}c}
\fontMatrix{A}	&	\fontMatrix{I}_{3}
\end{array}
\right]
&=
\left[
\begin{array}{c c c;{2pt/2pt}c c c}
0	&	0	&	1	&	1	&	0	&	0	\\
1	&	2	&	0	&	0	&	1	&	0	\\
-2	&	-2	&	1	&	0	&	0	&	1
\end{array}
\right]
\\
&\xrightarrow{R_{1} \leftrightarrow R_{2}}
\left[
\begin{array}{c c c;{2pt/2pt}c c c}
1	&	2	&	0	&	0	&	1	&	0	\\
0	&	0	&	1	&	1	&	0	&	0	\\
-2	&	-2	&	1	&	0	&	0	&	1
\end{array}
\right]
\\
&\xrightarrow{R_{3} = R_{3} + 2 R_{1}}
\left[
\begin{array}{c c c;{2pt/2pt}c c c}
1	&	2	&	0	&	0	&	1	&	0	\\
0	&	0	&	1	&	1	&	0	&	0	\\
0	&	2	&	1	&	0	&	2	&	1
\end{array}
\right]
\\
&\xrightarrow{R_{1} = R_{1} - R_{3}}
\left[
\begin{array}{c c c;{2pt/2pt}c c c}
1	&	0	&	-1	&	0	&	-1	&	-1	\\
0	&	0	&	1	&	1	&	0	&	0	\\
0	&	2	&	1	&	0	&	2	&	1
\end{array}
\right]
\\
&\xrightarrow{R_{3} = \frac{1}{2} R_{3}}
\left[
\begin{array}{c c c;{2pt/2pt}c c c}
1	&	0	&	-1		&	0	&	-1	&	-1		\\
0	&	0	&	1		&	1	&	0	&	0		\\
0	&	1	&	\frac{1}{2}	&	0	&	1	&	\frac{1}{2}
\end{array}
\right]
\\
&\xrightarrow{R_{2} \leftrightarrow R_{3}}
\left[
\begin{array}{c c c;{2pt/2pt}c c c}
1	&	0	&	-1		&	0	&	-1	&	-1		\\
0	&	1	&	\frac{1}{2}	&	0	&	1	&	\frac{1}{2}	\\
0	&	0	&	1		&	1	&	0	&	0
\end{array}
\right]
\\
&\xrightarrow{R_{1} = R_{1} + R_{3},R_{2} = R_{2} - \frac{1}{2} R_{3}}
\left[
\begin{array}{c c c;{2pt/2pt}c c c}
1	&	0	&	0	&	1			&	-1	&	-1		\\
0	&	1	&	0	&	-\frac{1}{2}	&	1	&	\frac{1}{2}	\\
0	&	0	&	1	&	1			&	0	&	0
\end{array}
\right].
\end{align*}
The $3 \times 3$ matrix to the right of the dashed lines is $A^{-1}$. Factoring out $\frac{1}{2}$ from all of the entries of this matrix, we conclude that
\begin{align*}
A^{-1}
=
\begin{bmatrix}
1			&	-1	&	-1		\\
-\frac{1}{2}	&	1	&	\frac{1}{2}	\\
1			&	0	&	0
\end{bmatrix}
=
\frac{1}{2}
\begin{bmatrix}
2	&	-2	&	-2	\\
-1	&	2	&	1	\\
2	&	0	&	0
\end{bmatrix}%
,
\end{align*}
as claimed.}% End solution.



\begin{enumerate}[resume,label=(\alph*)]
\item (1 pt) Explain how existence of $A^{-1}$ in part \ref{itm : Quiz10 Part b} is consistent with our answer in part \ref{itm : Quiz10 Part a}.
\end{enumerate}

\spaceSolution{1in}{% Begin solution.
A square matrix is invertible if and only if its determinant is nonzero. In part \ref{itm : Quiz10 Part a}, we computed $\det A = 2 \neq 0$, so we know $A^{-1}$ exists (and hence our computations in part \ref{itm : Quiz10 Part b} to find it are not in vain).}% End solution.