%
% Quiz 10A : 2019-07-19 (F)
%
\section{Exercise}

(5 pt) Consider the matrix
\begin{align*}
A
=
\begin{bmatrix}
-1	&	0	&	-2	\\
1	&	-1	&	2	\\
0	&	-2	&	4
\end{bmatrix}%
.
\end{align*}
\begin{enumerate}[label=(\alph*)]
\item\label{itm : Quiz10 Part a} (2 pt) Show that $\det A = 4$.
\end{enumerate}

\spaceSolution{1.5in}{% Begin solution.
Using expansion by minors along column 1, we compute
\begin{align*}
\det A
&=
(-1)^{1 + 1} (-1) \det
\begin{bmatrix}
-1	&	2	\\
-2	&	4
\end{bmatrix}
+ (-1)^{1 + 1} (1) \det
\begin{bmatrix}
0	&	-2	\\
-2	&	4
\end{bmatrix}
+ 0
\\
&=
(1) (-1) (-4 - (-4)) + (-1) (1) (0 - 4)
=
0 + 4
=
4.
\end{align*}}% End solution.



\begin{enumerate}[resume,label=(\alph*)]
\item\label{itm : Quiz10 Part b} (2 pt) Apply the row reduction algorithm to $\left[\begin{array}{c;{2pt/2pt}c}\fontMatrix{A}&\fontMatrix{I}_{3}\end{array}\right]$ to show that
\begin{align*}
A^{-1}
=
\frac{1}{4}
\begin{bmatrix}
0	&	4	&	-2	\\
-4	&	-4	&	0	\\
-2	&	-2	&	1
\end{bmatrix}%
.
\end{align*}
\end{enumerate}

\spaceSolution{3in}{% Begin solution.
We compute
\begin{align*}
\left[
\begin{array}{c;{2pt/2pt}c}
\fontMatrix{A}	&	\fontMatrix{I}_{3}
\end{array}
\right]
&=
\left[
\begin{array}{c c c;{2pt/2pt}c c c}
-1	&	0	&	-2	&	1	&	0	&	0	\\
1	&	-1	&	2	&	0	&	1	&	0	\\
0	&	-2	&	4	&	0	&	0	&	1
\end{array}
\right]
\\
&\xrightarrow{R_{1} = -R_{1}}
\left[
\begin{array}{c c c;{2pt/2pt}c c c}
1	&	0	&	2	&	-1	&	0	&	0	\\
1	&	-1	&	2	&	0	&	1	&	0	\\
0	&	-2	&	4	&	0	&	0	&	1
\end{array}
\right]
\\
&\xrightarrow{R_{2} = R_{2} - R_{1}}
\left[
\begin{array}{c c c;{2pt/2pt}c c c}
1	&	0	&	2	&	-1	&	0	&	0	\\
0	&	-1	&	0	&	1	&	1	&	0	\\
0	&	-2	&	4	&	0	&	0	&	1
\end{array}
\right]
\\
&\xrightarrow{R_{2} = -R_{2}}
\left[
\begin{array}{c c c;{2pt/2pt}c c c}
1	&	0	&	2	&	-1	&	0	&	0	\\
0	&	1	&	0	&	-1	&	-1	&	0	\\
0	&	-2	&	4	&	0	&	0	&	1
\end{array}
\right]
\\
&\xrightarrow{R_{3} = R_{3} + 2 R_{2}}
\left[
\begin{array}{c c c;{2pt/2pt}c c c}
1	&	0	&	2	&	-1	&	0	&	0	\\
0	&	1	&	0	&	-1	&	-1	&	0	\\
0	&	0	&	4	&	-2	&	-2	&	1
\end{array}
\right]
\\
&\xrightarrow{R_{3} = \frac{1}{4} R_{3}}
\left[
\begin{array}{c c c;{2pt/2pt}c c c}
1	&	0	&	2	&	-1			&	0			&	0		\\
0	&	1	&	0	&	-1			&	-1			&	0		\\
0	&	0	&	1	&	-\frac{1}{2}	&	-\frac{1}{2}	&	\frac{1}{4}
\end{array}
\right]
\\
&\xrightarrow{R_{1} = R_{1} - 2 R_{3}}
\left[
\begin{array}{c c c;{2pt/2pt}c c c}
1	&	0	&	2	&	0			&	1			&	-\frac{1}{2}	\\
0	&	1	&	0	&	-1			&	-1			&	0			\\
0	&	0	&	1	&	-\frac{1}{2}	&	-\frac{1}{2}	&	\frac{1}{4}
\end{array}
\right].
\end{align*}
The $3 \times 3$ matrix to the right of the dashed line is $A^{-1}$. Factoring out $\frac{1}{4}$ from all of the entries of this matrix, we conclude that
\begin{align*}
A^{-1}
=
\begin{bmatrix}
0			&	1			&	-\frac{1}{2}	\\
-1			&	-1			&	0			\\
-\frac{1}{2}	&	-\frac{1}{2}	&	\frac{1}{4}
\end{bmatrix}
=
\frac{1}{4}
\begin{bmatrix}
0	&	4	&	-2	\\
-4	&	-4	&	0	\\
-2	&	-2	&	1
\end{bmatrix}%
,
\end{align*}
as claimed.}% End solution.



\begin{enumerate}[resume,label=(\alph*)]
\item (1 pt) What is $\det A^{-1}$? \fontHint{You can answer this without computing $\det A^{-1}$ directly. How?}
\end{enumerate}

\spaceSolution{1in}{% Begin solution.
Recall that, for any square matrices $A,B$ of the same order,
\begin{align*}
\det(A B)
=
(\det A) (\det B).
\end{align*}
In particular, if $A$ is invertible, then we may take $B = A^{-1}$, giving
\begin{align*}
1
=
\det I
=
\det(A A^{-1})
=
(\det A) (\det (A^{-1})).
\end{align*}
Because $A$ is invertible, $\det A \neq 0$, so we may solve this equation for $\det A^{-1}$:
\begin{align*}
\det(A^{-1})
=
(\det A)^{-1}.
\end{align*}
For our particular matrix $A$, we conclude that
\begin{align*}
\det(A^{-1})
=
\frac{1}{4}.
\end{align*}
One can check that this agrees with the result we get via computation of $\det A^{-1}$ directly. (Try it! To simplify computation, use the multilinearity property of the determinant: For an $n \times n$ matrix $B$, and for any scalar $c$, $\det(c B) = c^{n} \det B$. In particular, to compute the determinant of our matrix $A^{-1}$, we may compute the determinant of the matrix of integers (we should find it equals $16$), and multiply the final result by $(\frac{1}{4})^{3} = \frac{1}{64}$.)}% End solution.