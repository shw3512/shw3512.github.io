%
% Quiz 11 : 2019-07-22 (M)
%
\section{Exercise}

(2 pt) Fix the following notation:
\begin{itemize}
\item Let $I \subseteq \reals$ denote the closed interval $[0,1]$.
\item Let $\continuousFunctions^{0}(I)$ denote the vector space over $\reals$ of continuous functions $f : I \rightarrow \reals$.
\item Let $\reals^{3}$ denote the vector space over $\reals$ of $3 \times 1$ matrices whose entries are real numbers.
\end{itemize}
Circle the corresponding letter if the subset $W \subseteq V$ described is a subspace of the given vector space $V$.
\begin{enumerate}[label=(\alph*)]
\item\label{itm : Quiz11 R3 Entries Add To 0} $V = \reals^{3}$, $W$ is the set of matrices $\transpose{\begin{bmatrix}x_{1}&x_{2}&x_{3}\end{bmatrix}} \in V$ such that $x_{1} + x_{2} + x_{3} = 0$.
\item\label{itm : Quiz11 R3 Entries Multiply To 0} $V = \reals^{3}$, $W$ is the set of matrices $\transpose{\begin{bmatrix}x_{1}&x_{2}&x_{3}\end{bmatrix}} \in V$ such that $x_{1} x_{2} x_{3} = 0$.
\item\label{itm : Quiz11 f(1)=0} $V = \continuousFunctions^{0}(I)$, $W$ is the set of functions $f \in V$ such that $f(1) = 0$.
\item\label{itm : Quiz11 f(0)=1} $V = \continuousFunctions^{0}(I)$, $W$ is the set of functions $f \in V$ such that $f(0) = 1$.
\end{enumerate}

\spaceSolution{2in}{% Begin solution.
We analyze each subset in turn. Each time, we check the two defining properties of a subspace: (i) nonempty, and (ii) closed under linear combinations.
\begin{enumerate}[label=(\alph*)]
\item is a subspace. Nonempty : The $3 \times 1$ zero matrix $\transpose{\begin{bmatrix}0&0&0\end{bmatrix}}$ satisfies the condition that its entries sum to $0$, so $W \neq \emptyset$. Closure : Let $w_{1},w_{2} \in W$, and let $a_{1},a_{2} \in \reals$. The linear combination
\begin{align*}
a_{1} w_{1} + a_{2} w_{2}
=
\begin{bmatrix}
a_{1} w_{1,1} + a_{2} w_{2,1}	\\
a_{1} w_{1,2} + a_{2} w_{2,2}	\\
a_{1} w_{1,3} + a_{2} w_{2,3}
\end{bmatrix}%
.
\end{align*}
The sum of the entries of this linear combination is
\begin{align*}
&(a_{1} w_{1,1} + a_{2} w_{2,1}) + (a_{1} w_{1,2} + a_{2} w_{2,2}) + (a_{1} w_{1,3} + a_{2} w_{2,3})
\\
&=
a_{1} (w_{1,1} + w_{1,2} + w_{1,3}) + a_{2} (w_{2,1} + w_{2,2} + w_{2,3})
\\
&=
a_{1} (0) + a_{2} (0)
=
0,
\end{align*}
where in the last line we use the hypothesis that $w_{1},w_{2} \in W$.

\item is not a subspace. It is not closed under linear combinations. For example, $w_{1} = \transpose{\begin{bmatrix}1&1&0\end{bmatrix}}$ and $w_{2} = \transpose{\begin{bmatrix}0&1&1\end{bmatrix}}$ are both in $W$, but their sum $w_{1} + w_{2} = \transpose{\begin{bmatrix}1&2&1\end{bmatrix}}$ is not.

\item is a subspace. Nonempty : The zero function $f(t) \equiv 0$ satisfies $f(1) = 0$, so $W \neq \emptyset$. Closure : Let $f_{1},f_{2} \in W$, and let $a_{1},a_{2} \in \reals$. The linear combination $a_{1} f_{1} + a_{2} f_{2}$ satisfies
\begin{align*}
(a_{1} f_{1} + a_{2} f_{2})(1)
&=
a_{1} \cdot f_{1}(1) + a_{2} \cdot (f_{2}(1)
\\
&=
a_{1} \cdot 0 + a_{2} \cdot 0
=
0,
\end{align*}
where in the last line we use the hypothesis that $f_{1},f_{2} \in W$.

\item is not a subspace. It is not closed under linear combinations. For example, $f(t) \equiv 1$ is in $W$, but $2 f$ is not (because $(2 f)(1) = 2 f(1) = 2 \cdot 1 = 2 \neq 1$).
\end{enumerate}}% End solution.