%
% Quiz 14 : 2019-07-25 (R)
%
\section{Exercise}

(2 pt) Consider the 3rd-order linear ODE
\begin{align}
y^{(3)} + 6 y^{(2)} + 11 y^{(1)} + 6 y
=
0,%
\label{eq : Quiz 14 ODE}
\end{align}
where $y^{(n)}$ denotes the $n$th (ordinary) derivative $\frac{\intd^{n} y}{\intd t^{n}}$. Using the change of variables
\begin{align*}
x_{n}
=
y^{(n)},
\end{align*}
for $n = 0,1,2$, translate the 3rd-order ODE \eqref{eq : Quiz 14 ODE} into a 1st-order linear system
\begin{align*}
\begin{bmatrix}
x_{0}	\\
x_{1}	\\
x_{2}
\end{bmatrix}
'
=
A
\begin{bmatrix}
x_{0}	\\
x_{1}	\\
x_{2}
\end{bmatrix}%
,
\end{align*}
where $A$ is a $3 \times 3$ matrix of constants. \fontHint{The entries in the third row of $A$ should be related to the coefficients of the original ODE \eqref{eq : Quiz 14 ODE}; the entries in the first two rows of $A$ should all be $0$ or $1$.}

(Not required : Compute the characteristic polynomial of the coefficient matrix $A$, i.e. the polynomial $\det(A - \lambda I)$. (This is the polynomial we've met before, whose roots are the eigenvalues of $A$.) How does this compare to the equation we get by replacing each $y^{(n)}$ in the original ODE \eqref{eq : Quiz 14 ODE} with $\lambda^{n}$? How do the roots of these two polynomials compare?)

\spaceSolution{4in}{% Begin solution.
Computing the derivatives of the new variables $x_{n}$, we find
\begin{align*}
x_{0}'
&=
(y)'
=
y^{(1)}
=
x_{1}
\\
x_{1}'
&=
(y^{(1)})'
=
y^{(2)}
=
x_{2}
\\
x_{2}'
&=
(y^{(2)})'
=
y^{(3)}
=
-6 y^{(2)} - 11 y^{(1)} - 6
=
-6 x_{2} - 11 x_{1} - 6 x_{0}.
\end{align*}
Writing this system of 1st-order ODEs as a matrix equation, we have
\begin{align*}
\begin{bmatrix}
x_{0}	\\
x_{1}	\\
x_{2}
\end{bmatrix}
'
=
\begin{bmatrix}
x_{0}'	\\
x_{1}'	\\
x_{2}'
\end{bmatrix}
=
\left[
\begin{array}{r c r c r}
		&		&	x_{1}		&		&			\\
		&		&			&		&	x_{2}		\\
-6 x_{0}	&	-	&	11 x_{1}	&	-	&	6 x_{2}
\end{array}
\right]
=
\begin{bmatrix}
0	&	1	&	0	\\
0	&	0	&	1	\\
-6	&	-11	&	-6
\end{bmatrix}
\begin{bmatrix}
x_{0}	\\
x_{1}	\\
x_{2}
\end{bmatrix}%
.
\end{align*}

The characteristic polynomial of the coefficient matrix $A$ is
\begin{align*}
\det(A - \lambda I)
=
\det
\begin{bmatrix}
-\lambda	&	1		&	0			\\
0		&	-\lambda	&	1			\\
-6		&	-11		&	-6 - \lambda
\end{bmatrix}
=
-\lambda^{3} - 6 \lambda^{2} - 11 \lambda - 6.
\end{align*}
This is the same as the polynomial we get if we replace each $y^{(n)}$ on the left side of our original linear ODE \eqref{eq : Quiz 14 ODE} with $\lambda^{n}$, then multiply everything by $-1$.

By definition, a root of a polynomial $p$ is an input $\lambda$ that makes $p(\lambda) = 0$. Multiplying both sides of this equation by $-1$, we get $-p(\lambda) = -0 = 0$. Thus, a root of a polynomial $p$ is also a root of $-p$, and vice versa. That is, a polynomial and $-1$ times that polynomial have the same roots. In our context here, this says that the roots of the polynomial we get from \eqref{eq : Quiz 14 ODE} are exactly the eigenvalues of the matrix $A$.}% End solution.