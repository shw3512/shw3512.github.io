%
% Quiz 15 : 2019-07-26 (F)
%
\section{Exercise}

(5 pt) Solve the initial value problem given by the homogeneous 2nd-order ODE
\begin{align}
y'' + 6 y' + 8 y
=
0%
\label{eq : Quiz 15 ODE}
\end{align}
and the initial conditions
\begin{align*}
y(0)
=
3,
&&
y'(0)
=
-10.
\end{align*}
Your final answer should be an explicit equation for $y(t)$.% Begin footnote.
\footnote{\fontHint{We have two approaches to find the general solution $y(t)$ to \eqref{eq : Quiz 15 ODE}: (1) translate the given higher-order linear ODE into the corresponding 1st-order linear system (what do the initial conditions become in this case?), and (2) jump directly to the characteristic equation. The general solution involves two parameters, call them $c_{1},c_{2}$. Compute $y(t)$, take its derivative to get $y'(t)$ (which will also involve $c_{1},c_{2}$), then apply the initial conditions to get a system of two linear equations in the two unknowns $c_{1},c_{2}$. Now solve this system, e.g., using our techniques from linear algebra.}} % End footnote.

\spaceSolution{6in}{% Begin solution.
Applying the change of variables
\begin{align*}
x_{0}
=
y,
&&
x_{1}
=
y',
\end{align*}
we get the corresponding homogeneous 1st-order linear system
\begin{align*}
\begin{bmatrix}
x_{0}	\\
x_{1}
\end{bmatrix}%
'
=
\begin{bmatrix}
0	&	1	\\
-8	&	-6
\end{bmatrix}
\begin{bmatrix}
x_{0}	\\
x_{1}
\end{bmatrix}%
.
\end{align*}
The characteristic polynomial of the (constant) coefficient matrix, call it $A$, is
\begin{align*}
p(\lambda)
=
\det(A - \lambda I)
=
\det
\begin{bmatrix}
-\lambda	&	1			\\
-8		&	-6 - \lambda
\end{bmatrix}
=
\lambda^{2} + 6 \lambda + 8.
\end{align*}
Note that this is the same characteristic polynomial we obtain by replacing each derivative $y^{(n)}$ with $\lambda^{n}$ in the given ODE \eqref{eq : Quiz 15 ODE}. The eigenvalues are the zeros of this characteristic polynomial:
\begin{align*}
0
\seteq
p(\lambda)
=
(\lambda + 4) (\lambda + 2)
&&
\Leftrightarrow
&&
\lambda
=
-4,-2.
\end{align*}
Because all the eigenvalues are distinct, the general solution $y$ to \eqref{eq : Quiz 15 ODE} is obtained by putting these eigenvalues as coefficients inside exponentials, and taking the linear combination:
\begin{align*}
y(t)
=
c_{1} e^{-4 t} + c_{2} e^{-2 t}.
\end{align*}
To find the solution to the IVP, we compute $y'$,
\begin{align*}
y'(t)
=
-4 c_{1} e^{-4 t} - 2 c_{2} e^{-2 t},
\end{align*}
and apply the initial conditions (the degree of the ODE \eqref{eq : Quiz 15 ODE} is 2, so 2 initial conditions are required):
\begin{align*}
3
&=
y(0)
=
c_{1} + c_{2}
\\
-10
&=
y'(0)
=
-4 c_{1} - 2 c_{2}.
\end{align*}
This is a linear system of equations in the unknowns $c_{1},c_{2}$, which we can solve in several ways. Here we'll form the augmented matrix associated to the system and row reduce:
\begin{align*}
\left[
\begin{array}{c c;{2pt/2pt}c}
1	&	1	&	3	\\
-4	&	-2	&	-10
\end{array}
\right]
\xrightarrow{R_{2} = R_{2} + 4 R_{1}}
\left[
\begin{array}{c c;{2pt/2pt}c}
1	&	1	&	3	\\
0	&	2	&	2
\end{array}
\right]
\xrightarrow{R_{2} = \frac{1}{2} R_{2}}
\left[
\begin{array}{c c;{2pt/2pt}c}
1	&	1	&	3	\\
0	&	1	&	1
\end{array}
\right]
\xrightarrow{R_{1} = R_{1} - R_{2}}
\left[
\begin{array}{c c;{2pt/2pt}c}
1	&	0	&	2	\\
0	&	1	&	1
\end{array}
\right].
\end{align*}
Reading of the solution, $c_{1} = 2$ and $c_{2} = 1$. Thus the solution to the IVP is
\begin{align*}
y(t)
=
2 e^{-4 t} + e^{-2 t}.
\end{align*}
One can check that this $y(t)$ indeed satisfies both the ODE and the initial conditions, as required.}% End solution.