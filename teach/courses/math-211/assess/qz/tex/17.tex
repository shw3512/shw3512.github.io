%
% Quiz 17 : 2019-07-30 (T)
%
\section{Exercise}

(5 pt) For each of the following linear maps $T : \reals^{n} \rightarrow \reals^{m}$, where $\reals^{n},\reals^{m}$ are viewed as vector spaces over $\reals$,
\begin{enumerate}[label=(\roman*)]
\item write a basis for the image $\image(T)$ and a basis for the kernel $\ker(T)$,
\item find the dimensions $\dim(\image(T))$ and $\dim(\ker(T))$, and
\item confirm the rank--nullity theorem:
\begin{align*}
\dim(\text{domain}(T))
=
\dim(\image(T))
+
\dim(\ker(T)).
\end{align*}
\end{enumerate}
\fontHint{For each linear map $T$, write a corresponding matrix representation $A$, so that $T(x) = y$ corresponds to the matrix equation $A x = y$. Then focus on the pivot or nonpivot columns of $A$.}
\begin{enumerate}[label=(\alph*)]
\item\label{itm : Quiz17a} (2.5 pt) The linear map $T_{1}$ given by
\begin{align*}
T_{1}
:
\reals^{4}
&\rightarrow
\reals^{3}
\\
\begin{bmatrix}
x_{1}		\\
\vdots	\\
x_{4}
\end{bmatrix}
&\mapsto
\left[
\begin{array}{c c c c c c c}
x_{1}	&	+	&		&		&	x_{3}	\\
	&		&	x_{2}	&	-	&	x_{3}	\\
	&		&	0	&		&	
\end{array}
\right].
\end{align*}
\end{enumerate}

\spaceSolution{1.75in}{% Begin solution.
The corresponding matrix $A_{1}$ is
\begin{align*}
A_{1}
=
\begin{bmatrix}
1	&	0	&	1	&	0	\\
0	&	1	&	-1	&	0	\\
0	&	0	&	0	&	0
\end{bmatrix}%
.
\end{align*}
This matrix is already in reduced row echelon form (RREF).

\textbf{Image.} The pivot columns are columns 1 and 2. They form a basis for the image of $T_{1}$:
\begin{align*}
\text{basis}(\image(T_{1}))
=
\left(
\begin{bmatrix}
1	\\
0	\\
0
\end{bmatrix}%
,
\begin{bmatrix}
0	\\
1	\\
0
\end{bmatrix}%
\right),
\end{align*}
The number of pivot columns equals the dimension of the image:
\begin{align*}
\dim(\image(T_{1}))
=
2.
\end{align*}

\textbf{Kernel.} The other two columns, 3 and 4, correspond to free variables, namely $x_{3}$ and $x_{4}$. The number of free variables equals the dimension of the kernel:
\begin{align*}
\dim(\ker(T_{1}))
=
2.
\end{align*}
Solving the system of equations represented by the matrix equation $A_{1} x = 0$, where $x$ is a $4 \times 1$ matrix of variables $x_{i}$ and $0$ is the $3 \times 1$ zero matrix, gives the kernel of $T_{1}$:
\begin{align*}
\ker(T_{1})
=
\left\{
\begin{bmatrix}
-x_{3}	\\
x_{3}		\\
x_{3}		\\
x_{4}
\end{bmatrix}
\st
x_{3},x_{4} \in \reals
\right\}
=
\left\{
x_{3}
\begin{bmatrix}
-1	\\
1	\\
1	\\
0
\end{bmatrix}%
+
x_{4}
\begin{bmatrix}
0	\\
0	\\
0	\\
1
\end{bmatrix}%
\st
x_{3},x_{4} \in \reals\right\}
=
\Span\left\{
\begin{bmatrix}%
-1	\\
1	\\
1	\\
0
\end{bmatrix}%
,
\begin{bmatrix}
0	\\
0	\\
0	\\
1
\end{bmatrix}%
\right\}.
\end{align*}
The two vectors in this last expression for $\ker(T_{1})$ are linearly independent, and hence they form a basis for $\ker(T_{1})$.

\textbf{Rank--nullity theorem.} For $T_{1}$, we have
\begin{align*}
\dim(\text{domain}(T_{1}))
=
\dim(\reals^{4})
=
4
=
2 + 2
=
\dim(\image(T_{1})) + \dim(\ker(T_{1})),
\end{align*}
confirming that the rank--nullity theorem holds for $T_{1}$.}% End solution.



\begin{enumerate}[resume,label=(\alph*)]
\item\label{itm : Quiz17b} (2.5 pt) The linear map $T_{2}$ given by
\begin{align*}
T_{2}
:
\reals^{4}
&\rightarrow
\reals^{3}
\\
\begin{bmatrix}
x_{1}		\\
\vdots	\\
x_{4}
\end{bmatrix}
&\mapsto
\left[
\begin{array}{c c c c c c c}
x_{1}		&	+	&	x_{2}		&	+	&	2 x_{3}	&	+	&	x_{4}		\\
3 x_{1}	&	+	&	2 x_{2}	&	+	&	3 x_{3}	&	+	&	2 x_{4}	\\
x_{1}		&	+	&	2 x_{2}	&	+	&	3 x_{3}	&	+	&	4 x_{4}
\end{array}
\right].
\end{align*}
\end{enumerate}

\spaceSolution{1.75in}{% Begin solution.
The corresponding matrix $A_{2}$ is
\begin{align*}
A_{2}
=
\begin{bmatrix}
1	&	1	&	2	&	1	\\
3	&	2	&	3	&	2	\\
1	&	2	&	3	&	4
\end{bmatrix}%
.
\end{align*}
Applying the row reduction algorithm to this matrix, we get
\begin{align*}
\rref(A_{2})
=
\begin{bmatrix}
1	&	0	&	0	&	-1	\\
0	&	1	&	0	&	4	\\
0	&	0	&	1	&	-1
\end{bmatrix}%
.
\end{align*}

\textbf{Image.} The pivot columns are columns 1, 2, and 3. These columns, in the original matrix $A_{2}$ (!), form a basis for the image of $T_{2}$:
\begin{align*}
\text{basis}(\image(T_{2}))
=
\left(
\begin{bmatrix}
1	\\
3	\\
1
\end{bmatrix}%
,
\begin{bmatrix}
1	\\
2	\\
2
\end{bmatrix}%
,
\begin{bmatrix}
2	\\
3	\\
3
\end{bmatrix}%
\right).
\end{align*}
The number of pivot columns equals the dimension of the image of $T_{2}$:
\begin{align*}
\dim(\image(T_{2}))
=
3.
\end{align*}
N.B. Because $\image(T_{2})$ is a subspace of the vector space $\reals^{3}$, and their dimensions are equal --- $\dim(\image(T_{2})) = \dim(\reals^{3})$ --- this subspace must be the entire vector space. That is,
\begin{align*}
\image(T_{2})
=
\reals^{3}.
\end{align*}

\textbf{Kernel}. The nonpivot column, column 4, corresponds to the free variable, namely $x_{4}$. The number of free variables equals the dimension of the kernel:
\begin{align*}
\dim(\ker(T_{2}))
=
1.
\end{align*}
The set of solutions to the system of equations represented by the matrix equation $A_{2} x = 0$ is equivalent to the set of solutions to the system of equations represented by the matrix equation $\rref(A_{2}) x = 0$; in both equations, $x$ is a $4 \times 1$ matrix of variables $x_{i}$ and $0$ is the $3 \times 1$ zero matrix. This set of solutions is the kernel of $T_{2}$:
\begin{align*}
\ker(T_{2})
=
\left\{
\begin{bmatrix}
x_{4}		\\
-4 x_{4}	\\
x_{4}		\\
x_{4}
\end{bmatrix}
\st
x_{4} \in \reals
\right\}
=
\left\{
x_{4}
\begin{bmatrix}
1	\\
-4	\\
1	\\
1
\end{bmatrix}
\st
x_{4} \in \reals
\right\}
=
\Span\left\{
\begin{bmatrix}
1	\\
-4	\\
1	\\
1
\end{bmatrix}
\right\}.
\end{align*}
The vector in this last expression for $\ker(T_{2})$ is a basis for $\ker(T_{2})$.

\textbf{Rank--nullity theorem.} For $T_{2}$, we have
\begin{align*}
\dim(\text{domain}(T_{2}))
=
\dim(\reals^{4})
=
4
=
3 + 1
=
\dim(\image(T_{2})) + \dim(\ker(T_{2})),
\end{align*}
confirming that the rank--nullity theorem holds for $T_{2}$.}% End solution.

% WolframAlpha code :
% row reduce {{1,1,2,1},{3,2,3,2},{1,2,3,4}}