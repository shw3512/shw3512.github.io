%
% Quiz 17A : 2019-08-02 (F)
%
\section{Exercise}

(5 pt) For each of the following linear maps $T : \reals^{n} \rightarrow \reals^{m}$, where $\reals^{n},\reals^{m}$ are viewed as vector spaces over $\reals$,
\begin{enumerate}[label=(\roman*)]
\item write a basis for the image $\image(T)$ and a basis for the kernel $\ker(T)$, and
\item confirm the rank--nullity theorem.
\end{enumerate}
\begin{enumerate}[label=(\alph*)]
\item\label{itm : Quiz17Aa} (2.5 pt) The linear map $T_{1}$ given by
\begin{align*}
T_{1}
:
\reals^{3}
&\rightarrow
\reals^{5}
\\
\begin{bmatrix}
x_{1}		\\
\vdots	\\
x_{3}
\end{bmatrix}
&\mapsto
\left[
\begin{array}{c c c c c c c}
x_{1}	&	+	&			&		&	x_{3}		\\
x_{1}	&	+	&	x_{2}		&	-	&	x_{3}		\\
	&		&	x_{2}		&	-	&	2 x_{3}	\\
	&		&	x_{2}		&	-	&	2 x_{3}	\\
	&		&	-2 x_{2}	&	+	&	4 x_{3}
\end{array}
\right].
\end{align*}
\end{enumerate}

% WolframAlpha code :
% row reduce {{1,0,1},{1,1,-1},{0,1,-2},{0,1,-2},{0,-2,4}}

\spaceSolution{2in}{% Begin solution.
We write the corresponding matrix $A_{1}$, then apply the row reduction algorithm (RRA) to get its reduced row echelon form (RREF):
\begin{align*}
A_{1}
=
\begin{bmatrix}
1	&	0	&	1	\\
1	&	1	&	-1	\\
0	&	1	&	-2	\\
0	&	1	&	-2	\\
0	&	-2	&	4
\end{bmatrix}%
\xrightarrow{\text{RRA}}
\begin{bmatrix}
1	&	0	&	1	\\
0	&	1	&	-2	\\
0	&	0	&	0	\\
0	&	0	&	0	\\
0	&	0	&	0
\end{bmatrix}%
=
\rref(A_{1}).
\end{align*}
The image is the span of the pivot columns of the original (!) matrix $A_{1}$
\begin{align*}
\image(T_{1})
=
\Span\left\{
\begin{bmatrix}
1	\\
1	\\
0	\\
0	\\
0	
\end{bmatrix}%
,
\begin{bmatrix}
0	\\
1	\\
1	\\
1	\\
-2	
\end{bmatrix}%
\right\}.
\end{align*}
These columns are linearly independent and hence form a basis for $\image(T_{1})$.

Notice that the third column of $A_{1}$ is a linear combination of the first two. More precisely, letting $(A_{1})_{\bullet,j}$ denote the $j$th column of $A_{1}$, we can check that $(A_{1})_{\bullet,3} = (A_{1})_{\bullet,1} - 2 (A_{1})_{\bullet,2}$.\fontNeedsEdit{ (elaborate?)}

The nonpivot columns, in this case column 3, correspond to free variables, in this case $x_{3}$. The general vector in the kernel of $T_{1}$ can be read off from the RREF of $A_{1}$ --- more precisely, from the augmented matrix $\left[\begin{array}{c;{2pt/2pt}c}\rref(A)&\fontMatrix{0}\end{array}\right]$:
\begin{align*}
\ker(T_{1})
=
\left\{
\begin{bmatrix}
-x_{3}	\\
2 x_{3}	\\
x_{3}
\end{bmatrix}
\st
x_{3} \in \reals
\right\}
=
\Span\left\{
\begin{bmatrix}
-1	\\
2	\\
1
\end{bmatrix}
\right\}.
\end{align*}
A set containing a single nonzero vector is always linearly independent. Thus any nonzero vector in $\ker(T_{1})$ is a basis of $\ker(T_{1})$.

The rank--nullity theorem writes as
\begin{align*}
\dim(\text{domain}(T_{1}))
=
3
=
2 + 1
=
\dim(\image(T_{1})) + \dim(\ker(T_{1})).
\end{align*}}% End solution.



\begin{enumerate}[resume,label=(\alph*)]
\item\label{itm : Quiz17Ab} (2.5 pt) The linear map $T_{2}$ given by
\begin{align*}
T_{2}
:
\reals^{4}
&\rightarrow
\reals^{4}
\\
\begin{bmatrix}
x_{1}		\\
\vdots	\\
x_{4}
\end{bmatrix}
&\mapsto
\left[
\begin{array}{c c c c c c c}
x_{1}		&	+	&			&		&	x_{3}		&		&			\\
x_{1}		&	+	&	x_{2}		&	+	&	2 x_{3}	&	+	&	3 x_{4}	\\
2 x_{1}	&	+	&			&		&	2 x_{3}	&	+	&	2 x_{4}	\\
		&		&			&		&			&		&	4 x_{4}
\end{array}
\right].
\end{align*}
\end{enumerate}

% WolframAlpha code :
% row reduce {{1,0,1,0},{1,1,2,3},{2,0,2,2},{0,0,0,4}}

\spaceSolution{2in}{% Begin solution.
We write the corresponding matrix $A_{2}$, then apply the RRA to get its RREF:
\begin{align*}
A_{2}
=
\begin{bmatrix}
1	&	0	&	1	&	0	\\
1	&	1	&	2	&	3	\\
2	&	0	&	2	&	2	\\
0	&	0	&	0	&	4
\end{bmatrix}%
\xrightarrow{\text{RRA}}
\begin{bmatrix}
1	&	0	&	1	&	0	\\
0	&	1	&	1	&	0	\\
0	&	0	&	0	&	1	\\
0	&	0	&	0	&	0
\end{bmatrix}%
=
\rref{A_{2}}.
\end{align*}

By the same logic as in part \ref{itm : Quiz17Aa}, we find
\begin{align*}
\image(T_{2})
=
\Span\left\{
\begin{bmatrix}
1	\\
1	\\
2	\\
0
\end{bmatrix}%
,
\begin{bmatrix}
0	\\
1	\\
0	\\
0
\end{bmatrix}%
,
\begin{bmatrix}
0	\\
3	\\
2	\\
4
\end{bmatrix}%
\right\}
\end{align*}
and
\begin{align*}
\ker(T_{2})
=
\Span\left\{
\begin{bmatrix}
-1	\\
-1	\\
1	\\
0
\end{bmatrix}%
\right\}.
\end{align*}

The rank--nullity theorem writes as
\begin{align*}
\dim(\text{domain}(T_{2}))
=
4
=
3 + 1
=
\dim(\image(T_{2})) + \dim(\ker(T_{2})).
\end{align*}}% End solution.