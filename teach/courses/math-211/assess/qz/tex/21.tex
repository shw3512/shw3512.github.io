%
% Quiz 21 : 2019-08-05 (M)
%
% Phase planes plotted using online software at
% http://www.calvin.edu/~scofield/demos/other/PhasePortrait2D.html
%
\section{Exercise}

(5 pt) Consider the homogeneous 1st-order $2 \times 2$ linear system of ODEs
\begin{align}
\fontMatrix{x}'
=
\begin{bmatrix}%
-1	&	-4	\\
4	&	-1
\end{bmatrix}%
\fontMatrix{x}.%
\label{eq : Quiz21 ODE}
\end{align}
\begin{enumerate}[label=(\alph*)]
\item (2 pt) Diagonalize the coefficient matrix (i.e. compute the eigenvalues and corresponding eigenvectors of the given matrix $\fontMatrix{A}$, and write it in the form $\fontMatrix{A} = \fontMatrix{P} \fontMatrix{D} \fontMatrix{P}^{-1}$).
\end{enumerate}

\spaceSolution{1.75in}{% Begin solution.
Denote the coefficient matrix in \eqref{eq : Quiz21 ODE} by
\begin{align*}
\fontMatrix{A}
=
\begin{bmatrix}%
-1	&	-4	\\
4	&	-1
\end{bmatrix}%
.
\end{align*}
\textbf{Eigenvalues}. We compute
\begin{align*}
0
\seteq
\det(\fontMatrix{A} - \lambda \fontMatrix{I})
=
\det
\begin{bmatrix}%
-1 - \lambda	&	-4			\\
4			&	-1 - \lambda
\end{bmatrix}%
=
\lambda^{2} + 2 \lambda + 17.
\end{align*}
Using the quadratic equation,
\begin{align*}
\lambda
=
\frac{-(2) \pm{} \sqrt{2^{2} - 4 (1) (17)}}{2 (1)}
=
\frac{-2 \pm{} \sqrt{4 - 4 (17)}}{2}
=
\frac{-2 \pm{} \sqrt{4 (1 - 17)}}{2}
=
-1 \pm{} 4 i.
\end{align*}
\textbf{Eigenvectors.} Compute the eigenspace for $\lambda = -1 + 4 i$:
\begin{align*}
\begin{bmatrix}%
0	\\
0
\end{bmatrix}%
=
\fontMatrix{0}
\seteq
(\fontMatrix{A} - \lambda \fontMatrix{I}) \fontMatrix{v}
=
\begin{bmatrix}%
-4 i	&	-4	\\
4	&	-4 i
\end{bmatrix}%
\begin{bmatrix}%
v_{1}	\\
v_{2}
\end{bmatrix}%
.
\end{align*}
Form the corresponding augmented matrix and apply the row reduction algorithm:
\begin{align*}
\left[
\begin{array}{c c;{2pt/2pt}c}
-4 i	&	-4	&	0	\\
4	&	-4i	&	0
\end{array}
\right]
\xrightarrow{R_{2} = R_{2} - i R_{1}}
\left[
\begin{array}{c c;{2pt/2pt}c}
-4 i	&	-4	&	0	\\
0	&	0	&	0
\end{array}
\right]
\xrightarrow{R_{1} = -\frac{1}{4 i} R_{1}}
\left[
\begin{array}{c c;{2pt/2pt}c}
1	&	-i	&	0	\\
0	&	0	&	0
\end{array}
\right],
\end{align*}
where in the last step we use $\frac{1}{i} = \frac{1 \cdot i}{i \cdot i} = \frac{i}{i^{2}} = \frac{i}{-1}= -i$. We conclude that
\begin{align*}
\eigenspace(\fontMatrix{A},-1 + 4 i)
=
\Span
\left\{
\begin{bmatrix}%
i	\\
1
\end{bmatrix}%
\right\}.
\end{align*}
We can perform the analogous computations for the other eigenvalue $\lambda = -1 - 4 i$, or we can note that because it is the complex conjugate of $-1 + 4 i$, its eigenvectors are complex conjugate to the eigenvectors of $-1 + 4 i$. Either way, we get
\begin{align*}
\eigenspace(\fontMatrix{A},-1 - 4 i)
=
\Span
\left\{
\begin{bmatrix}%
-i	\\
1
\end{bmatrix}%
\right\}.
\end{align*}
\textbf{Diagonalization.} Using these results, we conclude
\begin{align}
\fontMatrix{A}
=
\begin{bmatrix}
-1	&	-4	\\
4	&	-1
\end{bmatrix}
=
\begin{bmatrix}
i	&	-i	\\
1	&	1
\end{bmatrix}
\begin{bmatrix}
-1 + i	&	0	\\
0	&	-1 - i
\end{bmatrix}
\begin{bmatrix}
i	&	-i	\\
1	&	1
\end{bmatrix}%
^{-1}
=
\fontMatrix{P} \fontMatrix{D} \fontMatrix{P}^{-1}.%
\label{eq : Quiz21 Diagonalization}
\end{align}}% End solution.

\begin{enumerate}[resume,label=(\alph*)]
\item (2 pt) Use this information to write the general real solution $\fontMatrix{x}(t)$ to \eqref{eq : Quiz21 ODE}. \fontHint{Recall that when the eigenvalues of a homogeneous $2 \times 2$ linear system are complex, we can compute one solution $e^{\lambda t} v$, just as we did with real eigenvalues; decompose it into real and imaginary parts; then use these parts as our basis for the solution space. The formulas
\begin{align*}
e^{a + i b}
=
e^{a} e^{i b},
&&
e^{i b}
=
\cos b + i \sin b,
\end{align*}
will be useful.}
\end{enumerate}

\spaceSolution{2.25in}{% Begin solution.
Call columns 1 and 2 of $\fontMatrix{P}$ $\fontMatrix{v}_{+}$ and $\fontMatrix{v}_{-}$, respectively. (They are eigenvectors corresponding to the eigenvalues $\lambda_{+} = -1 + 4 i$ and $\lambda_{-} = -1 - 4 i$, respectively. Hence our choice of subscripts $+$ and $-$.) Our diagonalization \eqref{eq : Quiz21 Diagonalization} gives one basis for the vector space of complex (!) solutions to \eqref{eq : Quiz21 ODE}, namely, the vectors
\begin{align*}
\fontMatrix{x}_{+}(t)
=
e^{\lambda_{+} t} \fontMatrix{v}_{+}
=
e^{(-1 + 4 i) t}
\begin{bmatrix}%
i	\\
1
\end{bmatrix}%
,
&&
\fontMatrix{x}_{-}(t)
=
e^{\lambda_{-} t} \fontMatrix{v}_{-}
=
e^{(-1 - 4 i) t}
\begin{bmatrix}%
-i	\\
1
\end{bmatrix}%
.
\end{align*}
To describe the real (!) solutions to \eqref{eq : Quiz21 ODE}, we can decompose either one of these complex basis vectors into real and imaginary parts, $\fontMatrix{x}_{\realPart}$ and $\fontMatrix{x}_{\imaginaryPart}$, and take these as our basis vectors. Decomposing $\fontMatrix{x}_{+}(t)$, using the formulas in the hint, we find
\begin{align*}
\fontMatrix{x}_{+}(t)
&=
e^{(-1 + 4 i) t}
\begin{bmatrix}
i	\\
1
\end{bmatrix}
=
e^{-t} e^{i (4 t)}
\begin{bmatrix}
i	\\
1
\end{bmatrix}
=
e^{-t} (\cos(4 t) + i \sin(4 t))
\begin{bmatrix}
i	\\
1
\end{bmatrix}
\\
&=
e^{-t}
\begin{bmatrix}
-\sin(4 t) + i \cos(4 t)	\\
\cos(4 t) + i \sin(4 t)
\end{bmatrix}
=
\underbrace{e^{-t}
\begin{bmatrix}
-\sin(4 t)	\\
\cos(4 t)
\end{bmatrix}%
}_{\fontMatrix{x}_{\realPart}}
+
i \underbrace{e^{-t}
\begin{bmatrix}
\cos(4 t)	\\
\sin(4 t)
\end{bmatrix}%
}_{\fontMatrix{x}_{\imaginaryPart}}.
\end{align*}
(Note that our definition of $\fontMatrix{x}_{\imaginaryPart}$ does not include the coefficient $i$.) The general real solution to \eqref{eq : Quiz21 ODE} is the real linear combination of $\fontMatrix{x}_{\realPart}$ and $\fontMatrix{x}_{\imaginaryPart}$, i.e.
\begin{align*}
\fontMatrix{x}(t)
=
c_{1} \fontMatrix{x}_{\realPart}(t) + c_{2} \fontMatrix{x}_{\imaginaryPart}(t)
=
c_{1} e^{-t}
\begin{bmatrix}%
-\sin(4 t)	\\
\cos(4 t)
\end{bmatrix}%
+
c_{2} e^{-t}
\begin{bmatrix}%
\cos(4 t)	\\
\sin(4 t)
\end{bmatrix}%
,
\end{align*}
where $c_{1},c_{2} \in \reals$.}% End solution.

\begin{enumerate}[resume,label=(\alph*)]
\item (1 pt) Quickly sketch the phase plane. Your sketch need not be exact; focus on whether trajectories move toward or away from the origin, and the direction of rotation.
\end{enumerate}

\spaceSolution{2in}{% Begin solution.
The real part of both eigenvalues $\lambda_{\pm{}}$ is $-1$, which is negative; thus trajectories move toward the origin (as time increases). To determine the direction of rotation, we can use the original ODE \eqref{eq : Quiz21 ODE} to compute the tangent vectors $\fontMatrix{x}'$ at a few points $(x_{1},x_{2})$:
\begin{align*}
\fontMatrix{x}'
=
\begin{bmatrix}%
-1	&	-4	\\
4	&	-1
\end{bmatrix}%
\begin{bmatrix}%
1	\\
1
\end{bmatrix}%
=
\begin{bmatrix}%
-5	\\
3
\end{bmatrix}%
,
&&
\fontMatrix{x}'
=
\begin{bmatrix}%
-1	&	-4	\\
4	&	-1
\end{bmatrix}%
\begin{bmatrix}%
-1	\\
1
\end{bmatrix}%
=
\begin{bmatrix}%
-3	\\
-5
\end{bmatrix}%
.
\end{align*}
This shows that, at $(x_{1},x_{2}) = (1,1)$, tangent vectors point left ($x_{1}' = -5 < 0$) and up ($x_{2}' = 3 > 0$); at $(x_{1},x_{2}) = (-1,1)$, tangent vectors point left ($x_{1}' = -3 < 0$) and down ($x_{2}' = -5 < 0$). Thus trajectories rotate counterclockwise.

A phase plane for the linear system \eqref{eq : Quiz21 ODE} is shown in Figure \ref{fig : Quiz21 Phase Plane}.
\begin{figure}[b]
\begin{center}
\includegraphics[scale=0.5]{\filePathGraphics Quiz21_PhasePlane}
\caption{Phase plane, in the $(x_{1},x_{2})$ plane, for the homogeneous 1st-order linear system \eqref{eq : Quiz21 ODE}. Arrows point in the direction of increasing $t$.}
\label{fig : Quiz21 Phase Plane}
\end{center}
\end{figure}
}% End solution.
%
% WolframAlpha code :
% eigenvalues {{-1,-4},{4,-1}}