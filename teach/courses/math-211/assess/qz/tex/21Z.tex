%
% Quiz 21 : 2019-08-05 (M)
%
% Phase planes plotted using online software at
% http://www.calvin.edu/~scofield/demos/other/PhasePortrait2D.html
%
\section{Exercise}

(5 pt) Consider the homogeneous 1st-order $2 \times 2$ linear system of ODEs
\begin{align}
\fontMatrix{x}'
=
\begin{bmatrix}%
-1	&	-4	\\
4	&	1
\end{bmatrix}%
\fontMatrix{x}.%
\label{eq : Quiz21Z ODE}
\end{align}
\begin{enumerate}[label=(\alph*)]
\item (2 pt) Diagonalize the coefficient matrix (i.e. compute the eigenvalues and corresponding eigenvectors of the given matrix $\fontMatrix{A}$, and write it in the form $\fontMatrix{A} = \fontMatrix{P} \fontMatrix{D} \fontMatrix{P}^{-1}$).
\end{enumerate}

\spaceSolution{1.75in}{% Begin solution.
Denote the coefficient matrix in \eqref{eq : Quiz21Z ODE} by
\begin{align*}
\fontMatrix{A}
=
\begin{bmatrix}%
-1	&	-4	\\
4	&	1
\end{bmatrix}%
.
\end{align*}

\textbf{Eigenvalues}. We compute
\begin{align*}
\lambda_{\pm{}}
=
\pm{}i \sqrt{15}.
\end{align*}

\textbf{Eigenvectors.} We compute
\begin{align*}
\fontMatrix{v}_{+}
=
\begin{bmatrix}%
-1 + i \sqrt{15}	\\
4
\end{bmatrix}%
,
&&
\fontMatrix{v}_{-}
=
\begin{bmatrix}%
-1 - i \sqrt{15}	\\
4
\end{bmatrix}%
.
\end{align*}
Any nonzero (complex) scalar multiples of these vectors will also be eigenvectors. Note that once we compute one of the two eigenvectors $\fontMatrix{v}_{+},\fontMatrix{v}_{-}$, we can obtain the other immediately by taking the complex conjugate of the first.

\textbf{Diagonalization.} Using these results, we conclude
\begin{align}
\begin{bmatrix}%
-1	&	-4	\\
4	&	1
\end{bmatrix}%
=
\begin{bmatrix}%
-1 + i \sqrt{15}	&	-1 - i \sqrt{15}	\\
4			&	4
\end{bmatrix}%
\begin{bmatrix}%
i \sqrt{15}	&	0		\\
0		&	-i \sqrt{15}
\end{bmatrix}%
\begin{bmatrix}%
-1 + i \sqrt{15}	&	-1 - i \sqrt{15}	\\
4			&	4
\end{bmatrix}%
^{-1}.%
\label{eq : Quiz21Z Diagonalization}
\end{align}
Call the matrices in this equation as follows:
\begin{align*}
\fontMatrix{A}
=
\fontMatrix{P} \fontMatrix{D} \fontMatrix{P}^{-1}.
\end{align*}}% End solution.

\begin{enumerate}[resume,label=(\alph*)]
\item (2 pt) Use this information to write the general real solution $\fontMatrix{x}(t)$ to \eqref{eq : Quiz21Z ODE}. \fontHint{Recall that when the eigenvalues of a homogeneous $2 \times 2$ linear system are complex, we can compute one solution $e^{\lambda t} v$, just as we did with real eigenvalues; decompose it into real and imaginary parts; then use these parts as our basis for the solution space. The formulas
\begin{align*}
e^{a + i b}
=
e^{a} e^{i b},
&&
e^{i b}
=
\cos b + i \sin b,
\end{align*}
will be useful.}
\end{enumerate}

\spaceSolution{2.25in}{% Begin solution.
Call columns 1 and 2 of $\fontMatrix{P}$ $\fontMatrix{v}_{+}$ and $\fontMatrix{v}_{-}$, respectively. (They are eigenvectors corresponding to the eigenvalues $\lambda_{+} = i \sqrt{15}i$ and $\lambda_{-} = -i \sqrt{15}$, respectively. Hence our choice of subscripts $+$ and $-$.) Our diagonalization \eqref{eq : Quiz21Z Diagonalization} gives one basis for the vector space of complex (!) solutions to \eqref{eq : Quiz21Z ODE}, namely, the vectors
\begin{align*}
\fontMatrix{x}_{+}(t)
=
e^{\lambda_{+} t} \fontMatrix{v}_{+}
=
e^{i \sqrt{15} t}
\begin{bmatrix}%
-1 + i \sqrt{15}	\\
4
\end{bmatrix}%
,
&&
\fontMatrix{x}_{-}(t)
=
e^{\lambda_{-} t} \fontMatrix{v}_{-}
=
e^{-i \sqrt{15} t}
\begin{bmatrix}%
-1 - i \sqrt{15}	\\
4
\end{bmatrix}%
.
\end{align*}
To describe the real (!) solutions to \eqref{eq : Quiz21Z ODE}, we can decompose either one of these complex basis vectors into real and imaginary parts, $\fontMatrix{x}_{\realPart}$ and $\fontMatrix{x}_{\imaginaryPart}$, and take these as our basis vectors. Decomposing $\fontMatrix{x}_{+}(t)$, using the formulas in the hint, we find
\begin{align*}
\fontMatrix{x}_{+}(t)
&=
e^{i \sqrt{15} t}
\begin{bmatrix}
-1 + i \sqrt{15}	\\
4
\end{bmatrix}
=
e^{0 t} e^{i \sqrt{15} t}
\begin{bmatrix}
-1 + i \sqrt{15}	\\
4
\end{bmatrix}
=
(\cos(\sqrt{15} t) + i \sin(\sqrt{15} t))
\begin{bmatrix}
-1 + i \sqrt{15}	\\
4
\end{bmatrix}
\\
&=
\left[
\begin{array}{c c c}
\left(-\cos(\sqrt{15} t) - \sqrt{15} \sin(\sqrt{15} t)\right)	&	+ i	&	\left(\sqrt{15} \cos(\sqrt{15} t) - \sin(\sqrt{15} t)\right)	\\
4 \cos(\sqrt{15} t)							&	+ i	&	\sin(\sqrt{15} t)
\end{array}
\right]
\\
&=
\underbrace{
\begin{bmatrix}%
-\cos(\sqrt{15} t) - \sqrt{15} \sin(\sqrt{15} t)	\\
4 \cos(\sqrt{15} t)
\end{bmatrix}%
}_{\fontMatrix{x}_{\realPart}}
+
i \underbrace{
\begin{bmatrix}%
\sqrt{15} \cos(\sqrt{15} t) - \sin(\sqrt{15} t)	\\
\sin(\sqrt{15} t)
\end{bmatrix}%
}_{\fontMatrix{x}_{\imaginaryPart}}.
\end{align*}
(Note that our definition of $\fontMatrix{x}_{\imaginaryPart}$ does not include the coefficient $i$.) The general real solution to \eqref{eq : Quiz21Z ODE} is the real linear combination of $\fontMatrix{x}_{\realPart}$ and $\fontMatrix{x}_{\imaginaryPart}$, i.e.
\begin{align*}
\fontMatrix{x}(t)
&=
c_{1} \fontMatrix{x}_{\realPart}(t) + c_{2} \fontMatrix{x}_{\imaginaryPart}(t)
\\
&=
c_{1}
\begin{bmatrix}%
-\cos(\sqrt{15} t) - \sqrt{15} \sin(\sqrt{15} t)	\\
4 \cos(\sqrt{15} t)
\end{bmatrix}%
+
c_{2}
\begin{bmatrix}%
\sqrt{15} \cos(\sqrt{15} t) - \sin(\sqrt{15} t)	\\
\sin(\sqrt{15} t)
\end{bmatrix}%
,
\end{align*}
where $c_{1},c_{2} \in \reals$.}% End solution.

\begin{enumerate}[resume,label=(\alph*)]
\item (1 pt) Quickly sketch the phase plane. Your sketch need not be exact; focus on whether trajectories move toward or away from the origin, and the direction of rotation.
\end{enumerate}

\spaceSolution{2in}{% Begin solution.
The real part of both eigenvalues $\lambda_{\pm{}}$ is $0$; thus trajectories are closed ellipses. To determine the direction of rotation, we can use the original ODE \eqref{eq : Quiz21Z ODE} to compute the tangent vectors $\fontMatrix{x}'$ at a few points $(x_{1},x_{2})$:
\begin{align*}
\fontMatrix{x}'
=
\begin{bmatrix}%
-1	&	-4	\\
4	&	1
\end{bmatrix}%
\begin{bmatrix}%
1	\\
1
\end{bmatrix}%
=
\begin{bmatrix}%
-5	\\
5
\end{bmatrix}%
,
&&
\fontMatrix{x}'
=
\begin{bmatrix}%
-1	&	-4	\\
4	&	1
\end{bmatrix}%
\begin{bmatrix}%
-1	\\
1
\end{bmatrix}%
=
\begin{bmatrix}%
-3	\\
-3
\end{bmatrix}%
.
\end{align*}
This shows that, at $(x_{1},x_{2}) = (1,1)$, tangent vectors point left ($x_{1}' = -5 < 0$) and up ($x_{2}' = 5 > 0$); at $(x_{1},x_{2}) = (-1,1)$, tangent vectors point left ($x_{1}' = -3 < 0$) and down ($x_{2}' = -3 < 0$). Thus trajectories rotate counterclockwise.

A phase plane for the linear system \eqref{eq : Quiz21Z ODE} is shown in Figure \ref{fig : Quiz21 Phase Plane}.
\begin{figure}[b]
\begin{center}
\includegraphics[scale=0.5]{\filePathGraphics Quiz21Z_PhasePlane}
\caption{Phase plane, in the $(x_{1},x_{2})$ plane, for the homogeneous 1st-order linear system \eqref{eq : Quiz21Z ODE}. Arrows point in the direction of increasing $t$.}
\label{fig : Quiz21 Phase Plane}
\end{center}
\end{figure}
}% End solution.
%
% WolframAlpha code :
% eigenvalues {{-1,-4},{4,-1}}