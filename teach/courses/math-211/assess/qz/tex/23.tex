%
% Quiz 23 : 2019-08-07 (W)
%
% Exercise 8.2.4 (p 483) in Farlow et al.
%
\section{Exercise}

(5 pt) Solve the following nonhomogeneous 1st-order linear initial value problem, using the laplace transform:
\begin{align}
y' + y
=
e^{-t},
&&
y(0)
=
-1.%
\label{eq : Quiz23 IVP}
\end{align}
\fontHint{Recall that, from the definition of the lapace transform,
\begin{align*}
\laplaceTransform\{y'\}(s)
=
s \laplaceTransform\{y\} - y(0).
\end{align*}
The following transform--inverse-transform pairs may be useful:
\begin{align*}
\laplaceTransform\left\{e^{a t}\right\}
&=
\frac{1}{s - a},
\qquad
s > a;
&&
\laplaceTransform\left\{t^{n} e^{a t}\right\}
=
\frac{n!}{(s - a)^{n + 1}},
\qquad
s > a.
\end{align*}}

\spaceSolution{5in}{% Begin solution.
Applying the laplace transform to both sides of the ODE in \eqref{eq : Quiz23 IVP}, using the fact that the laplace transform is linear, we get
\begin{align}
\laplaceTransform\{y'\} + \laplaceTransform\{y\}
=
\laplaceTransform\{y' + y\}
=
\laplaceTransform\{e^{-t}\}.
\label{eq : Quiz23 Laplace Transform Intermediate}
\end{align}
Denote
\begin{align*}
Y(s)
=
\laplaceTransform\{y\}.
\end{align*}
From the definition of the laplace transform, and using the given initial condition, we compute
\begin{align*}
\laplaceTransform\{y'\}
=
s \laplaceTransform\{y\} - y(0)
=
s Y(s) - (-1)
=
s Y(s) + 1.
\end{align*}
Using the transform--inverse-transform dictionary, we see that
\begin{align*}
\laplaceTransform\{e^{-t}\}
=
\laplaceTransform\{e^{(-1) t}\}
=
\frac{1}{s - (-1)}
=
\frac{1}{s + 1}.
\end{align*}
Substituting these results into \eqref{eq : Quiz23 Laplace Transform Intermediate}, we get
\begin{align*}
\left(s Y(s) + 1\right) + Y(s)
=
\frac{1}{s + 1}.
\end{align*}
Solving this equation for $Y(s)$, we find
\begin{align}
Y(s)
=
\frac{\frac{1}{s + 1} - 1}{s + 1}
=
\frac{1}{(s + 1)^{2}} - \frac{1}{s + 1}.%
\label{eq : Quiz23 Y(s)}
\end{align}
We can't do partial fraction decomposition on fractions of this form. Nor do we need to. These transforms appear directly in our transform--inverse-transform dictionary: The first is the laplace transform of $t^{n} e^{a t}$, with $n = 1$ and $a = -1$; the second is the laplace transform of $e^{a t}$ with $a = -1$. Applying the inverse laplace transform $\laplaceTransform^{-1}$ to both sides of \eqref{eq : Quiz23 Y(s)}, and using the fact that $\laplaceTransform^{-1}$ is linear, we find
\begin{align*}
y
&=
\laplaceTransform^{-1}\left\{\laplaceTransform\left\{y\right\}\right\}
=
\laplaceTransform^{-1}\left\{Y(s)\right\}
=
\laplaceTransform^{-1}\left\{\frac{1}{(s + 1)^{2}} - \frac{1}{s + 1}\right\}
\\
&=
\laplaceTransform^{-1}\left\{\frac{1}{(s + 1)^{2}}\right\} - \laplaceTransform^{-1}\left\{\frac{1}{s + 1}\right\}
=
t e^{-t} - e^{-t}.
\end{align*}

One can check that this function $y(t)$ indeed solves the IVP \eqref{eq : Quiz23 IVP}, by plugging it in and verifying that the equation is true. One can also check that we obtain the same solution $y(t)$ if we solve \eqref{eq : Quiz23 IVP} by other means, e.g., using an integrating factor, variation of parameters, or linearity combined with guessing a particular solution (in this case, one would have to guess $y_{p}(t) = t e^{-t}$).}% End solution.