% https://ga.rice.edu/syllabus/
% https://registrar.rice.edu/students/syllabus/
% http://cte.rice.edu/syllabus/



\begin{center}
{\Large Math 211: Ordinary Differential Equations}\\
{\scriptsize (draft version : last updated : \Year--\Month--\Day)}
\end{center}





\section{Disclaimer}

The information contained in this syllabus, other than the absence policy, is subject to change with reasonable advance notice.



%
%
%	Course information
%
%

\section{Course Information}



\subsection{Course Logistics}

\begin{tabular}{r c l}
Class time 	&	:	&	Monday -- Friday, 10h30 -- 12h00	\\
Class room	&	:	&	\href{https://goo.gl/maps/CfwNovWqwqMdsBUV6}{George R. Brown Hall} (GRB) W211 (2F)	\\
Office hours	&	:	&	Tuesday : 14h30 -- 15h45 ; Thursday : 16h30 -- 17h45
\end{tabular}



\subsection{Instructor Information}


\begin{tabular}{r c l}
Instructor	&	:	&	Stephen Wolff	\\
Office 	&	:	&	HBH 038 (basement)	\\
E-mail	&	:	&	\href{mailto:Stephen.Wolff@rice.edu?subject=[Math\%20211]}{Stephen.Wolff@rice.edu}
\end{tabular}



\subsection{Textbook}

This course is based on the following textbook:
\begin{itemize}
\item \href{https://www.pearson.com/us/higher-education/program/Farlow-Differential-Equations-and-Linear-Algebra-Classic-Version-2nd-Edition/PGM1714691.html}{\fontBookTitle{Differential equations and linear algebra}}, 2nd edition, by Farlow, Hall, McDill, \&{} West
\end{itemize}
We will wrestle with the following topics (numbers refer to chapters in Farlow et al.):
\begin{multicols}{2}
\begin{enumerate}
\setcounter{enumi}{0}
\item First-order ODEs
\item Linearity
\item Linear algebra
\item Higher-order linear ODEs
\item Linear transformations
\item Linear systems
\item Nonlinear systems
\item Laplace transforms
\end{enumerate}
\end{multicols}
\noindent{}You are not required to purchase a copy of the textbook. Your understanding will be well served if you secure access to material covering the topics listed above.





%
%
%	Grading policy
%
%

\subsection{Grading Policy}

We will assess progress using the following tools:
\begin{enumerate}
\item Homework (H). There will be homework assigned daily. Homework is due promptly by 10h35 the following class. Late homework will not be accepted. Homework will be graded using a ``reasonable attempt'' rubric.
\item Quizzes (Q). There will be an in-class quiz every day. Quizzes come in two flavors: short quizzes (SQ) and long quizzes (LQ). Short quizzes are 1- or 2-minute quizzes on key concepts (to help clarify key ideas). Long quizzes are 5- to 15-minute quizzes asking you to work on an exercise (exam practice). You have the opportunity to retake long quizzes.
\item Exams (E). There will be three midterm exams (ME) and one final exam (FE). Each exam will be cumulative. The final day of classes, each student will have a 20--30 minute oral exam --- a chat, really. See page \pageref{sec: Calendar} for exam dates.
\item \LaTeX{} project (L). There will be one team project. Your team will solve past exam questions and typeset your solutions. I will provide a template and assistance.
\end{enumerate}

\noindent{}\emph{How your grade is calculated.}
Your final average is allocated among the above tools as follows: H : 10\% ; SQ : 10\% ; LQ : 10\% ; each ME : $\frac{40}{3}$\% ; L : 10\%
The final exam applies to the remaining 20\% plus any yet-unsecured part of the other 80\% (except the team project).

For example, suppose that prior to the final exam, you submit all homework, earn an average of 50\% on both types of quizzes, score 75\% on each midterm exam, and earn 80\% on the \LaTeX{} project. Then you have secured (denoting final-grade $\%$ by $P$)
\begin{align*}
\underbrace{100\% \times 10 P}_{H} + \underbrace{50\% \times 10 P}_{SQ} + \underbrace{50\% \times 10 P}_{LQ} + \underbrace{3 \times \left(75\% \times \frac{40}{3} P\right)}_{ME} + \underbrace{80\% \times 10 P}_{L}
%=
%10 P + 5 P + 5 P + 3 \times 10 P + 8 P
=
60 P.
\end{align*}
The final exam is worth 40 P (20 P plus the unsecured $5 P + 5 P + 3 \times (25\% \times \frac{40}{3} P) = 20 P$ from all non-L components).





\subsection{Absence Policy}

Class attendance is strongly encouraged but not required. We are old enough to accept responsibility for our actions and decisions.





\section{Rice Honor Code}

As a student at Rice University, you pledge to uphold the Rice Honor Code, which you can find in the \href{http://honor.rice.edu/honor-system-handbook/}{Honor System Handbook}.

On homework, all resources are permitted. In particular, you are strongly encouraged to work with one another. The purpose of homework is to help you to learn and internalize the material.

On quizzes and exams, no external resources are permitted, unless the instructor explicitly indicates otherwise. The purpose of quizzes and exams is to help you to see what you can do so far and identify what to work on.





\section{Students with Disabilities}

Any student with a documented disability that requires accommodation is encouraged to contact both the course instructor (\href{mailto:Stephen.Wolff@rice.edu?subject=[Math\%20211]}{Stephen.Wolff@rice.edu}) and the \href{https://drc.rice.edu/}{Rice Disability Resource Center} (\href{mailto:adarice@rice.edu}{adarice@rice.edu}; Allen Center, Room 111).





%
%
%	Course objectives
%
%

\section{Course Objectives and Expected Learning Outcomes}

By the end of this course, you should know how to
\begin{itemize}
\item Analyze and create direction fields.
\item Analyze the existence and uniqueness of solutions to ODEs and linear systems.
\item Use the superposition and nonhomogeneous principles, and where they apply.
\item Use techniques for solving first-order ordinary differential equations (ODEs).
\item Use fundamental concepts in linear algebra, including the determinant, Gauss--Jordan reduction algorithm, rank--nullity theorem, eigenvalues and eigenspaces.
\item Analyze second-order ODEs, including the characteristic equation and equivalent linear system.
\item Analyze linear systems of first-order ODEs.
\item Analyze and create phase portraits of linear systems.
\item Linearize nonlinear systems of first-order ODEs, and why.
\item Use the Laplace transform to solve ODEs.
\end{itemize}





%
%
%	Resources
%
%

\section{Resources}

Potentially helpful resources:
\begin{itemize}
\item Past exams are hosted on the \href{https://canvas.rice.edu/courses/9592/files}{Calculus Resources page} on Canvas. See the page ``\href{https://math.rice.edu/exam-help}{Math Exam Help}'' for log-in instructions. (You will need a valid Rice NetID.)
\end{itemize}




%
%
%	Calendar
%
%


\newpage

\section{Calendar}
\label{sec: Calendar}

Below is a preliminary schedule of topics. Section numbers and exercise numbers refer to Farlow et al. (2e) unless indicated otherwise. Exercises are \emph{assigned} on the date of the line on which they appear and are \emph{due} the next class.
\begin{center}
\begingroup%
%\resizebox{\textwidth}{!}{%
%\scriptsize
\footnotesize
\begin{tabular}{*5{l}}
\hline\hline
Day	&	Date		&	Topics						&	Sections	&	Exercises	\\
\hline
M	&	08 Jul	&	Diagnostic quiz; Modeling; Slope fields	&	1.1--1.2	&	\fontSectionNumber{1.2.}03,07,16--21,39	\\
T	&	09 Jul	&	Separation of variables; Numerical methods	&	1.2--1.4	&	\fontSectionNumber{1.2.}05,65;\fontSectionNumber{1.3.}07,09,17,23	\\
W	&	10 Jul	&	Numerical methods; Existence, uniqueness	&	1.4--1.5	&	\fontSectionNumber{1.3.}25--30;\fontSectionNumber{1.4.}05;\fontSectionNumber{1.5.}01,05	\\
R	&	11 Jul	&	Vector spaces; Linear equations	&	1.5,2.1	&	\fontSectionNumber{1.5.}09,15,17;\fontSectionNumber{2.1.}03,09,27,37,45	\\
F	&	12 Jul	&	Applications					&	2.2--2.4	&	\fontSectionNumber{2.2.}03,23;\fontSectionNumber{2.3.}07;\fontSectionNumber{2.4.}07	\\
\hline
M	&	15 Jul	&	Applications; Phase line			&	2.2--2.5	&	\fontSectionNumber{2.2.}47;\fontSectionNumber{2.5.}1,3,22,24,36	\\
T	&	16 Jul	&	Bifurcation; Review; Matrices		&	2.5,3.1	&	\fontSectionNumber{3.1.}7,13,25,59	\\
W	&	17 Jul	&	Matrices; Row reduction			&	3.1--3.2	&	\fontSectionNumber{3.2.}25,62	\\
\fontExam{W}	&	\fontExam{17 Jul}	&	\fontExam{Exam 1 (p.m.)}	&	\fontExam{1.1--2.5}	&		\\
R	&	18 Jul	&	Row reduction; Matrix inverse	 	&	3.2--3.3	&	\fontSectionNumber{3.2.}33,46;\fontSectionNumber{3.3.}07,22	\\
F	&	19 Jul	&	Determinant; Vector spaces		&	3.4--3.6	&	\fontSectionNumber{3.4.}03,15,35;\fontSectionNumber{3.5.}13,19,25,41;\fontSectionNumber{3.6.}49--53	\\
\hline
M	&	22 Jul	&	Linear maps; Eigendinge			&	5.1--5.3	&	\fontSectionNumber{3.6.}21,32;\fontSectionNumber{5.1.}35,37;\fontSectionNumber{5.2.}31,37,70	\\
T	&	23 Jul	&	Eigendinge; Diagonalization; Review	&	5.3--5.4	&	\fontSectionNumber{5.3.}13,15,21,31,37*;\fontSectionNumber{5.4.}31,33,49*	\\
W	&	24 Jul	&	Undetermined coefficients;		&	4.4		&	\fontSectionNumber{4.4.}25,33,45;	\\
	&			&	Variation of parameters			&	4.5		&	\fontSectionNumber{4.5.}3,13		\\
\fontExam{W}	&	\fontExam{24 Jul}	&	\fontExam{Exam 2 (p.m.)}	&	\fontExam{1.1--5.4}	&	\fontExam{(no Chapter 4)}	\\
R	&	25 Jul	&	Higher-order ODEs; Linear systems	&	6.1--6.3	&	\fontSectionNumber{6.1.}7,17,19	\\
F	&	26 Jul	&	Real and complex eigenvalues		&	4.2--4.3	&	\fontSectionNumber{4.2.}7,9,21;\fontSectionNumber{4.3.}1,3,13,19;	\\
	&			&								&	6.2--6.3	&	\fontSectionNumber{6.2.}19,21,23,25;\fontSectionNumber{6.3.}3,13	\\
\hline
M	&	29 Jul	&	Trace--determinant plane			&	6.4	&	\fontSectionNumber{6.2.}5--8;\fontSectionNumber{6.4.}1--6	\\
T	&	30 Jul	&	Decoupling; Matrix exponential; Review 	&	6.5--6.6		&	\fontSectionNumber{6.5.}1,13;\fontSectionNumber{6.6.}1,11,15	\\
W	&	31 Jul	&	Nonhomogeneous linear systems	&	6.7	&	\fontSectionNumber{}	\\
\fontExam{W}	&	\fontExam{31 Jul}	&	\fontExam{Exam 3 (p.m.)}	&	\fontExam{1.1--6.6}	&		\\
R	&	01 Aug	&	Nonlinear systems; Linearization	&	7.1--7.2	&	\fontSectionNumber{}	\\
F	&	02 Aug	&	Laplace transform				&	8.1--8.2	&	\LaTeX{} project\fontSectionNumber{}	\\
\hline
M	&	05 Aug	&	Delta functions; Convolution		&	8.2--8.4	&	\fontSectionNumber{}	\\
T	&	06 Aug	&	Solutions with Laplace; Qualitative theory	&	8.5	&	Practice exam (\fontNeedsEdit{Sem YYYY})	\\
W	&	07 Aug	&	Review	&	All	&		\\
\fontExam{R}	&	\fontExam{08 Aug}	&	\fontExam{Final exam (in-class)}	&	\fontExam{1.1--8.5}	&		\\
\fontExam{F}	&	\fontExam{09 Aug}	&	\fontExam{Oral exam (a.m. \& p.m.)}	&	\fontExam{1.1--8.5}	&		\\
\hline
\end{tabular}
\endgroup
\end{center}