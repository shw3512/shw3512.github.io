%
%
%	Exercise 1
%
%

\section{Exercise \ref{sec: Math212 2016 Fall HalfExam01Q01}}
\label{sec: Math212 2016 Fall HalfExam01Q01}

\pagenumbering{arabic}

(20 pt) Find an equation for the plane containing both (i) the point of intersection of the lines
\begin{align*}
L_{1}&:
x
=
\frac{y + 1}{3}
=
1 - z,
&
L_{2}&:
3 - x
=
\frac{y + 2}{2}
=
\frac{z + 4}{2}
\end{align*}
and (ii) the line of intersection of the planes
\begin{align*}
H_{1}&:
x + y + z
=
6,
&
H_{2}&:
2 y - x
=
0.
\end{align*}
\fontHint{Break it down. First find the coordinates of the point $P_{0}$ of intersection of the lines $L_{1}$ and $L_{2}$. Also find an equation for the line $L$ of intersection of the planes $H_{1}$ and $H_{2}$. Then put it all together to get an equation for the plane containing $P_{0}$ and $L$. Think geometrically!}



\spaceSolution{6in}{% Begin solution.
\subsection{Solution}

We know an easy way to write an equation for a plane in $\reals^{3}$:% Begin footnote.
\footnote{This approach generalizes nicely to hyperplanes in $\reals^{n}$. Do you see how?} % End footnote.
Given a point $P_{0} = (x_{0},y_{0},z_{0})$ in the plane and a normal vector $\fontVector{n} = (n_{1},n_{2},n_{3})$ to the plane, any point $P = (x,y,z)$ in the plane satisfies
\begin{align}
0
=
\fontVector{n} \cdot \fontVectorBetweenPoints{P_{0} P}
=
(n_{1},n_{2},n_{3}) \cdot \left(x - x_{0},y - y_{0},z - z_{0}\right).%
\label{eq: General Equation Plane R3}
\end{align}
Let's use the given information to find a $P_{0}$ and an $\fontVector{n}$.



\subsubsection{Point $P_{0}$ of intersection of $L_{1}$ and $L_{2}$}

By definition, the point of intersection of the lines $L_{1}$ and $L_{2}$ is a point $P_{0} = (x_{0},y_{0},z_{0})$ whose coordinates satisfy the equation for both $L_{1}$ and $L_{2}$. We can find this point in at least two ways.

\paragraph{Method 1: Using the symmetric equations}

The key idea is that \emph{the same point} $P_{0} = (x_{0},y_{0},z_{0})$ satisfies both sets of symmetric equations. Thus we can equate two expressions for the same variable (one expression from each set of symmetric equations) and then solve. For the variables $x,y$, we have
\begin{align*}
\frac{y_{0} + 1}{3}
=
x_{0}
&=
3 - \frac{y_{0} + 2}{2}
&
&\Leftrightarrow
&
2 y_{0} + 2
&=
18 - 3 y_{0} - 6
&
&\Leftrightarrow
&
y_{0}
&=
2.
\end{align*}
We can now use any of the equalities to find the remaining coordinates $x_{0},z_{0}$. From the symmetric equations for $L_{1}$, for example,
\begin{align*}
x_{0}
&=
\frac{y_{0} + 1}{3}
=
1,
&
z_{0}
&=
1 - x_{0}
=
0.
\end{align*}
We conclude that $P_{0} = (1,2,0)$.

\paragraph{Method 2: Using the parametric equations}

Recall that the symmetric equations of a line are obtained by eliminating the parameter, call it $t$, Thus all the expressions in the symmetric equations equal $t$. Solving for $x,y,z$ as functions of $t$, we find
\begin{align*}
\fontVector{r}_{1}(t)
&=
\left(t,-1 + 3 t,1 - t\right),
&
\fontVector{r}_{2}(t)
&=
\left(3 - t,-2 + 2 t,-4 + 2 t\right),
\end{align*}
where $\fontVector{r}_{i}$ is a vector equation for the line $L_{i}$, for $i \in \{1,2\}$. The key idea is that, to say that $L_{1}$ and $L_{2}$ have a point of intersection (i.e. a point in common) is equivalent to saying that there exist parameter values $t_{1}$ and $t_{2}$ such that
\begin{align*}
\fontVector{r}_{1}(t_{1})
=
\fontVector{r}_{2}(t_{2}).
\end{align*}
Writing out what this vector equality means in terms of components, we obtain a system of three equations (one for each component) in two unknowns (the parameters $t_{1},t_{2}$):
\begin{align}
t_{1}
&=
3 - t_{2},
&
-1 + 3 t_{1}
&=
-2 + 2 t_{2},
&
1 - t_{1}
&=
-4 + 2 t_{2}.%
\label{eq: System Of Equations For Point Of Intersection}
\end{align}
Substituting the first equation into the second, we find that if a solution to this system exists (i.e. if $L_{1}$ and $L_{2}$ have a point of intersection), then
\begin{align*}
-1 + 3 (3 - t_{2})
&=
-2 + 2 t_{2}
&
&\Leftrightarrow
&
10
&=
5 t_{2}
&
&\Leftrightarrow
&
t_{2}
&=
2.
\end{align*}
This is the parameter value for $L_{2}$ corresponding to the candidate intersection point. Substituting this into the first equation, we find the corresponding parameter value for $L_{1}$:
\begin{align*}
t_{1}
=
3 - t_{2}
=
1.
\end{align*}
To verify that these parameter values indeed correspond to a point of intersection, we need to check that they satisfy the remaining inequalities --- in this case, just the third equality (corresponding to the $z$-coordinate):
\begin{align*}
1 - t_{1}
=
0
=
-4 + 2 t_{2}.
\end{align*}
Thus all three equalities in \eqref{eq: System Of Equations For Point Of Intersection} are satisfied when $t_{1} = 1$ and $t_{2} = 2$, i.e. $\fontVector{r}_{1}(t_{1}) = \fontVector{r}_{2}(t_{2})$, and we have a point of intersection. Its coordinates are found by evaluating $\fontVector{r}_{i}(t_{i})$ for either value of $i$:
\begin{align*}
P_{0}
=
\fontVector{r}_{1}(t_{1})
=
(1,2,0).
\end{align*}



\subsubsection{Line $L$ of intersection of $H_{1}$ and $H_{2}$}

Close your eyes and view the geometric picture of two different planes intersecting. You will see that they intersect in a line. Algebraically, we can describe that line by giving a point on the line and a direction vector.

Any point $P'_{0} = (x'_{0},y'_{0},z'_{0})$ on the line of intersection of $H_{1}$ and $H_{2}$ will do. Analogous to our discussion for the point of intersection of two lines, a point on the line of intersection of two planes satisfies the equation describing both planes. This gives us two equations (one for each plane) in three unknowns ($x'_{0},y'_{0},z'_{0}$). We can choose an arbitrary value for one of these variables,% Begin footnote.
\footnote{\fontNeedsEdit{(discuss subtlety, e.g., a line whose $z$ values are constant)}} % End footnote.
say $z_{0} = 0$. This reduces the system of equations to
\begin{align*}
x_{0} + y_{0}
&=
6,
&
2 y_{0} - x_{0}
&=
0.
\end{align*}
Solving the second equation for $x_{0}$ and substituting the result into the first, we have
\begin{align*}
(2 y_{0}) + y_{0}
&=
6
&
&\Leftrightarrow
&
y_{0}
&=
2.
\end{align*}
Hence $x_{0} = 2 y_{0} = 4$, so
\begin{align*}
P'_{0}
=
(4,2,0).
\end{align*}

A direction vector of the line of intersection is even easier to find. Again, think geometrically: The line of intersection lies in both planes, so in particular, a (nonzero) direction vector of that line lies in both planes. Because a normal vector to a plane is perpendicular to all vectors in the plane, we conclude that the direction vector of the line of intersection of two planes must be normal to \emph{both} normal vectors. And we know a way, if we are given two nonparallel vectors, to construct a vector normal (a.k.a. perpendicular) to both: take their cross product.

The components of normal vectors $\fontVector{n}_{1}$ and $\fontVector{n}_{2}$ of the planes $H_{1}$ and $H_{2}$, respectively, can be read off from the corresponding coefficients of the variables $x,y,z$ in their defining equations:
\begin{align*}
\fontVector{n}_{1}
&=
(1,1,1),
&
\fontVector{n}_{2}
&=
(-1,2,0).
\end{align*}
By the preceding discussion, a direction vector $\fontVector{v}$ for the line of intersection is
\begin{align*}
\fontVector{v}
=
\fontVector{n}_{1} \times \fontVector{n}_{2}
=
\det
\begin{pmatrix}
\fontVector{i}	&	\fontVector{j}	&	\fontVector{k}	\\
1			&	1			&	1			\\
-1			&	2			&	0
\end{pmatrix}
=
(-2,-1,3).
\end{align*}



\subsubsection{Putting it all together}

At this point we have (i) the point
\begin{align*}
P_{0}
=
(1,2,0)
\end{align*}
of intersection of the lines $L_{1}$ and $L_{2}$ and (ii) an equation
\begin{align*}
\fontVector{r}(t)
=
(4,2,0) + t (-2,-1,3)
\end{align*}
for the line $L$ of intersection of the planes $H_{1}$ and $H_{2}$. We want an equation for the plane containing both $P_{0}$ and $L$. Note that $P_{0}$ does not lie in $L$.% Begin footnote.
\footnote{You can check this by substituting the coordinates of $P_{0}$ for $\fontVector{r}(t)$ and trying to solve the resulting vector equation
\begin{align*}
(1,2,0)
=
(2,1,0) + t (-2,-1,3)
\end{align*}
for $t$. (Keep in mind that this vector equation can be viewed as a system of three equations (one for each component) in one unknown (the parameter $t$).) You will find that there is no solution.} % End footnote.
Thus the plane we are asked to find indeed exists.

We already have an explicit point in this plane (in fact, two: $P_{0}$ and $P'_{0}$), so all that remains is to find a normal vector to the plane. We also have an explicit vector in this plane, namely, the direction vector $\fontVector{v} = (-2,-1,3)$ of the line $L$. If we can find a second vector in the plane that is not parallel to $\fontVector{v}$, then we can take a cross product and be done. Think geometrically about our situation: We have a line $L$ in the desired plane, and a point $P_{0}$ in the desired plane, with $P_{0}$ not in $L$. If we take any point $P'_{0}$ in $L$, and form the vector $\fontVectorBetweenPoints{P'_{0} P_{0}}$, then this vector cannot be parallel to the direction vector $\fontVector{v}$ of the line $L$ (if it were, then $P_{0}$ would be in $L$). This gives us our desired second vector:
\begin{align*}
\fontVectorBetweenPoints{P'_{0} P_{0}}
=
(-3,0,0).
\end{align*}
Hence a normal vector $\fontVector{n}$ to the plane $H$ containing $P_{0}$ and $L$ is
\begin{align*}
\fontVector{n}
=
\fontVectorBetweenPoints{P'_{0} P_{0}} \times \fontVector{v}
=
\det
\begin{pmatrix}
\fontVector{i}	&	\fontVector{j}	&	\fontVector{k}	\\
-3			&	0			&	0			\\
-2			&	-1			&	3
\end{pmatrix}
=
(0,9,3).
\end{align*}
Any nonzero scalar multiple of a normal vector is also a normal vector (do you see why, geometrically?), so we can also take the vector
\begin{align*}
\fontVector{n}'
=
\frac{1}{3} \fontVector{n}
=
(0,3,1)
\end{align*}
as a normal vector to the plane $H$.% Begin footnote.
\footnote{Using $\fontVector{n}'$ just makes the subsequent computation a bit simpler. Using $\fontVector{n}$ yields an equivalent equation for the plane, which we can reduce to the equation obtained from using $\fontVector{n}'$ by dividing by $3$. Do you see why?} % End footnote.
Substituting $\fontVector{n}'$ and $P_{0}$ into \eqref{eq: General Equation Plane R3}, we conclude that an equation for the plane $L$ containing $P_{0}$ and $L$ is
\begin{align*}
0
=
(0,3,1) \cdot (x - 1,y - 2,z - 0)
=
3 y - 6 + z,
\end{align*}
or equivalently,
\begin{align*}
3 y + z
=
6.
\end{align*}
}% End solution.





%
%
%	Exercise 2
%
%

\section{Exercise \ref{sec: Math212 2016 Fall HalfExam01Q02}}
\label{sec: Math212 2016 Fall HalfExam01Q02}

(10 pt) Find the point $P^{*}$ in the line
\begin{align*}
L: \fontVector{r}(t)
=
\left(2 t + 4,-2 t,t - 4\right)
\end{align*}
that is closest to the point $Q = (12,6,1)$. \fontHint{Think geometrically! Use vectors.}



\spaceSolution{4in}{% Begin solution.
\subsection{Solution}

As usual, we start by thinking geometrically, to acquire some intuition for how to approach this problem. The line $L$ and the point $Q$ are in $\reals^{3}$, but we can restrict our attention to a plane containing $L$ and $Q$.% Begin footnote.
\footnote{If the point $Q$ is not in the line $L$ --- as is the case here --- then this plane is unique. If the point $Q$ is in the line $L$, then there are infinitely many planes containing $Q$ and $L$, but for the purposes of our problem, we would be done: $Q$ itself would be point in $L$ closest to $Q$.

The point $Q$ lies in $L$ if and only if there exists a parameter value $t$ such that $\fontVector{r}(t) = Q$. This vector equality is equivalent to a system of three equations (one for each component) in one unknown (the parameter $t$). It is straightforward to show that this system of equations has no solution (do it).} % End footnote.
That is, we can think geometrically in two dimensions. So imagine $L$ and $Q$ in $\reals^{2}$. We want the point $P^{*}$ in $L$ that is closest to $Q$. Visualizing this situation, you will convince yourself that this point $P^{*}$ is unique and is the intersection of $L$ with the line through $Q$ that is perpendicular to $L$\fontNeedsEdit{ (draw it!)}.

For each point $P$ in the line $L$, we can associate the vector $\fontVectorBetweenPoints{P Q}$ from $P$ to $Q$. More precisely, to say that $\fontVector{r}(t)$ describes the line $L$ means that any point $P$ in $L$ has coordinates
\begin{align*}
\fontVector{r}(t)
=
\left(2 t + 4,-2 t,t - 4\right)
\end{align*}
for some $t \in \reals$. Thus for each point $P$ in $L$,
\begin{align}
\fontVectorBetweenPoints{P Q}
=
\left(8 - 2 t,6 + 2 t,5 - t\right).%
\label{eq: Vector From Point In Line To Q}
\end{align}
In particular, $\fontVectorBetweenPoints{P^{*} Q}$ has this form. 

The geometric intuition that we developed above now provides us with at least two ways to find the coordinates of $P^{*}$.

\paragraph{Using inner product}

In our geometric picture, we see that the vector $\fontVectorBetweenPoints{P^{*} Q}$ corresponding to the special point $P^{*}$ is orthogonal to the direction vector $\fontVector{v}$ of the line $L$. We can read off the components of the direction vector from the corresponding coefficients of the parameter $t$ in the equation defining $\fontVector{r}(t)$:
\begin{align*}
\fontVector{v}
=
(2,-2,1).
\end{align*}
Because $\fontVectorBetweenPoints{P^{*} Q}$ is orthogonal to $\fontVector{v}$, their inner product equals zero:
\begin{align*}
0
=
\fontVectorBetweenPoints{P^{*} Q} \cdot \fontVector{v}
=
\left(8 - 2 t,6 + 2 t,5 - t\right) \cdot (2,-2,1)
=
(16 - 4 t) + (-12 - 4 t) + (5 - t)
=
9 - 9 t,
\end{align*}
which yields the solution
\begin{align*}
t
=
1.
\end{align*}
Thus
\begin{align*}
P^{*}
=
\fontVector{r}(1)
=
(6,-2,-3).
\end{align*}

\paragraph{Using norm}

In our geometric picture, we see that the vector $\fontVectorBetweenPoints{P^{*} Q}$ corresponding to the special point $P^{*}$ is the shortest vector of the form \eqref{eq: Vector From Point In Line To Q}. That is, the point $P^{*}$ corresponds to the parameter value $t$ that minimizes
\begin{align*}
\norm{\fontVectorBetweenPoints{P Q}}
&=
\sqrt{(8 - 2 t)^{2} + (6 + 2 t)^{2} + (5 - t)^{2}}
\\
&=
\sqrt{(64 - 32 t + 4 t^{2}) + (36 + 24 t + 4 t^{2}) + (25 - 10 t + t^{2})}
\\
&=
\sqrt{9 t^{2} - 18 t + 125}
=
\sqrt{9 (t^{2} - 2 t + 16)}
\\
&=
3 \sqrt{(t - 1)^{2} + 15}.
\end{align*}
This expression is minimized when the expression in the square root is minimized, which in turn occurs when $t = 1$.% Begin footnote.
\footnote{The expression in the square root describes an upward-opening parabola with vertex $(1,15)$.} % End footnote.
Thus
\begin{align*}
P^{*}
=
\fontVector{r}(1)
=
(6,-2,-3).
\end{align*}
}% End solution.







%
%
%	Exercise 3
%
%

\section{Exercise \ref{sec: Math212 2016 Fall HalfExam01Q03}}
\label{sec: Math212 2016 Fall HalfExam01Q03}

(20 pt) At time $t = 0$, you espy a flying cockroach whizzing around your bedroom. Always quick on your feet, you apply your mad skillz of vector calculus to determine that the cockroach's position vector is given by
\begin{align*}
\fontVector{r}(t)
=
\left(\frac{4 \sqrt{2}}{5} t^{\frac{5}{2}} - 1,t^{2},\frac{2}{3} t^{3} + 1\right),
\end{align*}
where the component functions are measured in meters, and $t$ is measured in seconds. At time $t = 3$, you squash the sucker.
\begin{enumerate}[label=(\alph*)]
\item\label{itm: Math212 2016 Fall HalfExam01Q03a} (5 pt) Find the cockroach's velocity function $\fontVector{v}(t)$. What are the units of the component functions?
\end{enumerate}



\spaceSolution{.75in}{% Begin solution.
\subsection{Solution to part \ref{itm: Math212 2016 Fall HalfExam01Q03a}}

The velocity vector is the derivative of the position vector with respect to time. Remembering that derivatives of vector-valued functions are computed componentwise, we have
\begin{align*}
\fontVector{v}(t)
=
\fontVector{r}'(t)
=
\left(2 \sqrt{2} t^{\frac{3}{2}},2 t,2 t^{2}\right).
\end{align*}
The component functions are measured in meters/second.
}% End solution.



\begin{enumerate}[resume,label=(\alph*)]
\item\label{itm: Math212 2016 Fall HalfExam01Q03b} (10 pt) Determine how far the cockroach travels from the time you spot it to the time you squash it.
\end{enumerate}



\spaceSolution{3.75in}{% Begin solution.
\subsection{Solution to part \ref{itm: Math212 2016 Fall HalfExam01Q03b}}

The distance $s$ traveled by the cockroach is the arc length of $\fontVector{r}(t)$ from $t = 0$ to $t = 3$:
\begin{align}
s
=
\int_{0}^{3} \norm{\fontVector{r}'(t)} \intd t.%
\label{eq: Arc Length General Form}
\end{align}
The derivative $\fontVector{r}'(t)$ is precisely the velocity function that we found in part \ref{itm: Math212 2016 Fall HalfExam01Q03a}. Thus
\begin{align*}
\norm{\fontVector{r}'(t)}
=
\sqrt{(8 t^{3}) + (4 t^{2}) + (4 t^{4})}
=
\sqrt{4 t^{2} \left(t^{2} + 2 t + 1\right)}
=
\sqrt{(2 t)^{2} (t + 1)^{2}}
=
2 t (t + 1),
\end{align*}
where in the final equality we have used the fact that $\fontVector{r}(t)$ (and hence $\fontVector{r}'(t)$) is defined only for $t \geq 0$ (do you see why?). Substituting this expression into \eqref{eq: Arc Length General Form}, we find
\begin{align*}
s
=
2 \int_{0}^{3} \left(t^{2} + t\right) \intd t
=
2 \left[\frac{1}{3} t^{3} + \frac{1}{2} t^{2}\right]_{t = 0}^{t = 3}
=
2 \left[\left(9 + \frac{9}{2}\right) - 0\right]
=
27.
\end{align*}
That is, the cockroach travels 27 meters from the time you see it to the time you squash it.
}% End solution.



\begin{enumerate}[resume,label=(\alph*)]
\item\label{itm: Math212 2016 Fall HalfExam01Q03c} (5 pt) You suspect that the cockroach was flying to its nest. In one (!) sentence, explain how you can (use something about vectors to) locate the nest.
%\item\label{itm: Math212 2016 Fall HalfExam01Q04a} (Bonus 5 pt) Find the minimum speed of the cockroach over the time interval $(-\infty,3]$, and the time(s) when this minimum speed is achieved. \fontHint{Be careful. Physically, what inequality does speed satisfy?}
\end{enumerate}



\spaceSolution{1in}{% Begin solution.
\subsection{Solution to part \ref{itm: Math212 2016 Fall HalfExam01Q03c}}

If our suspicion is correct, then following the curve described by $\fontVector{r}(t)$ for $t > 3$ will lead us to the nest.
}% End solution.