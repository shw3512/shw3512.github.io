\section{Geometry}



\subsection{Exercise \ref{sec: Geometry Q1}}
\label{sec: Geometry Q1}

(2 pt) Compute the distance between the points $(-2,-1)$ and $(4,7)$ in $\reals^{2}$.

\spaceSolution{1.25in}{% Begin solution.
The distance is
\begin{align*}
d
=
\sqrt{(4 - (-2))^{2} + (7 - (-1))^{2}}
=
\sqrt{6^{2} + 8^{2}}
=
10.
\end{align*}
}% End solution.



\subsection{Exercise \ref{sec: Geometry Q2}}
\label{sec: Geometry Q2}

(2 pt) Write the area of a circular sector with radius $r$ and angle $\theta$, where $0 \leq \theta \leq 2 \pi$.

\spaceSolution{1.25in}{% Begin solution.
Note that a sector of a circle is a fraction of the circle, more precisely, the fraction $\frac{\theta}{2 \pi}$. Thus the area of the sector is given by
\begin{align*}
\frac{\theta}{2 \pi} \pi r^{2}
=
\frac{1}{2} \theta r^{2}.
\end{align*}
}% End solution.



\subsection{Exercise \ref{sec: Geometry Q3}}
\label{sec: Geometry Q3}

(2 pt) Write the area of a parallelogram with adjacent side lengths $u,v$ and enclosed angle $\theta$.

\spaceSolution{1.25in}{% Begin solution.
Imagine dropping a perpendicular from one side, say a side of length $u$, to the opposite parallel side (also of length $a$); cutting off the resulting right triangle (with hypotenuse $v$); and pasting this triangle to the other side of the parallelogram, lining up the sides of length $v$. This yields a rectangle with side lengths $u$ and $v \sin \theta$. Thus the area of the parallelogram is
\begin{align*}
u v \sin \theta.
\end{align*}
}% End solution.



\subsection{Exercise \ref{sec: Geometry Q4}}
\label{sec: Geometry Q4}

(2 pt) Convert the point $(4,\frac{7 \pi}{6})$, given in polar coordinates $(r,\theta)$, to rectangular coordinates $(x,y)$.

\spaceSolution{1.25in}{% Begin solution.
The point $(4,\frac{7 \pi}{6})$ in polar coordinates lies in the third quadrant (negative $x$, negative $y$) of the euclidean plane. The corresponding rectangular coordinates are $(x,y)$, where
\begin{align*}
x
&=
r \cos \theta
=
4 \cos \frac{7 \pi}{6}
=
-2 \sqrt{3},
&
y
&=
r \sin \theta
=
4 \sin \frac{7 \pi}{6}
=
-2.
\end{align*}
}% End solution.





\section{Single-Variable Calculus}



\subsection{Exercise \ref{sec: Single Variable Calculus Q1}}
\label{sec: Single Variable Calculus Q1}

(2 pt) Evaluate the integral $\displaystyle\int \cos^{2} \theta \spaceIntd \intd \theta$.

\spaceSolution{3in}{% Begin solution.
Recall the trigonometric identity
\begin{align*}
\cos 2 \theta
&=
2 \cos^{2} \theta - 1
&
&\Leftrightarrow
&
\cos^{2} \theta
&=
\frac{1}{2} (1 + \cos 2 \theta).
\end{align*}
Substituting this into the given integral we find
\begin{align*}
\int \cos^{2} \theta \spaceIntd \intd \theta
&=
\frac{1}{2} \int (1 + \cos 2 \theta) \spaceIntd \intd \theta
\\
&=
\frac{1}{2} \int \intd \theta + \frac{1}{4} \int \cos 2 \theta \spaceIntd 2 \intd \theta
\\
&=
\frac{1}{2} \theta + \frac{1}{4} \sin 2 \theta + C,
\end{align*}
where $C$ is a constant of integration.
}% End solution.



\subsection{Exercise \ref{sec: Single Variable Calculus Q2}}
\label{sec: Single Variable Calculus Q2}

% S6e Example 5.5.2 (p 402)
(2 pt) Evaluate the integral $\displaystyle\int \sqrt{2 x + 1} \spaceIntd \intd x$.

\spaceSolution{3in}{% Begin solution.
We cannot evaluate the given integral directly. Consider the change of variables (a.k.a. $u$-substitution)
\begin{align*}
u
&\defeq
2 x + 1
&
&\Rightarrow
&
\intd u
&=
2 \intd x
&
&\Leftrightarrow
&
\intd x
&=
\frac{1}{2} \intd u.
\end{align*}
When we make this substitution, the given integral becomes
\begin{align*}
\int \sqrt{2 x + 1} \spaceIntd \intd x
=
\int \sqrt{u} \left(\frac{1}{2} \intd u\right)
=
\frac{1}{2} \int \sqrt{u} \spaceIntd \intd u
=
\frac{1}{3} u^{\frac{3}{2}} + C
=
\frac{1}{3} (2 x + 1)^{\frac{3}{2}} + C,
\end{align*}
where $C$ is a constant of integration.
}% End solution.



\subsection{Exercise \ref{sec: Single Variable Calculus Q3}}
\label{sec: Single Variable Calculus Q3}

% S6e Formula 8.1.2 (p 526)

(2 pt) Let $f$ be a real-valued function defined on the closed interval $[a,b]$, and let $f'$ be continuous on $[a,b]$. State the length of the curve $f(x)$ from $x = a$ to $x = b$.

\spaceSolution{5in}{% Begin solution.
The length of the curve is
\begin{align}
\int_{a}^{b} \sqrt{1 + (f'(x))^{2}} \spaceIntd \intd x.%
\label{eq: Length Of Curve Integral}
\end{align}

\hspace{.25in}% Fix default indentation so that we can remove this hspace!!!
One can arrive at this result as follows. Choose $n + 1$ (distinct) points $(x_{i},f(x_{i}))$ on the graph of the function $f$ on the closed interval $[a,b]$, with $(x_{0},f(x_{0})) = (a,f(a))$ and $(x_{n},f(x_{n})) = (b,f(b))$. Then ``connect the dots'', i.e. draw a line segment from the point $(x_{i - 1},f(x_{i - 1}))$ to the point $(x_{i},f(x_{i}))$, for each $i \in \{1,\ldots,n\}$. This yields a polygonal approximation to the graph of $f$. Let
\begin{align*}
\Delta x_{i}
&\defeq
x_{i} - x_{i - 1},
&
\Delta y_{i}
&\defeq
f(x_{i}) - f(x_{i - 1});
\end{align*}
these are the changes in $x$ and $y$, respectively, over the line segment connecting $(x_{i - 1},f(x_{i - 1}))$ to $(x_{i},f(x_{i}))$. Thus by Pythagorean's theorem, the length of this line segment, denote it $\Delta s_{i}$, is
\begin{align*}
\Delta s_{i}
=
\sqrt{(\Delta x_{i})^{2} + (\Delta y_{i})^{2}}
=
\sqrt{1 + \left(\frac{\Delta y_{i}}{\Delta x_{i}}\right)^{2}} \Delta x_{i},
\end{align*}
and the length of the polygonal approximation to the length $s$ of the curve $f(x)$ is
\begin{align}
s
\approx
\sum_{i = 1}^{n} \Delta s_{i}
=
\sum_{i = 1}^{n} \sqrt{1 + \left(\frac{\Delta y_{i}}{\Delta x_{i}}\right)^{2}} \Delta x_{i}.%
\label{eq: Length Of Curve Finite Sum Approximation}
\end{align}
If we choose more and more points along the curve (i.e. as the number of points $n$ approaches infinity),% Begin footnote.
\footnote{We need to choose these points so that $\lim_{n \rightarrow \infty} \Delta x_{i} = 0$. This is ensured, for example, if we take $\Delta x_{i}$ to be the same for all $i$, namely $\Delta x_{i} = \frac{b - a}{n}$.} % End footnote.
then this polygonal approximation becomes better and better. In fact, as $n \rightarrow \infty$, $\frac{\Delta y_{i}}{\Delta x_{i}} \rightarrow f'(x_{i})$, and the sum \eqref{eq: Length Of Curve Finite Sum Approximation} becomes the integral \eqref{eq: Length Of Curve Integral}.\fontNeedsEdit{ (draw the picture)}}% End solution.



\subsection{Exercise \ref{sec: Single Variable Calculus Q4}}
\label{sec: Single Variable Calculus Q4}

(2 pt) Let $f$ be a real-valued function defined on a closed interval $[a,b]$ with $a < b$. Draw a picture depicting a Riemann sum corresponding to the definite integral
\begin{align*}
\int_{a}^{b} f(x) \intd x.
\end{align*}

\spaceSolution{.5in}{% Begin solution.
See Sections 5.1 and 5.2 of Stewart.\fontNeedsEdit{ (include sketch; make $f$ noncontinuous, changing sign)}
}% End solution.





\section{Vector Calculus}



\subsection{Exercise \ref{sec: Vector Calculus Q1}}
\label{sec: Vector Calculus Q1}

(2 pt) Let $\fontVector{u} \defeq (2,0,1)$ and $\fontVector{v} \defeq (0,-1,1)$ be vectors in $\reals^{3}$.
\begin{enumerate}[label=(\alph*)]
\item\label{itm: Vector Calculus Q1a} (1 pt) Compute the inner product (a.k.a. dot product) $\fontVector{u} \cdot \fontVector{v}$.
\end{enumerate}

\spaceSolution{1in}{% Begin solution.
We compute
\begin{align*}
\fontVector{u} \cdot \fontVector{v}
=
(2,0,1) \cdot (0,-1,1)
=
2 \cdot 0 + 0 \cdot (-1) + 1 \cdot 1
=
1.
\end{align*}
}% End solution.

\begin{enumerate}[resume,label=(\alph*)]
\item\label{itm: Vector Calculus Q1b} (1 pt) Compute the cross product $\fontVector{u} \times \fontVector{v}$.
\end{enumerate}

\spaceSolution{2in}{% Begin solution.
We compute
\begin{align*}
\fontVector{u} \times \fontVector{v}
&=
\det
\begin{pmatrix}
\fontVector{i}	&	\fontVector{j}	&	\fontVector{k}	\\
2			&	0			&	1			\\
0			&	-1			&	1
\end{pmatrix}
\\
&=
(-1)^{1 + 1} \det
\begin{pmatrix}
0	&	1	\\
-1	&	1
\end{pmatrix}
\fontVector{i} +
(-1)^{1 + 2} \det
\begin{pmatrix}
2	&	1	\\
0	&	1
\end{pmatrix}
\fontVector{j} +
(-1)^{1 + 3} \det
\begin{pmatrix}
2	&	0	\\
0	&	-1
\end{pmatrix}
\fontVector{k}
\\
&=
\fontVector{i} - 2 \fontVector{j} - 2 \fontVector{k}
=
(1,-2,-2).
\end{align*}
}% End solution.



\subsection{Exercise \ref{sec: Vector Calculus Q2}}
\label{sec: Vector Calculus Q2}

% S6e Example 14.7.7 (p 929)
(2 pt) Find the absolute maximum and minimum values of the function
\begin{align*}
f(x,y)
\defeq
x^{2} - 2 x y + 2 y
\end{align*}
on the closed rectangle
\begin{align*}
D
\defeq
\left\{(x,y) \st 0 \leq x \leq 3,0 \leq y \leq 2\right\}.
\end{align*}

\vspace{3in}



\subsection{Exercise \ref{sec: Vector Calculus Q3}}
\label{sec: Vector Calculus Q3}

%S6e Example 15.9.3 (p 1018)
(2 pt) Evaluate the integral
\begin{align*}
\iint_{R} e^{\frac{x + y}{x - y}} \intd A,
\end{align*}
where $R$ is the trapezoidal region in $\reals^{2}$ with vertices $(1,0),(2,0),(0,-2),(0,-1)$.

\newpage



\subsection{Exercise \ref{sec: Vector Calculus Q4}}
\label{sec: Vector Calculus Q4}

% S6e Example 16.9.2 (p 1101)
(2 pt) Evaluate
\begin{align*}
\iint_{S} \fontVector{F} \cdot \intd \fontVector{S},
\end{align*}
where
\begin{align*}
\fontVector{F}(x,y,z)
\defeq
\left(x y,y^{2} + e^{x z^{2}},\sin(x y)\right),
\end{align*}
and $S$ is the surface of the region $E$ bounded by the parabolic cylinder $z = 1 - x^{2}$ and the planes $z = 0$, $y = 0$, and $y + z = 2$.