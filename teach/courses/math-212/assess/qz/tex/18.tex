\section{Exercise}

(2 pt) Let $f: \reals^{2} \rightarrow \reals$ be given by
\begin{align*}
f(x,y)
=
3 x^{2} - 6 x y^{2} + 2 y,
\end{align*}
and let $R \subseteq \reals^{2}$ be the rectangle
\begin{align*}
R
=
\left\{(x,y) \in \reals^{2} \st 0 \leq x \leq 2,0 \leq y \leq 1\right\}.
\end{align*}
We wish to evaluate the double integral $\iint_{R} f(x,y) \, \intd A$.
\begin{enumerate}[label=(\alph*)]
\item (0.5 pt) Justify why you can write the double integral as an iterated integral. \fontHint{Two words, rhymes with ``Houdini's serum''. Heart points for noting a sufficient condition on $f$.}
\end{enumerate}

\spaceSolution{1in}{% Begin solution.
Fubini's theorem. More precisely, the integrand $f(x,y)$ is a polynomial, hence continuous everywhere on its domain, and in particular on the region $R$ of integration.
}% End solution.



\begin{enumerate}[resume,label=(\alph*)]
\item (1.5 pt) Evaluate the integral $\iint_{R} f(x,y) \, \intd A$.
\end{enumerate}

\spaceSolution{4in}{% Begin solution.
By Fubini's theorem, we may evaluate the double integral as an iterated integral in either order. Using the order $\intd x \, \intd y$, we compute
\begin{align*}
\iint_{R} f(x,y) \, \intd A
&=
\int_{y = 0}^{y = 1} \int_{x = 0}^{x = 2} \left(3 x^{2} - 6 x y^{2} + 2 y\right) \intd x \, \intd y
\\
&=
\int_{y = 0}^{y = 1} \left[x^{3} - 3 x^{2} y^{2} + 2 x y\right]_{x = 0}^{x = 2} \, \intd y
\\
&=
\int_{y = 0}^{y = 1} \left(8 - 12 y^{2} + 4 y\right) \, \intd y
\\
&=
\left[8 y - 4 y^{3} + 2 y^{2}\right]_{y = 0}^{y = 1}
\\
&=
6.
\end{align*}
If instead we integrate using the order $\intd y \, \intd x$, then we obtain
\begin{align*}
\iint_{R} f(x,y) \, \intd A
&=
\int_{x = 0}^{x = 2} \int_{y = 0}^{y = 1} \left(3 x^{2} - 6 x y^{2} + 2 y\right) \, \intd y \, \intd x
\\
&=
\int_{x = 0}^{x = 2} \left[3 x^{2} y - 2 x y^{3} + y^{2}\right]_{y = 0}^{y = 1} \, \intd x
\\
&=
\int_{x = 0}^{x = 2} \left(3 x^{2} - 2 x + 1\right) \, \intd x
\\
&=
\left[x^{3} - x^{2} + x\right]_{x = 0}^{x = 2}
\\
&=
6.
\end{align*}
Note that both orders of integration yield the same result, as required by Fubini's theorem.
}% End solution.