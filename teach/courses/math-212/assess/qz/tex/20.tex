\section{Exercise}

(2 pt) Let $D \subseteq \reals^{2}$ be the region in the upper half plane (i.e. $y \geq 0$) bounded by the circle $C: x^{2} + y^{2} = 4$ and the lines $y = x$ and $y = -x$.
\begin{enumerate}[label=(\alph*)]
\item (0.5 pt) Sketch and shade the region $D$. \fontHint{The point $(0,1) \in D$.}
\end{enumerate}

\spaceSolution{1.5in}{% Begin solution.
The region $D$ is shown below.
\begin{center}
\includegraphics[scale=.4]{\filePathGraphics Quiz20_RegionD}
\end{center}
}% End solution.



\begin{enumerate}[resume,label=(\alph*)]
\item (1.5 pt) Let $f: \reals^{2} \rightarrow \reals$ be the function
\begin{align*}
f(x,y)
=
2 x y.
\end{align*}
Set up (but do NOT evaluate) an iterated (!) integral for $\iint_{D} f(x,y) \, \intd A$ \textbf{using polar coordinates}. \fontHint{Describe the region $D$ algebraically using polar coordinates. When writing the iterated integral, remember to translate $(x,y)$ to $(r,\theta)$, and mind your integration factor.}
\end{enumerate}

\spaceSolution{3in}{% Begin solution.
Because $f(x,y)$ is continuous on the region $D$ of integration, Fubini's theorem allows us to write the double integral as an iterated integral. In polar coordinates, the region $D$ can be described as
\begin{align*}
D
=
\left\{(r,\theta) \st 0 \leq r \leq 2,\frac{\pi}{4} \leq \theta \leq \frac{3 \pi}{4}\right\},
\end{align*}
the function $f(x,y)$ writes as
\begin{align*}
f(x,y)
=
2 x y
=
2 (r \cos \theta) (r \sin \theta)
=
r^{2} 2 \sin \theta \cos \theta
=
r^{2} \sin(2 \theta),
\end{align*}
where in the final equality we have used the trigonometric identity
\begin{align*}
\sin(2 \theta)
=
2 \sin \theta \cos \theta,
\end{align*}
and
\begin{align*}
\intd A
=
r \, \intd r \, \intd \theta
\end{align*}
(note the integration factor of $r$). Thus
\begin{align*}
\iint_{D} f(x,y) \, \intd A
&=
\int_{\theta = \frac{\pi}{4}}^{\theta = \frac{3 \pi}{4}} \int_{r = 0}^{r = 2} r^{2} \sin(2 \theta) \, r \, \intd r \, \intd \theta
\\
&=
\int_{\theta = \frac{\pi}{4}}^{\theta = \frac{3 \pi}{4}} \sin(2 \theta) \, \intd \theta \int_{r = 0}^{r = 2} r^{3} \, \intd r.
\end{align*}
Note that this iterated integral would be easy to compute, if we so desired.
}% End solution.