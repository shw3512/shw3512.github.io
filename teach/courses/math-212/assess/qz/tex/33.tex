%
%
%	2016-11-21 (M)
%
%
\section{Exercise}

(2 pt) For each of the following vector fields $\fontVector{F}: \reals^{3} \rightarrow \reals^{3}$, write ``Conservative'' if $\fontVector{F}$ is conservative, and ``Not conservative'' otherwise. Justify your answer. \fontHint{
\begin{center}
\begin{tabular}{l}
Conservative? Liberal? Head in a whirl ---\\
For vector-field politics, compute the \_\_\_\_\_\_\_\_.
\end{tabular}
\end{center}}
\begin{enumerate}[label=(\alph*)]
\item (1 pt) $\fontVector{F}(x,y,z) = (e^{z},1,x e^{z})$
\end{enumerate}

\spaceSolution{3in}{
Conservative. We compute
\begin{align*}
\curl \fontVector{F}
&=
\det
\begin{pmatrix}
\fontVector{i}			&	\fontVector{j}			&	\fontVector{k}			\\
\frac{\partial}{\partial x}	&	\frac{\partial}{\partial y}	&	\frac{\partial}{\partial z}	\\
e^{z			}		&	1					&	x e^{z}
\end{pmatrix}
\\
&=
\fontVector{i} \left(0 - 0\right) - \fontVector{j} \left(e^{z} - e^{z}\right) + \fontVector{k} \left(0 - 0\right)
\\
&=
\fontVector{0},
\end{align*}
the zero vector field. Because the domain of $\fontVector{F}$, i.e. $\reals^{3}$, is simply connected and $\curl \fontVector{F} = \fontVector{0}$, it follows that $\fontVector{F}$ is conservative. (For practice, show that a potential function $f: \reals^{3} \rightarrow \reals$ for $\fontVector{F}$ always has the form $f(x,y,z) = x e^{z} + y + C$ for $C \in \reals$.)}



\begin{enumerate}[resume,label=(\alph*)]
\item (1 pt) $\fontVector{F}(x,y,z) = (y e^{-x},e^{-x},2 z)$
\end{enumerate}

\spaceSolution{3in}{
Not conservative. We compute
\begin{align*}
\curl \fontVector{F}
&=
\det
\begin{pmatrix}
\fontVector{i}			&	\fontVector{j}			&	\fontVector{k}			\\
\frac{\partial}{\partial x}	&	\frac{\partial}{\partial y}	&	\frac{\partial}{\partial z}	\\
y e^{-x}				&	e^{-x}				&	2 z
\end{pmatrix}
\\
&=
\fontVector{i} \left(0 - 0\right) - \fontVector{j} \left(0 - 0\right) + \fontVector{k} \left(-e^{-x} - e^{-x}\right)
\\
&=
\left(0,0,-2 e^{-x}\right).
\end{align*}
Because $\curl \fontVector{F} \neq \fontVector{0}$,% Begin footnote.
\footnote{Note that $-2 e^{-x} \neq 0$ on $\reals^{3}$; e.g., at $(x,y,z) = (0,0,0)$, $-2 e^{-x} = -2 \neq 0$.} % End footnote.
it follows that $\fontVector{F}$ is not conservative (recall that $\curl(\gradient f) = \fontVector{0}$).}