%
%
%	2016-11-30 (W)
%
%
\section{Exercise}

(5 pt) For each of the following integrals, name a theorem that allows you to write an equivalent expression (e.g., integral), write it, and evaluate it. \emph{Guiding light:} Integrating a derivative over a region is related to evaluating the original function on the boundary of that region.
\begin{enumerate}[label=(\alph*)]% S6eQ16.08.10
\item (2.5 pt) Let $\fontVector{F}: \reals^{3} \rightarrow \reals^{3}$ be given by
\begin{align*}
\fontVector{F}(x,y,z)
=
\left(x y,2 z,3 y\right),
\end{align*}
and let $C \subseteq \reals^{3}$ be the curve of intersection of the plane $x + z = 5$ and the cylinder $x^{2} + y^{2} = 9$, oriented counterclockwise when viewed from above. Show that $\int_{C} \fontVector{F} \cdot \intd \fontVector{r} = 9 \pi$.
\end{enumerate}

\spaceSolution{2.75in}{% Begin solution.
Stokes's theorem: $C$ is a (piecewise-) smooth simple closed curve, and the component functions of $\fontVector{F}$ are continuously differentiable, so taking $S$ to be any piecewise-smooth oriented surface with boundary $C$ permits us to apply Stokes's theorem, rewriting the given line integral as a surface integral. A convenient choice of surface $S$ is the region of the plane $x + z = 5$ bounded by $C$, equipped with upward-pointing unit normal vectors (to agree with the given orientation on $C$). Let $D \subseteq \reals^{2}$ be the disc in the $x y$-plane bounded by the given cylinder $x^{2} + y^{2} = 9$:
\begin{align*}
D
=
\left\{(x,y) \in \reals^{2} \st x^{2} + y^{2} \leq 9\right\}
=
\left\{(r,\theta) \st r \in [0,3],\theta \in [0,2 \pi]\right\}.
\end{align*}
Then a parametrization of this surface $S$ is given by
\begin{align*}
\fontVector{r}:
D
&\rightarrow
\reals^{3}
\\
(x,y)
&\mapsto
\left(x,y,5 - x\right).
\end{align*}
Note that
\begin{align*}
\fontVector{r}_{x}
&=
\left(1,0,-1\right),
&
\fontVector{r}_{y}
&=
\left(0,1,0\right),
\end{align*}
so the induced normal vector at each point of $S$ is
\begin{align*}
\fontVector{r}_{x} \times \fontVector{r}_{y}
=
\left(1,0,1\right),
\end{align*}
upward-pointing, as required. Stokes's theorem gives
\begin{align*}
\int_{C} \fontVector{F} \cdot \intd \fontVector{r}
&=
\iint_{S} (\curl \fontVector{F}) \cdot \intd \fontVector{S}
=
\iint_{D} (\curl \fontVector{F}) \cdot (\fontVector{r}_{x} \times \fontVector{r}_{y}) \, \intd A
\\
&=
\iint_{D} \left(1,0,-x\right) \cdot (1,0,1) \, \intd A
=
\iint_{D} (1 - x) \, \intd A
\\
&=
\int_{\theta = 0}^{\theta = 2 \pi} \int_{r = 0}^{r = 3} (1 - r \cos \theta) r \, \intd r \, \intd \theta
\\
&=
\int_{\theta = 0}^{\theta = 2 \pi} \left[\frac{1}{2} r^{2} - \frac{1}{3} r^{3} \cos \theta\right]_{r = 0}^{r = 3} \intd \theta
\\
&=
\int_{\theta = 0}^{\theta = 2 \pi} \left(\frac{9}{2} - 9 \cos \theta\right) \intd \theta
\\
&=
\left[\frac{9}{2} \theta - 9 \sin \theta\right]_{\theta = 0}^{\theta = 2 \pi}
\\
&=
9 \pi.
\end{align*}}% End solution.



\begin{enumerate}[resume,label=(\alph*)]% S6eQ16.09.12
\item (2.5 pt) Let $\fontVector{F}: \reals^{3} \rightarrow \reals^{3}$ be given by
\begin{align*}
\fontVector{F}(x,y,z)
=
\left(x^{4},-x^{3} z^{2},4 x y^{2} z\right),
\end{align*}
and let $S \subseteq \reals^{3}$ be the surface of the solid bounded by the cylinder $x^{2} + y^{2} = 1$ and the planes $z = x + 2$ and $z = 0$, where $S$ is equipped with its outward-pointing unit normal vectors. Show that $\iint_{S} \fontVector{F} \cdot \intd \fontVector{S} = \frac{2 \pi}{3}$.
\end{enumerate}

\spaceSolution{2.75in}{% Begin solution.
Gauss's (divergence) theorem: Let $E \subseteq \reals^{3}$ denote the solid bounded by $S$. Note that $E$ is a simple region, $S$ has positive (i.e. outward-pointing) orientation (by hypothesis), and the component functions of $\fontVector{F}$ are continuously differentiable. Thus Gauss's theorem applies, and we can rewrite the given surface integral as a triple integral. Using cylindrical coordinates to compute the resulting triple integral, we find
\begin{align*}
\iint_{S} \fontVector{F} \cdot \intd \fontVector{S}
&=
\iiint_{E} \divergence \fontVector{F} \, \intd V
=
\iiint_{E} \left(4 x^{3} + 0 + 4 x y^{2}\right) \intd V
=
4 \iiint_{E} x \left(x^{2} + y^{2}\right) \intd V
\\
&=
4 \int_{\theta = 0}^{\theta = 2 \pi} \int_{r = 0}^{r = 1} \int_{z = 0}^{z = 2 + r \cos \theta} (r \cos \theta) r^{2} \, r \, \intd z \, \intd r \, \intd \theta
\\
&=
4 \int_{\theta = 0}^{\theta = 2 \pi} \int_{r = 0}^{r = 1} \left(r^{4} \cos \theta\right) \int_{z = 0}^{z = 2 + r \cos \theta} \intd z \, \intd r \, \intd \theta
\\
&=
4 \int_{\theta = 0}^{\theta = 2 \pi} \int_{r = 0}^{r = 1} \left(2 r^{4} \cos \theta + r^{5} \cos^{2} \theta\right) \intd r \, \intd \theta
\\
&=
4 \int_{\theta = 0}^{\theta = 2 \pi} \left[\frac{2}{5} r^{5} \cos \theta + \frac{1}{6} r^{6} \cos^{2} \theta\right]_{r = 0}^{r = 1} \intd \theta
\\
&=
4 \int_{\theta = 0}^{\theta = 2 \pi} \left(\frac{2}{5} \cos \theta + \frac{1}{12} \left(1 + \cos(2 \theta)\right)\right) \intd \theta
\\
&=
4 \left[\frac{2}{5} \sin \theta + \frac{1}{12} \theta + \frac{1}{24} \sin(2 \theta)\right]_{\theta = 0}^{\theta = 2 \pi}
\\
&=
\frac{2 \pi}{3}.
\end{align*}
}% End solution.