%
%
%	2016-12-02 (F)
%
%
\section{Exercise}

Using a line integral, compute the area of the (closed) annulus (a.k.a. ``donut'') $A \subseteq \reals^{2}$, bounded by the circles
\begin{align*}
x^{2} + y^{2}
&=
1
&
&\text{and}
&
x^{2} + y^{2}
&=
4.
\end{align*}
\fontHint{What should the answer be? Recall our mantra: Think geometrically, prove algebraically.}

\spaceSolution{3in}{% Begin solution.
Denote the closed annulus by
\begin{align*}
D
\defeq
\left\{(x,y) \in \reals^{2} \st 1 \leq x^{2} + y^{2} \leq 4\right\}.
\end{align*}
The area of the annulus is given by the double integral
\begin{align*}
\iint_{D} 1 \, \intd A.
\end{align*}
We can convert this double integral into a line integral using Green's theorem. To do so, we need a vector field $\fontVector{F}: D \rightarrow \reals^{2}$ whose ``curl'' is the given integrand. More precisely, if $\fontVector{F}$ has component functions $(F_{1},F_{2})$, then
\begin{align*}
\frac{\partial F_{2}}{\partial x} - \frac{\partial F_{1}}{\partial y}
=
1.
\end{align*}
Any such vector field $\fontVector{F}$ will do; common $\fontVector{F}$ are
\begin{align*}
\fontVector{F}(x,y)
&=
(-y,0),
&
\fontVector{F}(x,y)
&=
(0,x),
&
\fontVector{F}(x,y)
&=
\left(-\frac{1}{2} y,\frac{1}{2} x\right).
\end{align*}
}% End solution.