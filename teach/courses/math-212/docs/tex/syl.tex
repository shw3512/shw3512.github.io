% https://ga.rice.edu/syllabus/
% https://registrar.rice.edu/students/syllabus/
% http://cte.rice.edu/syllabus/



\begin{center}
{\Large Math 212: Multivariable Calculus}
\end{center}





\section{Disclaimer}

The information contained in this syllabus, other than the absence policy, are subject to change with reasonable advance notice.



%
%
%	Course information
%
%

\section{Course Information}



\subsection{Course Logistics}

\begin{tabular}{r c l}
Class time 	&	:	&	Monday, Wednesday, Friday, 09h00 -- 09h50	\\
Class room	&	:	&	George R. Brown (GRB) W212 (2F)	\\
Office hours	&	:	&	Friday, Sunday, 16h00 -- 17h30, Humanities (HUM) 118 (1F)	\\
TA sessions	&	:	&	Tuesday, Thursday, 19h00 -- 21h00, Abercrombie Labs (AEL) B209 (2F)	\\
			&		&	Wednesday, 19h00 -- 21h00, Herring (HRG) 129 (1F)
\end{tabular}



\subsection{Instructor Information}


\begin{tabular}{r c l}
Instructor	&	:	&	Stephen Wolff	\\
Office 	&	:	&	HBH 038 (basement)	\\
E-mail	&	:	&	\href{mailto:Stephen.Wolff@rice.edu?subject=[Math\%20212]}{Stephen.Wolff@rice.edu}
\end{tabular}



\subsection{Textbook}

The textbook for this course is \fontBookTitle{Calculus: early transcendentals}, by James Stewart.

\textbf{You should feel free to procure any edition of Stewart's textbook covering the following chapters} (chapter numbers refer to the 6th edition, which I use):
\begin{multicols}{2}
\begin{enumerate}
\setcounter{enumi}{11}
\item Vectors and the geometry of space
\item Vector functions
\item Partial derivatives
\item Multiple integrals
\item Vector calculus
\end{enumerate}
\end{multicols}
Editions as far back as the 5th edition contain almost exactly the same sections in these chapters. I haven't conducted exhaustive research on each edition, but I conjecture that they are very similar.

I will type and share the exercises for each assignment, so you needn't worry about exercise numbers being different if you purchase an older edition.

This course will not require online submissions via the textbook's access-controlled site, so you needn't worry about valid codes.

Two final remarks regarding Stewart's textbook:
\begin{enumerate}
\item Stewart's textbook is sometimes sold in halves. If you purchase half the textbook, make sure it is the second half, titled ``Multivariable calculus''.
\item The ``early transcendentals'' subtitle to the book is irrelevant to the chapters we will cover. (It pertains to when transcendental functions are introduced in the earlier chapters. To read more about this (pedagogical) distinction, visit [\href{http://math.stackexchange.com/questions/535283/whats-the-difference-between-early-transcendentals-and-late-transcendentals}{1}], [\href{https://answers.yahoo.com/question/index?qid=20070730231738AAixrGh}{2}], [\href{https://www.reddit.com/r/learnmath/comments/2ww1ab/intro_calculus_what_is_the_difference_between/}{3}].)
\end{enumerate}





%
%
%	Grading policy
%
%

\subsection{Grading Policy}

Grades will be based on the following four metrics.
\begin{enumerate}
\item Homework (H). There will be homework assigned daily. Homework is due promptly by 09h05 the following class. Late homework will not be accepted. Homework will be graded using a ``reasonable completion'' rubric. Your lowest five homework scores will be dropped.
\item Quizzes (Q). There will be an in-class quiz every day at 9h00 sharp. Quizzes come in two flavors: reading quizzes (RQ) and exercise quizzes (EQ). Reading quizzes are short 1- or 2-minute quizzes on key content in the assigned reading (pre-reading incentive). Exercise quizzes are longer 5- to 15-minute quizzes asking you to solve an exercise (exam practice). Quizzes are weighted as follows:
\begin{align*}
Q
=
50\% \times RQ + 50\% \times EQ.
\end{align*}
Make-up quizzes will not be given. Your lowest five RQ scores and lowest three EQ scores will be dropped.
\item Exams (E). There will be three exams. Exams are weighted as follows:
\begin{align*}
E
=
30\% \times E_{1} + 30\% \times E_{2} + 40\% \times E_{3}.
\end{align*}
All exams will be cumulative. See page \pageref{sec: Calendar} for the exam dates.
\item \LaTeX{} (L). There will be one short typesetting assignment. Essentially, you will type up your solution to a homework or quiz exercise. I will provide a template and typesetting assistance.
\end{enumerate}
Your grade will be the maximum of the following weighted averages:
\begin{center}
\begin{tabular}{r@{ = }r}
Average 1	&	$20\% \times H + 20\% \times Q + 55\% \times E + 5\% \times L$	\\
Average 2	&	$20\% \times H \hspace{21.5mm} + 75\% \times E + 5\% \times L$	\\
Average 3	&	$20\% \times Q + 75\% \times E + 5\% \times L$	\\
Average 4	&	$95\% \times E + 5\% \times L$	\\
\end{tabular}
\end{center}
\textbf{Caveat discipulus:} Regarding grades, homework is easy points. Quizzes are usually easier than exams. Both homework and quizzes are meant to train you for optimal performance on exams (and after). If Average 4 is your highest average, then your average is probably in trouble.





\subsection{Absence Policy}

Class attendance is strongly encouraged but not required. We are old enough to accept responsibility for our actions and decisions.





\section{Rice Honor Code}

As a student at Rice University, you pledge to uphold the Rice Honor Code, which you can find in the \href{http://honor.rice.edu/honor-system-handbook/}{Honor System Handbook}.

On homework, all resources are permitted. In particular, you are strongly encouraged to work with one another. The purpose of homework is to help you to learn and internalize the material.

On quizzes and exams, no external resources are permitted, unless the instructor explicitly indicates otherwise. The purpose of quizzes and exams is to show that you have internalized the material.





\section{Students with Disabilities}

Any student with a documented disability that requires accommodation is encouraged to contact both the course instructor and Disability Support Services (\href{mailto:adarice@rice.edu}{adarice@rice.edu}; Allen Center, Room 111).





%
%
%	Course objectives
%
%

\section{Course Objectives and Expected Learning Outcomes}

By the end of this course, you should know how to
\begin{itemize}
\item Describe the geometry of \href{https://en.wikipedia.org/wiki/Euclidean\_space}{euclidean space} $\reals^{n}$, particularly $\reals^{2}$ and $\reals^{3}$
\item Work fluently with \href{https://en.wikipedia.org/wiki/Vector-valued\_function}{vector-valued functions}
\item Compute \href{https://en.wikipedia.org/wiki/Partial\_derivative}{partial derivatives} and \href{https://en.wikipedia.org/wiki/Directional\_derivative}{directional derivatives}
\item Optimize in $\reals^{n}$ (including the \href{https://en.wikipedia.org/wiki/Lagrange\_multiplier}{method of Lagrange multipliers})
\item Compute \href{https://en.wikipedia.org/wiki/Iterated\_integral}{iterated integrals} in \href{https://en.wikipedia.org/wiki/Cartesian\_coordinate\_system\#Two\_dimensions}{rectangular}, \href{https://en.wikipedia.org/wiki/Polar\_coordinate\_system}{polar}, \href{https://en.wikipedia.org/wiki/Cylindrical\_coordinate\_system}{cylindrical}, \href{https://en.wikipedia.org/wiki/Spherical\_coordinate\_system}{spherical} coordinates
\item Perform a \href{https://en.wikipedia.org/wiki/Change\_of\_variables}{change of variables} (with application to multiple integrals, including computation of the \href{https://en.wikipedia.org/wiki/Jacobian\_matrix\_and\_determinant}{Jacobian matrix and determinant})
\item Compute \href{https://en.wikipedia.org/wiki/Line\_integral}{line integrals} of scalar- and vector-valued functions
\item Use \href{https://en.wikipedia.org/wiki/Divergence}{$\divergence$}, \href{https://en.wikipedia.org/wiki/Gradient}{$\gradient$}, and \href{https://en.wikipedia.org/wiki/Curl_(mathematics)}{$\curl$}, and have a (physical) sense of what they measure
\item Parametrize surfaces and compute \href{https://en.wikipedia.org/wiki/Surface\_integral}{surface integrals}
\item Use a few fancy theorems (\href{https://en.wikipedia.org/wiki/Green\%27s\_theorem}{Green's}, \href{https://en.wikipedia.org/wiki/Stokes\%27\_theorem\#Kelvin.E2.80.93Stokes\_theorem}{Stokes's}, and the \href{https://en.wikipedia.org/wiki/Divergence\_theorem}{divergence theorem})
\item Typeset basic documents using \href{https://en.wikipedia.org/wiki/LaTeX}{\LaTeX{}}
\end{itemize}




%
%
%	Calendar
%
%


\newpage

\section{Calendar}
\label{sec: Calendar}

Following is a preliminary schedule of topics. Section numbers refer to the 6th edition unless indicated otherwise (e.g., 8e refers to the 8th edition). Exercise numbers refer to the ``Exercises'' document (NOT Stewart's numbering!). Exercises are \emph{assigned} on the date of the line on which they appear and are \emph{due} the following class.
\begin{center}
\begingroup%
%\resizebox{\textwidth}{!}{%
\scriptsize
%\footnotesize
\begin{tabular}{*5{l}}
\hline\hline
Day	&	Date		&	Topics					&	Sections	&	Exercises	\\
\hline
M	&	22 Aug	&	Diagnostic quiz				&			&	\\
W	&	24 Aug	&	Coordinate systems; Vectors	&	12.1,12.2	&	\fontSectionNumber{12.1.}2,3,4,5,8;\fontSectionNumber{12.2.}2,3,4	\\
F	&	26 Aug	&	Inner product; Cross product	&	12.3,12.4	&	\fontSectionNumber{12.3.}1,4,5,7,11;\fontSectionNumber{12.4.}2,3,5,8	\\
\hline
M	&	29 Aug	&	Lines, planes				&	12.5		&	\fontSectionNumber{12.5.}1,3,5	\\
W	&	31 Aug	&	Cylinders, quadric surfaces	&	12.6		&	\fontSectionNumber{12.6.}1,2,3	\\
F	&	02 Sep	&	Vector-valued functions		&	13.1,13.2	&	\fontSectionNumber{13.1.}1,2,3;\fontSectionNumber{13.2.}1,2	\\
\hline
\fontHoliday{M}	&	\fontHoliday{05 Sep}	&	\multicolumn{2}{l}{\fontHoliday{University holiday --- no class}}	\\
W	&	07 Sep	&	Derivatives and integrals		&	13.2,13.3	&	\fontSectionNumber{13.2.}5,6;\fontSectionNumber{13.3.}1	\\
F	&	09 Sep	&	Arc length; Velocity, acceleration	&	13.3,13.4	&	\fontSectionNumber{13.3.}2;\fontSectionNumber{13.4.}1,2,3	\\
\hline
M	&	12 Sep	&	Functions on $\reals^{n}$		&	13.4,14.1	&	\fontSectionNumber{14.1.}1,3,4,6	\\
W	&	14 Sep	&	Limits, continuity			&	14.1,14.2	&	\fontSectionNumber{14.1.}8;\fontSectionNumber{14.2.}1,2,3,5	\\
F	&	16 Sep	&	Partial derivatives			&	14.3		&	\fontSectionNumber{14.3.}1,2,3,6,8,10	\\
\hline
M	&	19 Sep	&	Linear approximation			&	14.4		&	\fontSectionNumber{14.4.}1,3,4,6	\\
W	&	21 Sep	&	Chain rule					&	14.5		&	\fontSectionNumber{14.5.}1,3,5,7	\\
F	&	23 Sep	&	Directional derivative, gradient	&	14.6		&	\fontSectionNumber{14.6.}2,5,6,7,8,10	\\
\hline
M	&	26 Sep	&	Optimization				&	14.7		&	\fontSectionNumber{14.7.}1,3	\\
W	&	28 Sep	&	Review					&	12.1--14.7	&	Mock Exam 1 (Fall 2016)	\\
\fontExam{R}	&	\fontExam{29 Sep}	&	\fontExam{Midterm exam 1}	&	\fontExam{12.1--14.6}	&		\\
F	&	30 Sep	&	Lagrange multipliers			&	14.8		&	\fontSectionNumber{14.7.}5,7,8;\fontSectionNumber{14.8.}1,2,6	\\
\hline
M	&	03 Oct	&	Double integrals			&	15.1,15.2	&	\fontSectionNumber{15.1.}1,2;\fontSectionNumber{15.2.}1,3	\\
W	&	05 Oct	&	Iterated integrals			&	15.2,15.3	&	\fontSectionNumber{15.2.}5,6,7;\fontSectionNumber{15.3.}2	\\
F	&	07 Oct	&	Double integrals over general regions	&	15.3	&	\fontSectionNumber{15.3.}4,6,7,8	\\
\hline
\fontHoliday{M}	&	\fontHoliday{10 Oct}	&	\multicolumn{2}{l}{\fontHoliday{University holiday --- no class}}	\\
W	&	12 Oct	&	Polar coordinates			&	15.4		&	\fontSectionNumber{15.4.}1,2,3,4	\\
F	&	14 Oct	&	Applications				&	15.5		&	\fontSectionNumber{15.5.}1,3	\\
\hline
M	&	17 Oct	&	Surface area of graphs		&	16.6 (6e),15.5 (8e)	&	\fontSectionNumber{16.6.}7,9	\\
W	&	19 Oct	&	Triple integrals				&	15.6		&	\fontSectionNumber{15.6.}1,4,5	\\
F	&	21 Oct	&	Cylindrical, spherical coordinates	&	15.7,15.8	&	\fontSectionNumber{15.7.}2,3,4	\\
\hline
M	&	24 Oct	&	Cylindrical, spherical coordinates	&	15.7,15.8	&	\fontSectionNumber{15.8.}2,3,5	\\
W	&	26 Oct	&	Change of variables			&	15.9		&	\fontSectionNumber{15.9.}1,2	\\
F	&	28 Oct	&	Change of variables			&	15.9		&	\fontSectionNumber{15.9.}3,4,5	\\
\hline
M	&	31 Oct	&	Vector fields				&	16.1		&	\fontSectionNumber{16.1.}1,2,3,4	\\
W	&	02 Nov	&	Line integrals				&	16.2		&	\fontSectionNumber{16.2.}1,2,3	\\
F	&	04 Nov	&	Fundamental theorem		&	16.2,16.3	&	\fontSectionNumber{16.2.}4;\fontSectionNumber{16.3.}1,3,4	\\
\hline
M	&	07 Nov	&	Green's theorem			&	16.3,16.4	&	\fontSectionNumber{16.3.}6,8;\fontSectionNumber{16.4.}1	\\
W	&	09 Nov	&	Review					&	12.1--16.4	&	Mock Exam 2 (Fall 2016)	\\
\fontExam{R}	&	\fontExam{10 Nov}	&	\fontExam{Midterm exam 2}	&	\fontExam{12.1--16.4}	&		\\
F	&	11 Nov	&	Green's theorem			&	16.4		&	\fontSectionNumber{16.4.}2,3,5,6	\\
\hline
M	&	14 Nov	&	Curl, divergence			&	16.5		&	\fontSectionNumber{16.5.}1,2,3,5	\\
W	&	16 Nov	&	Parametric surfaces			&	16.6		&	\fontSectionNumber{16.6.}2,3,5	\\
F	&	18 Nov	&	Surface integrals			&	16.6,16.7	&	\fontSectionNumber{16.6.}6,8;\fontSectionNumber{16.7.}1	\\
\hline
M	&	21 Nov	&	Surface integrals			&	16.7		&	\fontSectionNumber{16.7.}3,5	\\
W	&	23 Nov	&	Stokes's, divergence theorem	&	16.8,16.9	&	\fontSectionNumber{16.8.}1,4;\fontSectionNumber{16.9.}1,4	\\
\fontHoliday{F}	&	\fontHoliday{25 Nov}	&	\multicolumn{2}{l}{\fontHoliday{University holiday --- no class}}	\\
\hline
M	&	28 Nov	&	Stokes's, divergence theorem	&	16.8,16.9	&	\fontSectionNumber{16.8.}2;\fontSectionNumber{16.9.}2	\\
W	&	30 Nov	&	Review					&	All		&	Mock Exam 3 (Fall 2016)	\\
F	&	02 Dec	&	Review					&	All		&		\\
\hline
\fontExam{TBA}	&	\fontExam{TBA}		&	\fontExam{Final exam}	&	\fontExam{12.1--16.10}	&		\\
\hline
\end{tabular}
\endgroup
\end{center}

% 6 4 8 9 10