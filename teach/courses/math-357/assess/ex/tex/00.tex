%%%%%%%%%%%%%%%%%%%%
%
%	Mock exam 00 : 2024--01--08 (M)
%
%%%%%%%%%%%%%%%%%%%%



%
%	Section 1 : Linear algebra
%

\section*{Linear algebra}



\subsection{Exercise \ref{sec : me00laQ1}}
\label{sec : me00laQ1}

(4 pt) State and prove the rank--nullity theorem, which relates dimensions of certain subspaces arising from a linear map.

\spaceSolution{2.5in}{% Begin solution.
}% End solution.



\subsection{Exercise \ref{sec : me00laQ2}}
\label{sec : me00laQ2}

Let $\reals$ denote the field of real numbers, let $\complexes$ denote the field of complex numbers, and let $V = \{p \in \complexes[t] \st \degree(p) \leq 2\}$ denote the set of univariate polynomials of degree at most $2$ with complex coefficients.

\begin{enumerate}[label=(\alph*)]
\item\label{itm : me00laQ2a} (2 pt) View $V$ as a vector space over $\complexes$, and view $\complexes$ as a vector space over $\reals$, with addition and scalar multiplication defined as usual. Give the dimension of each vector space.
\end{enumerate}

\spaceSolution{1in}{% Begin solution.
}% End solution.

\begin{enumerate}[resume, label=(\alph*)]
\item\label{itm : me00laQ2b} (2 pt) Can we view $V$ as a vector space over $\reals$? If so, what is its dimension?
\end{enumerate}

\spaceSolution{1in}{% Begin solution.
}% End solution.

\begin{enumerate}[resume, label=(\alph*)]
\item\label{itm : me00laQ2c} (2 pt) Use the above to make a conjecture.
\end{enumerate}

\spaceSolution{2in}{% Begin solution.
}% End solution.

%(4 pt) Let $K_{0}, K_{1}$ be fields; let $V$ be a vector space of dimension $d_{2}$ over $K_{1}$; and suppose that $K_{1}$ is a vector space of dimension $d_{1}$ over $K_{0}$.% Begin footnote.
%\footnote{As a concrete example, consider $K_{0} = \R$, $K_{1} = \C$, and $V = \C^{3}$, all with the ``usual'' structure, so $d_{1} = 2$ and $d_{2} = 3$.} % End footnote.
%Show that $V$ is a vector space of dimension $d_{1} d_{2}$ over $K_{0}$.
%
%\spaceSolution{3in}% End solution.





%
%	Section 2 : Group theory
%

\section*{Group theory}

\subsection{Exercise \ref{sec : me00gtQ1}}
\label{sec : me00gtQ1}

\begin{enumerate}[label=(\alph*)]
\item\label{itm : me0gtQ1a} (2 pt) Define a normal subgroup. Explain its importance.
\end{enumerate}

\spaceSolution{2in}{% Begin solution.
}% End solution.

\begin{enumerate}[resume, label=(\alph*)]
\item\label{itm : me0gtQ1b} (4 pt) From the same ``parent'' group of your choice, give two examples of proper, nontrivial subgroups, one of which is normal, the other of which is not. Justify your assertions.
\end{enumerate}

\spaceSolution{2in}{% Begin solution.
}% End solution.



\subsection{Exercise \ref{sec : me00gtQ2}}
\label{sec : me00gtQ2}

% Gallian 7e Exercise 5.36.

(4 pt) From the symmetric group $\symGp_{4}$, give an example of a cyclic subgroup of order $4$ and, separately, a noncyclic subgroup of order $4$. Justify your assertions.





%
%	Section 3 : Ring theory
%

\section*{Ring theory}



\subsection{Exercise \ref{sec : me00rtQ1}}
\label{sec : me00rtQ1}

\begin{enumerate}[label=(\alph*)]
\item\label{itm : me00rtQ1a} (2 pt) Define ``integral domain'' and, separately, ``field''.
\end{enumerate}

\spaceSolution{2in}{% Begin solution.
}% End solution.

\begin{enumerate}[resume, label=(\alph*)]
\item\label{itm : me00rtQ1b} (4 pt) Prove that every finite integral domain is a field.
\end{enumerate}

\spaceSolution{2.5in}{% Begin solution.
}% End solution.



\subsection{Exercise \ref{sec : me00rtQ2}}
\label{sec : me00rtQ2}

\begin{enumerate}[label=(\alph*)]
\item\label{itm : me00rtQ2a} (2 pt) Define ``euclidean domain'' and, separately, ``principal ideal domain''.
\end{enumerate}

\spaceSolution{2in}{% Begin solution.
}% End solution.

\begin{enumerate}[resume, label=(\alph*)]
\item\label{itm : me00rtQ2a} (4 pt) Prove that a euclidean domain is a principal ideal domain.
\end{enumerate}

\spaceSolution{2.5in}{% Begin solution.
}% End solution.



\subsection{Exercise \ref{sec : me00rtQ3}}
\label{sec : me00rtQ3}

Let $A$ be an integral domain.

\begin{enumerate}[label=(\alph*)]
\item\label{itm : me00rtQ3a} (2 pt) Define what it means for an element of $A$ to be prime and, separately, irreducible.
\end{enumerate}

\spaceSolution{2in}{% Begin solution.
}% End solution.

\begin{enumerate}[label=(\alph*)]
\item\label{itm : me00rtQ3b} (4 pt) Relate, as fully as possible, the notions of prime and irreducible elements. Provide proof or counterexample for your assertions.
\end{enumerate}

\spaceSolution{3in}{% Begin solution.
}% End solution.





%
%	Section 4 : Math 357
%

\section*{Math 357}

\subsection{Exercise \ref{sec : me00tcQ1}}
\label{sec : me00tcQ1}

% DF p 310 Ex 4

(4 pt) Let $p \in \integersPositive$ be prime. Prove that the polynomial $f(t) = \sum_{j = 0}^{p - 1} t^{j}$ is irreducible in $\rationals[t]$.

\spaceSolution{2in}{% Begin solution.
}% End solution.



\subsection{Exercise \ref{sec : me00tcQ2}}
\label{sec : me00tcQ2}

% DF p 881

(4 pt) Construct a character table for the symmetric group $\symGp_{3}$.

\spaceSolution{2in}{% Begin solution.
}% End solution.



\subsection{Exercise \ref{sec : me00tcQ3}}
\label{sec : me00tcQ3}

% DF pp 564--5, 568, 576

(4 pt) Let $p(t) = t^{3} - 2 \in \Q[t]$, and let $K$ denote the splitting field of $p$ over $\Q$. Draw a diagram of intermediate fields of $K | \Q$ and a diagram of subgroups of the galois group $\gal(K | \Q)$. In your diagrams, indicate all normal subgroups and all galois extensions of $\Q$.

\spaceSolution{2in}{% Begin solution.
}% End solution.