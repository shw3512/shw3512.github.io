%
% LQ03 : 2024--01--31 (W)
%

\noindent{}Let $R$ be a commutative ring, and let $t$ be an indeterminate. Consider the polynomial ring $R[t]$.

\begin{enumerate}[label=(\alph*)]
\item Define the degree function, $\deg$, on $R[t]$.
\end{enumerate}

\spaceSolution{2in}{% Begin solution.
Given any nonzero $f \in R[t]$, we may write $f$ in the form
\begin{align}
f
=
a_{m} t^{m} + \ldots + a_{0}%
\label{eq : LQ03 Nonzero Polynomial}
\end{align}
where $m \in \integersNonnegative$, each $a_{i} \in R$, and $a_{m} \neq 0$. Define
\begin{align*}
\deg
:
R[t]
&\rightarrow
\integersNonnegative \cup \{-\infty\}
\\
f
&\mapsto
\begin{dcases*}
m			&	if $f \neq 0$, $f$ as in \eqref{eq : LQ03 Nonzero Polynomial}	\\
-\infty	&	if $f = 0$
\end{dcases*}
\end{align*}
Some mathematicians prefer to define $\deg : R[t] \setminus \{0\} \rightarrow \integersNonnegative$, leaving the degree of the zero polynomial undefined.}% End solution.



\begin{enumerate}[resume, label=(\alph*)]
\item Let $p, q \in R[t]$. Prove that if $R$ is an integral domain, then
\begin{align*}
\deg p q
=
\deg p + \deg q
\end{align*}
Give an example to show that this equation can fail if $R$ is not an integral domain.
\end{enumerate}

\spaceSolution{4in}{% Begin solution.
For the (counter)example, consider the ring $\Z / (4)$ (also written $\Z / 4 \Z$), which is not an integral domain, and the polynomials $p = q = 2 t + 1$ in $(\Z / (4))[t]$. Then
\begin{align*}
p q
=
(2 t + 1) (2 t + 1)
=
4 t^{2} + 4 t + 1
\equiv
0 t^{2} + 0 t + 1
=
1
\end{align*}
so
\begin{align*}
\deg p q
=
0
\neq
2
=
1 + 1
=
\deg p + \deg q
\end{align*}

Now suppose $R$ is an integral domain. Case 1: Either $p$ or $q$ is the zero polynomial. In this case, $p q = 0$. With the convention that for all $n \in \integers \cup \{-\infty\}$, $-\infty + n = -\infty = n + -\infty$, we get
\begin{align*}
\deg p q
=
-\infty
=
\deg p + \deg q
\end{align*}
Case 2: Neither $p$ nor $q$ is the zero polynomial. Denote their leading terms by
\begin{align*}
\leadingTerm(p)
&=
a_{m} t^{m}
&
\leadingTerm(q)
&=
b_{n} t^{n}
\end{align*}
By hypothesis, $p \neq 0$ and $q \neq 0$; hence $a_{m} \neq 0$ and $b_{n} \neq 0$, and $\deg p = m$ and $\deg q = n$. Because $R$ is an integral domain, it follows that $a_{m} b_{n} \neq 0$. Hence, by definition of multiplication in $R[t]$,
\begin{align*}
\leadingTerm(p q)
=
a_{m} b_{n} t^{m + n}
\end{align*}
so
\begin{align*}
\deg p q
=
m + n
=
\deg p + \deg q
\end{align*}
as desired.}% End solution.