%
% LQ04 : 2024--02--14 (W)
%

\noindent{}Let $\Q$ denote the field of rational numbers; given a prime $p \in \integersPositive$, let $\F_{p} \isomorphism \Z / (p)$ denote the finite field with $p$ elements; and let $t$ be an indeterminate. For each of the quotient rings below, characterize its algebraic structure as ``field'', ``integral domain but not field'', or ``ring but not integral domain''. Justify your characterization.

\begin{align*}
R_{1}
&=
\F_{5}[t] / (t^{3} - t^{2} + 2 t - 1)
&
R_{2}
&=
\Q[t] / (t^{3} - t^{2} + 2 t - 1)
&
R_{3}
&=
\Q[t] / (t^{6} - 300)
\end{align*}

\spaceSolution{5in}{% Begin solution.
We analyze each quotient ring in turn.

$R_{1}$ : Ring but not integral domain. Let $f_{1} = t^{3} - t^{2} + 2 t - 1 \in \F_{5}[t]$. Because $\deg f_{1} = 3$, $f_{1}$ is reducible if and only if it has a linear factor $t - \alpha$ for some $\alpha \in \F_{5}$, which we have shown is equivalent to the statement that $\alpha$ is a zero of the function $f_{1} : \F_{5} \rightarrow \F_{5}$. Evaluating the function $f_{1}$ at the five elements of $\F_{5}$, we find that $f_{1}(-1) = 0$.% Begin footnote.
\footnote{One can check that $-1$ is the only zero of $f_{1}$ in $\F_{5}$, and that it has multiplicity $1$.} % End footnote.
Thus $f_{1}$ is reducible; in fact,
\begin{align*}
f_{1}
=
(t + 1) (t^{2} - 2 t - 1)
\end{align*}
It follows that $t - 1$ and $t^{2} - 2 t - 1$ are zero divisors in $R_{1}$. Hence $R_{1}$ is a (commutative) ring that is not an integral domain.

$R_{2}$ : Field. Let $f_{2} = t^{3} - t^{2} + 2 t - 1 \in \Z[t]$. We may be tempted to apply the reduction homomorphism corresponding to the proper ideal $(5) \ideal \Z$ to $f_{2}$. This gives the polynomial $f_{1} \in \F_{5}[t]$, which we just showed is reducible. However, this does not imply that $f_{2}$ is reducible! Indeed, if we apply the reduction homomorphism corresponding to the proper ideal $(2) \ideal \Z$,% Begin footnote.
\footnote{An analogous argument works with other proper ideals of $\Z$; for example, the principal ideals generated by one of $3, 4, 6, 8, 9, 10, 12, 13$.} % End footnote.
then we get
\begin{align*}
\overline{f}_{2}
=
t^{3} + t^{2} + 1
\end{align*}
Because $\deg \overline{f}_{2} = 3$, $\overline{f}_{2}$ is reducible if and only if it has a linear factor, which is equivalent to the function $\overline{f}_{2}$ having a zero in $\F_{2}$. It is straightforward to check that the function $\overline{f}_{2}$ has no zero in $\F_{2}$, so $\overline{f}_{2}$ is irreducible, and thus $f_{2} \in \Z[t]$ is irreducible. Finally, Gau\ss's lemma implies that $f_{2} \in \Q[t]$ is irreducible. Hence $(f_{2}) \idealeq \Q[t]$ is maximal, so $R_{2}$ is a field.

$R_{3}$ : Field. Let $f_{3} = t^{6} - 300 \in \Z[t]$. Note that $300 = 2^{2} \cdot 3 \cdot 5^{2}$. In particular, $3$ divides all coefficients of $f_{3}$ except the leading coefficient, and $3^{2}$ does not divide the constant term. Thus we may apply the Eisenstein--Sch\"{o}nemann criterion to $f_{3}$ using the prime ideal $(3) \ideal \Z$ to conclude that $f_{3}$ is irreducible in $\Q[t]$. Hence $(f_{3}) \idealeq \Q[t]$ is maximal, so $R_{3}$ is a field.}% End solution