\documentclass[oneside, english, 11pt]{article}
\usepackage{geometry}
\geometry{letterpaper}
\geometry{left=1in, top=1in, right=1in, bottom=1in, nohead}
\setlength\parindent{0.25in}



\usepackage{amsmath}
%\usepackage{amssymb}
\usepackage{enumitem}
\usepackage{ifthen}
%\usepackage{mathtools}

\newcommand{\showSolutions}{1}% Show solutions?
% Allowed values:
% - 0 = Do not show
% - 1 = Show

% Display or do not display solutions.
\ifthenelse{\equal{\showSolutions}{1}}{%
  \newcommand{\spaceSolution}[2]{#2}
}%
{%
  \newcommand{\spaceSolution}[2]{\vspace{#1}}
}%



% Sets fonts. Euler for math | Palatino for rm | Helvetica for ss | Courier for tt
% http://www.tug.org/mactex/fonts/LaTeX_Preamble-Font_Choices.html

\renewcommand{\rmdefault}{ppl}% rm
\linespread{1.05}% Palatino needs more leading
\usepackage[scaled]{helvet}% ss
\usepackage{courier}% tt
%\usepackage{euler}% math
\usepackage{eulervm}% math --- a better implementation of the euler package (not in gwTeX)
\normalfont
\usepackage[T1]{fontenc}



\newcommand{\fontHint}[1]{\emph{Hint:} #1}



\newcommand{\divides}{\,|\,}
\newcommand{\isomorphism}{\cong}
\newcommand{\order}{\#}
\newcommand{\st}{\, | \,}



\title{Math 357\\Expositional homework 02}
\author{}
\date{Assigned: 2024--01--22 (M)\\\hspace{0.33in}Due: 2024--02--08 (R)}



% Document begins here.

\begin{document}

\maketitle

The goal of this homework is to recall ideas from general ring theory and engage them in the specific setting of polynomial rings. The exercises are adapted from Dummit \&{} Foote, 3e, Exercises 9.2.1--5.

Let $F$ be a field, let $t$ be an indeterminate over $F$, and let $f \in F[t]$.

\begin{enumerate}[label=(\alph*)]
\item\label{itm : EH02a} Let $\deg f = n \geq 1$. For each $g \in F[t]$, let $\overline{g}$ denote the residue of $g$ under the natural projection map $\varphi : F[t] \rightarrow F[t] / (f)$. Prove that for each $\overline{g} \in F[t] / (f)$ there exists a unique polynomial $g_{0} \in F[t]$ such that $\deg g_{0} \leq n - 1$ and $\overline{g}_{0} = \overline{g}$.
\item\label{itm : EH02b} Prove that $F[t] / (f)$ is a field if and only if $f$ is irreducible.
\item\label{itm : EH02c} Let $f = \prod_{i = 1}^{n} p_{i}$ be a factorization of $f$ into irreducible elements. Describe all ideals in the ring $F[t] / (f)$, in terms of the $p_{i}$.
\item\label{itm : EH02d} Prove that $F[t]$ has infinitely many prime elements. \fontHint{See Exercise 9.2.4 (p 301).}
\item\label{itm : EH02e} Further assume that $F$ is a finite field, of order $q$. Let $\deg f = n \geq 1$. Prove that $F[t] / (f)$ has exactly $q^{n}$ elements. \fontHint{Explain how to view the result of Exercise \ref{itm : EH02a} in the framework of vector spaces. See Exercise 9.2.1 (p 301).}
\end{enumerate}



%\spaceSolution{0in}{% Begin solution.
\subsection*{Solutions}



\subsubsection*{Exercise \ref{itm : EH02a}}

Note that, despite what loose use of notation may lead us to think, $\overline{g} \in F[t] / (f)$ is a \emph{coset} $g + (f)$, not an \emph{element} $g_{0} \in F[t]$. Specifically, $\overline{g}$ is not the remainder $g_{0}$ when dividing $g$ by $f$. This exercise essentially asks us to show that we may identify $g_{0}$ with $\overline{g}$.

In our argument below, notice how almost everything we want comes from the division algorithm on $F[t]$ associated with the norm on $F[t]$ induced by the degree function.

Our first step is to associate to each coset $\overline{g}$ a polynomial $g_{0} \in F[t]$ with the desired properties. Let $\overline{g} \in F[t] / (f)$, and choose any coset representative $g \in \overline{g}$. (Note that $g \in F[t]$.) By the division algorithm on $F[t]$,% Begin footnote.
\footnote{See DF3e, p 299.} % End footnote.
there exist unique $q_{g}, r_{g} \in F[t]$ such that
\begin{align}
g
=
q_{g} f + r_{g}%
\label{eq : EH02a}
\end{align}
with $r_{g} = 0$ or $\deg r_{g} < \deg f = n - 1$.% Begin footnote.
\footnote{Note that the second condition here is $N(r_{g}) < N(f)$, rewritten using the fact that the norm function $N$ we have chosen and the degree function $\deg$ agree on all nonzero input.} % End footnote.
We may rewrite equation \eqref{eq : EH02a} as $g - r_{g} = q_{g} f \in (f)$, so $g + (f) = r_{g} + (f)$; that is, $\overline{g} = \overline{r}_{g}$.

We claim $r_{g}$ is independent of the choice of coset representative. To see this, let $g_{2} \in \overline{g}$ be any coset representative. Then $g + (f) = \overline{g} = g_{2} + (f)$, if and only if $g_{2} - g \in (f)$, if and only if there exists some $q \in F[t]$ such that $g_{2} = q f + g$. By the division algorithm on $F[t]$, there exist unique $q_{2}, r_{2} \in F[t]$ such that
\begin{align*}
g_{2}
=
q_{2} f + r_{2}
\end{align*} 
with $r_{2} = 0$ or $\deg r_{2} < \deg f$. From our work above, we also have
\begin{align}
(q + q_{g}) f + r_{g}
=
q f + q_{g} f + r_{g}
=
q f + g
=
g_{2}%
\label{eq : EH02a Existence}
\end{align}
with $r_{g} = 0$ or $\deg r_{g} < \deg f$. Viewing equation \eqref{eq : EH02a Existence} from right to left, we see that we may view it also a ``result'' of the division algorithm (when dividing $g_{2}$ by $f$). Therefore, by the uniqueness of quotient and remainder, we must have
\begin{align*}
q_{2}
&=
q + q_{g}
&
&\text{and}
&
r_{2}
=
r_{g}
\end{align*}
In particular, the remainder terms are the same. This proves existence of the desired $g_{0}$, namely, set $g_{0} = r_{g}$.

To prove uniqueness, let $g_{0}, h_{0} \in F[t]$ satisfy the given conditions. By the condition on their residues, $\overline{g}_{0} = \overline{g} = \overline{h}_{0}$, so by definition $h_{0} - g_{0} \in (f)$, which is equivalent to the existence of $q \in F[t]$ such that
\begin{align}
h_{0} - g_{0}
=
q f%
\label{eq : EH02a Uniqueness}
\end{align}
By the condition on their degree, and using the fact that $F$ is a field and therefore an integral domain (for the first equality), it follows that
\begin{align*}
\deg q + \deg f
=
\deg(q f)
=
\deg(h_{0} - g_{0})
\leq
\max(\deg h_{0}, \deg g_{0})
<
\deg f
\end{align*}
By hypothesis, $\deg f \geq 0$, so
\begin{align*}
\deg q
<
0
\end{align*}
which is equivalent to $q = 0$ (in $F[t]$). Substituting this into equation \eqref{eq : EH02a Uniqueness}, we conclude that $h_{0} = g_{0}$, as desired.



\subsubsection*{Exercise \ref{itm : EH02b}}

Let $f \in F[t]$. Then $F[t] / (f)$ is a field if and only if $(f)$ is maximal.% Begin footnote.
\footnote{See DF3e, Proposition 7.12, p 254.} % End footnote.
Because $F[t]$ is a principal ideal domain, $(f)$ is maximal if and only if $(f)$ is a nonzero prime ideal,% Begin footnote.
\footnote{See DF3e, Proposition 8.7, p 280.} % End footnote.
if and only if $f$ is a nonzero prime element,% Begin footnote.
\footnote{By definition of a prime element; see DF3e, p 284..} % End footnote.
if and only if $f$ is irreducible.% Begin footnote.
\footnote{See DF3e, Proposition 8.11, p 284.}% End footnote.



\subsubsection*{Exercise \ref{itm : EH02c}}

Let $A = \{1, \ldots, n\}$. Then we may write the given factorization of $f$ into irreducible elements as $f = \prod_{i \in A} p_{i}$.

By the lattice isomorphism theorem,% Begin footnote.
\footnote{See DF3e, Theorem 7.8, p 246.} % End footnote.
there exists an inclusion-preserving bijection between the ideals of $F[t] / (f)$ and the ideals of $F[t]$ that contain $(f)$. $F[t]$ is a euclidean domain, and hence a principal ideal domain. Thus, by definition, every ideal of $F[t]$ has the form $(g)$ for some $g \in F[t]$. Also recall that containment relations among principal ideals encode divisibility properties. Specifically, $(b) \subseteq (a)$ if and only if $a \divides b$.% Begin footnote.
\footnote{See DF3e, p 252.} % End footnote.% (???) Requires commutative ring?
It follows that an ideal $I$ of $F[t]$ contains $(f)$ if and only if there exists a subset $B \subseteq A$ and a polynomial $g = \prod_{i \in B} p_{i}$ such that $I = (g)$, so by the lattice isomorphism theorem, every ideal in $F[t] / (f)$ is the image of such an ideal $(g) \in F[t]$ under the quotient map $F[t] \mapsto F[t] / (f)$.



\subsubsection*{Exercise \ref{itm : EH02d}}

Let $f \in F[t]$. Because $F[t]$ is a principal ideal domain, $f$ is a prime element% Begin footnote.
\footnote{By definition, prime elements are nonzero. See DF3e, p 284.} % End footnote.
if and only if $f$ is irreducible.% Begin footnote.
\footnote{See DF3e, Proposition 8.11, p 284.} % End footnote.
All polynomials of degree $1$ are irreducible, % Begin footnote.
\footnote{One can give a quick proof using an argument on the degree of any factorization.} % End footnote.
and because $F$ is a field, $F[t]$ has at least two polynomials of degree $1$, namely, $t$ and $t - 1$. Thus $F[t]$ has at least two prime elements.

Suppose for the sake of contradiction that $F[t]$ has finitely many prime elements, denote them $f_{1}, \ldots, f_{n}$. Consider the polynomial
\begin{align}
f
=
1 + \prod_{i = 1}^{n} f_{i}%
\label{eq : EH02d fi}
\end{align}
By our argument above, $\deg f \geq 2$. Because $F[t]$ is a euclidean domain, it is a unique factorization domain. Therefore $f$ has a factorization into finitely many irreducible elements, say
\begin{align}
f
=
\prod_{i = 1}^{m} g_{i}%
\label{eq : EH02d gi}
\end{align}
Because $\deg f \geq 2$, $m \geq 1$. We claim that $g_{1}$ (more generally, any $g_{i}$) is not associate to any $f_{i}$.% Begin footnote.
\footnote{It is sufficient to show that $g_{1} \neq f_{i}$ for any $i$. (Why?)} % End footnote.
Suppose for the sake of contradiction that it is. Reindex the $f_{i}$ so that $g_{1}$ is associate to $f_{1}$. Then by definition, there exists a $u \in (F[t])^{\times} \isomorphism F^{\times}$ such that
\begin{align}
f_{1} = u g_{1}%
\label{eq : EH02d Unit}
\end{align}
Equating the two expressions \eqref{eq : EH02d fi} and \eqref{eq : EH02d gi} for $f$ above, solving for $1$, factoring out $f_{1}$, and using equation \eqref{eq : EH02d Unit}, we get
\begin{align*}
1
=
\left(u \prod_{i = 2}^{m} g_{i} - \prod_{i = 2}^{n} f_{i}\right) f_{1}
\end{align*}
This equation shows that $f_{1} \in (F[t])^{\times}$, contradicting the hypothesis that $f_{1}$ is irreducible (and hence, by definition, not a unit). Thus $g_{1}$ is not associate to any $f_{i}$, and therefore $g_{1}$ is an irreducible, and hence prime, element of $F[t]$ not in the list $f_{1}, \ldots, f_{n}$, as desired.

If $F$ is infinite, then the set $\{t - \alpha \st \alpha \in F\}$ of polynomials of degree $1$ is already a set of infinitely many irreducible, and hence prime, elements.



\subsubsection*{Exercise \ref{itm : EH02e}}

We may give the polynomial ring $F[t]$ the structure of an (infinite-dimensional) $F$-vector space,% Begin footnote.
\footnote{How do we define this structure?} % End footnote.
with one basis $(1, t, t^{2}, \ldots)$. In this view, the ideal $(f)$ is an $F$-subspace; Exercise \ref{itm : EH02a} implies that the quotient space $F[t] / (f)$ has as one basis $(\overline{1}, \overline{t}, \ldots, \overline{t}^{n - 1})$, where $\overline{t}$ denotes the image of $t$ under the quotient map $F[t] \rightarrow F[t] / (f)$, etc.% Begin footnote.
\footnote{Can you justify these assertions?} % End footnote.
In particular, $\dim_{F} (F[t] / (f)) = n$. From linear algebra, we know that a finite-dimensional $F$-vector space of dimension $n$ is isomorphic to $F^{n}$. Therefore, if $F$ is a finite field with $q$ elements, then
\begin{align*}
\order(F[t] / (f))
=
\order(F^{n})
=
q^{n}
\end{align*}
as desired.%
%}% End solution.



\end{document}