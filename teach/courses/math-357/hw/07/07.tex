\documentclass[oneside, english, 11pt]{article}
\usepackage{geometry}
\geometry{letterpaper}
\geometry{left=1in, top=1in, right=1in, bottom=1in, nohead}
\setlength\parindent{0.25in}



\usepackage{amsmath}
\usepackage{amsthm}
\usepackage{enumitem}



% Sets fonts. Euler for math | Palatino for rm | Helvetica for ss | Courier for tt
% http://www.tug.org/mactex/fonts/LaTeX_Preamble-Font_Choices.html

\renewcommand{\rmdefault}{ppl}% rm
\linespread{1.05}% Palatino needs more leading
\usepackage[scaled]{helvet}% ss
\usepackage{courier}% tt
%\usepackage{euler}% math
\usepackage{eulervm}% math --- a better implementation of the euler package (not in gwTeX)
\normalfont
\usepackage[T1]{fontenc}



\newcommand{\fontDefWord}[1]{\textbf{#1}}
\newcommand{\fontField}[1]{\mathbf{#1}}
\newcommand{\fontHint}[1]{\emph{Hint:} #1}



\newcommand{\Aut}{\automorphisms}
\DeclareMathOperator{\automorphisms}{Aut}
\newcommand{\C}{\complexes}
%\DeclareMathOperator{\characteristic}{char}
\newcommand{\complexes}{\fontField{C}}
%\newcommand{\divides}{\,|\,}
%\DeclareMathOperator{\Endomorphisms}{End}
%\newcommand{\F}{\fontField{F}}
\newcommand{\fixedField}{\mathcal{F}}
\newcommand{\Gal}{\galoisGroup}
\DeclareMathOperator{\galoisGroup}{Gal}
%\DeclareMathOperator{\generalLinear}{GL}
%\newcommand{\GL}{\generalLinear}
%\DeclareMathOperator{\Hom}{Hom}
\newcommand{\id}{\identityMap}
\DeclareMathOperator{\identityMap}{id}
\newcommand{\integers}{\fontField{Z}}
%\newcommand{\integersPositive}{\integers_{> 0}}
%\newcommand{\iso}{\isomorphism}
%\newcommand{\isomorphism}{\cong}
%\newcommand{\notDivides}{{}\hspace{-1mm}{\not\hspace{-.15em}\divides{}}\,}
\newcommand{\order}{\#}
\newcommand{\Q}{\rationals}
\newcommand{\rationals}{\fontField{Q}}
\DeclareMathOperator{\Span}{Span}
\newcommand{\st}{{\, | \,}}
\newcommand{\Z}{\integers}



\newtheorem{theorem}{Theorem}
\newtheorem{lemma}[theorem]{Lemma}



\title{Math 357\\Expositional homework 07}
\author{}
\date{Assigned: 2024--04--12 (F)\\\hspace{9.7mm}Due: 2024--04--19 (F)}



% Document begins here.

\begin{document}

\maketitle

The goal of this homework is to work with elementary theory and examples in galois theory to better understand the essential building blocks.

\begin{enumerate}[label=(\alph*)]
\item\label{itm : EH07a} Let $K : K_{0}$ be a field extension; and let $\Aut(K : K_{0})$ be the automorphisms of $K$ that fix $K_{0}$, a group under composition of functions. Let $K_{i}$ be an intermediate field of $K : K_{0}$, let $H_{i}$ be a subgroup of $\Aut(K : K_{0})$, and let $\fixedField(H_{i})$ denote the fixed field of $H_{i}$:
\begin{align*}
\fixedField(H_{i})
=
\{\alpha \in K \st \forall \sigma \in H_{i}, \sigma(\alpha) = \alpha\}
\end{align*}
Prove that the associations
\begin{align*}
K_{i}
&\mapsto
\Aut(K : K_{i})
&
H_{i}
&\mapsto
\fixedField(H_{i})
\end{align*}
are inclusion-reversing, that is, if $K_{1} \subseteq K_{2}$ and $H_{1} \subseteq H_{2}$, then
\begin{align*}
\Aut(K : K_{1})
&\supseteq
\Aut(K : K_{2})
&
\fixedField(H_{1})
&\supseteq
\fixedField(H_{2})
\end{align*}
\item\label{itm : EH07b} Prove Proposition 14.5:% Begin footnote.
\footnote{See DF3e, pp 561--2.} % End footnote.
Let $K_{0}$ be a field, let $f \in K_{0}[t]$, and let $\tilde{K}_{0, f} : K_{0}$ be a splitting field for $f$ over $K_{0}$. Then
\begin{align*}
\# \Aut(\tilde{K}_{0, f} : K_{0})
\leq
[\tilde{K}_{0, f} : K_{0}]
\end{align*}
with equality if $f$ is separable. You may take as your starting point our diagram from class (see Classes 35 and 36).
\item\label{itm : EH07c} Let $\alpha = \sqrt{2} + \sqrt{5} \in \C$.
\begin{enumerate}[label=(\roman*)]
\item\label{itm : EH07c1} Find the minimal polynomial $m_{\alpha, \Q}$ for $\alpha$ over $\Q$.
\item\label{itm : EH07c2} Prove that $\Q(\alpha) = \Q(\sqrt{2}, \sqrt{5})$.
\item\label{itm : EH07c3} Prove that $m_{\alpha, \Q}$ splits completely in $\Q(\alpha)$. \fontHint{Use part \ref{itm : EH07c2}.}
\item\label{itm : EH07c4} Specify all automorphisms in the galois group $\Gal(\Q(\alpha) : \Q)$. State a (more common) group isomorphic to $\Gal(\Q(\alpha) : \Q)$, and draw its subgroup lattice.
\item\label{itm : EH07c5} Use part \ref{itm : EH07c4} and the fundamental theorem of galois theory to draw the lattice of intermediate fields for $\Q(\alpha) : \Q$.
\end{enumerate}
\end{enumerate}



\subsection*{Solution to part \ref{itm : EH07a}}

Let $K_{1} \subseteq K_{2} \subseteq K$, and let $\sigma \in \Aut(K : K_{2})$. Then by definition of $\Aut(K : K_{2})$, $\sigma$ is an automorphism of $K$ that fixes $K_{2}$; that is, for each $a \in K_{2}$, $\sigma(a) = a$. Because $K_{1} \subseteq K_{2}$, this shows that $\sigma$ fixes $K_{1}$. Hence $\sigma \in \Aut(K : K_{1})$.

Let $H_{1} \subseteq H_{2} \subseteq \Aut(K : K_{0})$, and let $\alpha \in \fixedField(H_{2})$. Then by definition of $\fixedField(H_{2})$, for all $\sigma \in H_{2}$, $\sigma(\alpha) = \alpha$. Because $H_{1} \subseteq H_{2}$, this shows that for all $\sigma \in H_{1}$, $\sigma(\alpha) = \alpha$. Thus $\alpha \in \fixedField(H_{1})$.



\subsection*{Solution to part \ref{itm : EH07b}}

We sketched a proof together in class. See DF3e, pp 561--562 for a more detailed exposition.



\subsection*{Solution to part \ref{itm : EH07c}}

Part \ref{itm : EH07c1}: One approach is to attempt to ``power away'' the radicals by successively isolating one radical and raising both sides of the equation to an appropriate power to cancel the radical. Taking this approach, we compute
\begin{align*}
\alpha
&=
\sqrt{2} + \sqrt{5}
&
&\Leftrightarrow
&
\alpha - \sqrt{5}
&=
\sqrt{2}
&
&\Rightarrow
&
\alpha^{2} - 2 \sqrt{5} \alpha + 5
&=
2
\\
&
&
&\Leftrightarrow
&
\alpha^{2} + 3
&=
2 \sqrt{5} \alpha
&
&\Rightarrow
&
\alpha^{4} + 6 \alpha^{2} + 9
&=
20 \alpha^{2}
\end{align*}
so
\begin{align*}
\alpha^{4} - 14 \alpha^{2} + 9
=
0
\end{align*}
Viewing this equation as a polynomial in $\Q[t]$ evaluated at $t = \alpha$, we define
\begin{align*}
m
=
t^{4} - 14 t^{2} + 9
\end{align*}
This polynomial has coefficients in $\Q$ and is monic (by inspection), and it has $\alpha$ as a zero (by construction).

For $m$ to be the minimal polynomial of $\alpha$ over $\Q$, it remains only to show that $m$ is irreducible. To do this, one can show that $m$ has no factors of degree $1$ or degree $2$, either directly in $\Q$ or in a quotient ring of $\Z$ (for example, $\Z / (5)$. Alternatively, one can use knowledge of the set of zeros of $m$ (see our response to part \ref{itm : EH07c3}) to show that (i) no zero of $m$ is in $\Q$; and (ii) for all distinct zeros $\alpha_{1}, \alpha_{2}$ of $m$, the degree-$2$ polynomial $(t - \alpha_{1}) (t - \alpha_{2})$ has a coefficient that is not in $\Q$. Note that with this latter approach, it suffices to check three such products (fix one zero $\alpha_{1}$ and let $\alpha_{2}$ run through the other three).

Another approach is to implement the technique in exercise (f) on expositional homework 06: Find a matrix representation of the ``multiplication by $\alpha$'' map on some finite-degree field extension $K : \Q$ with $\alpha \in K$. This approach requires that we find such an extension (as shown in part \ref{itm : EH07c2}, the extension field $K = \Q(\sqrt{2}, \sqrt{5})$ will do) and a basis for $K$ as a $\Q$-vector space. As above, we need to verify that the polynomial we obtain is irreducible. If not, then we need to find an irreducible factor that has $\alpha$ as a zero.

Part \ref{itm : EH07c2}: We prove set inclusion in both directions. By definition, $\alpha = \sqrt{2} + \sqrt{5}$, so $\Q(\sqrt{2}, \sqrt{5})$ is a field that contains both $\Q$ and $\alpha$, and hence it contains the field generated by $\alpha$ over $\Q$: $\Q(\alpha) \subseteq \Q(\sqrt{2}, \sqrt{5})$. For the reverse inclusion, because $\Q(\alpha)$ is a field, it is closed under field operations: addition, multiplication, and taking additive and multiplicative inverses. In particular, $\Q(\alpha)$ contains
\begin{align*}
\alpha^{-1}
=
\frac{1}{\sqrt{5} + \sqrt{2}}
=
\frac{\sqrt{5} - \sqrt{2}}{(\sqrt{5} + \sqrt{2}) (\sqrt{5} - \sqrt{2})}
=
\frac{1}{3} (\sqrt{5} - \sqrt{2})
\end{align*}
so it also contains
\begin{align*}
3 \alpha^{-1}
=
\sqrt{5} - \sqrt{2}
\end{align*}
so it also contains
\begin{align*}
\frac{1}{2} (\alpha + 3 \alpha^{-1})
&=
\sqrt{5}
&
&\text{and}
&
\frac{1}{2} (\alpha - 3 \alpha^{-1})
&=
\sqrt{2}
\end{align*}
That is, $\Q(\alpha)$ is a field that contains $\Q$ and the elements $\sqrt{2}$ and $\sqrt{5}$, so it contains the field generated by $\{\sqrt{2}, \sqrt{5}\}$ over $\Q$: $\Q(\alpha) \supseteq \Q(\sqrt{2}, \sqrt{5})$.

Part \ref{itm : EH07c3}: One can check that the four elements $\pm{}\sqrt{2} \pm{} \sqrt{5} \in \C$ are zeros of $m_{\alpha, \Q}$ (which has degree $4$, so these are all the zeros) and are distinct. Each element is in $\Q(\sqrt{2}, \sqrt{5}) = \Q(\alpha)$. Therefore, $m_{\alpha, \Q}$ splits completely in $\Q(\alpha)$.

Note that we have shown that $\Q(\alpha) : \Q$ is a splitting field of the separable polynomial $m_{\alpha, \Q} \in \Q[t]$. Thus the field extension $\Q(\alpha) : \Q$ is galois.

Part \ref{itm : EH07c4}: To specify automorphisms in $\Gal(\Q(\alpha) : \Q)$, we will find it convenient to use the fact that $\Q(\alpha) = \Q(\sqrt{2}, \sqrt{5})$, which we proved in part \ref{itm : EH07c2}. This equality shows that $\{\sqrt{2}, \sqrt{5}\}$ generates $\Q(\alpha)$ over $\Q$. Therefore, to specify an element $\sigma$ of $\Gal(\Q(\alpha) : \Q)$, it suffices to specify the images $\sigma(\sqrt{2})$ and $\sigma(\sqrt{5})$.

The elements $\sqrt{2}$ and $\sqrt{5}$ have minimal polynomials over $\Q$ of
\begin{align*}
m_{\sqrt{2}, \Q}
&=
t^{2} - 2
&
m_{\sqrt{5}, \Q}
&=
t^{2} - 5
\end{align*}
respectively. Let $\sigma \in \Gal(\Q(\alpha) : \Q)$. The requirement that $\sigma$ must permute the zeros of any irreducible polynomial gives us four candidate automorphisms:
\begin{align*}
\sqrt{2}
&\mapsto
\sqrt{2}
&
\sqrt{2}
&\mapsto
-\sqrt{2}
&
\sqrt{2}
&\mapsto
\sqrt{2}
&
\sqrt{2}
&\mapsto
-\sqrt{2}
\\
\sqrt{5}
&\mapsto
\sqrt{5}
&
\sqrt{5}
&\mapsto
\sqrt{5}
&
\sqrt{5}
&\mapsto
-\sqrt{5}
&
\sqrt{5}
&\mapsto
-\sqrt{5}
\end{align*}
In part \ref{itm : EH07c3} we showed that the field extension $\Q(\alpha) : \Q$ is galois. Therefore
\begin{align*}
\order \Gal(\Q(\alpha) : \Q)
=
[\Q(\alpha) : \Q]
=
\deg m_{\alpha, \Q}
=
4
\end{align*}
Thus all four candidate automorphisms are indeed automorphisms in $\Gal(\Q(\alpha) : \Q)$.

Let
\begin{align*}
\tau_{2}
:
{}&\sqrt{2}
\mapsto
-\sqrt{2}
&
\tau_{5}
:
{}&\sqrt{2}
\mapsto
\sqrt{2}
\\
&\sqrt{5}
\mapsto
\sqrt{5}
&
&\sqrt{5}
\mapsto
-\sqrt{5}
\end{align*}
It is straightforward to check that $\{\tau_{2}, \tau_{5}\}$ generate $\Gal(\Q(\alpha) : \Q)$ and that each automorphism has order $2$:
\begin{align*}
\Gal(\Q(\alpha) : \Q)
=
\langle{}\tau_{2}, \tau_{5} \st \tau_{2}^{2}, \tau_{5}^{2}\rangle{}
\end{align*}
Thus $\Gal(\Q(\alpha) : \Q)$ is isomorphic to the Klein four-group. Its subgroup lattice is given in DF3e, p 567. (The automorphisms that we have denoted $\tau_{2}$ and $\tau_{5}$ correspond to $\sigma$ and $\tau$ in this diagram.)

Part \ref{itm : EH07c5}: The subfield lattice (aka lattice of intermediate fields) is given in DF3e, p 568. (Replace $\sqrt{3}$ with $\sqrt{5}$ throughout.) We use the fundamental theorem of galois theory to match subgroup with subfield. For example, let's compute the subfield corresponding to the subgroup
\begin{align*}
H
=
\langle{}\tau_{2} \tau_{5}\rangle{}
=
\{\id_{\Q(\sqrt{2}, \sqrt{5})}, \tau_{2} \tau_{5}\}
\end{align*}
By the fundamental theorem of galois theory, this subfield is $\fixedField(H)$ Let $\beta \in \Q(\alpha) = \Q(\sqrt{2}, \sqrt{5})$ be arbitrary. Because $\{1, \sqrt{2}, \sqrt{5}, \sqrt{2} \sqrt{5}\}$ is a basis for $\Q(\alpha)$ as a $\Q$-vector space, there exist unique $a_{i, j} \in \Q$ such that
\begin{align*}
\beta
=
a_{0, 0} 1 + a_{0, 1} \sqrt{5} + a_{1, 0} \sqrt{2} + a_{1, 1} \sqrt{2} \sqrt{5}
\end{align*}
By definition, $\beta \in \fixedField(H)$ if and only if for all $\sigma \in H$, $\sigma$ fixes $\beta$: $\sigma(\beta) = \beta$. This equation is satisfied for the identity map $\id_{\Q(\sqrt{2}, \sqrt{5})}$. The only other element of $H$ is $\tau_{2} \tau_{5}$. Using the fact that elements of the galois group are field homomorphisms that fix each element of the base field (here, $\Q$), we compute
\begin{align*}
\tau_{2} \tau_{5}(\beta)
&=
\tau_{2} \tau_{5}(a_{0, 0} 1 + a_{0, 1} \sqrt{5} + a_{1, 0} \sqrt{2} + a_{1, 1} \sqrt{2} \sqrt{5})
\\
&=
a_{0, 0} \cdot \tau_{2} \tau_{5}(1) + a_{0, 1} \cdot \tau_{2} \tau_{5}(\sqrt{5}) + a_{1, 0} \cdot \tau_{2} \tau_{5}(\sqrt{2}) + a_{1, 1} \cdot \tau_{2} \tau_{5}(\sqrt{2} \sqrt{5})
\\
&=
a_{0, 0} 1 - a_{0, 1} \sqrt{5} - a_{1, 0} \sqrt{2} + a_{1, 1} \sqrt{2} \sqrt{5}
\end{align*}
Thus $\beta \in \fixedField(H) = \fixedField(\{\id_{\Q(\sqrt{2}, \sqrt{5})}, \tau_{2} \tau_{5}\})$ if and only if
\begin{align*}
&\tau_{2} \tau_{5}(\beta)
=
\beta
&
&\Leftrightarrow
&
&a_{0, 1} = 0 \text{ and } a_{1, 0} = 0
&
&\Leftrightarrow
&
&\beta \in \Span_{\Q}(1, \sqrt{2} \sqrt{5}) = \Q(\sqrt{10})
\end{align*}

\end{document}