%
% LQ01B : 2024--03--22 (F)
%

% Gallian 7e Exercise 12.50
\begin{enumerate}[label=(\alph*)]
\item\label{itm : LQ01Ba} Let $R$ be a ring with a multiplicative identity $1_{R}$. $R$ is \fontDefWord{boolean} if, for all $a \in R$, $a^{2} = a$. Prove that a boolean ring is commutative.
\end{enumerate}

\spaceSolution{3in}{% Begin solution.
Note that the ring axioms imply that $-1_{R} \in R$, and that it satisfies
\begin{align}
-1_{R}
=
(-1_{R})^{2}
=
1_{R}%
\label{eq : LQ01B boolean ring -1}
\end{align}
where for the first equality we have used the hypothesis that $R$ is boolean.

Case 1: $1_{R} = 0_{R}$. In this case, $R$ is the zero ring, which satisfies the axioms of a boolean ring. There is only one element $a \in R$ (namely $0_{R}$), and any element commutes with itself. Hence $R$ is commutative.

Case 2: $1_{R} \neq 0_{R}$. Let $a, b \in R$ be arbitrary, and consider the element $a + b \in R$. Then
\begin{align*}
a + b
=
(a + b)^{2}
=
a^{2} + a b + b a + b^{2}
=
a + a b + b a + b
\end{align*}
where in the first equality, we use the hypothesis that $R$ is boolean; in the second, the left- and right-distributivity axioms of a ring; and in the third, the hypothesis that $R$ is boolean. Because $R$ is a ring, by definition it contains the additive inverses $-a$ and $-b$. Adding these to both sides of the last equation, we get
\begin{align*}
a b + b a
&=
0
&
&\Leftrightarrow
&
a b
&=
-b a
\end{align*}
Applying the ring axioms and equation \eqref{eq : LQ01B boolean ring -1} to this last equality, we conclude that
\begin{align*}
a b
=
-b a
=
-1_{R} (b a)
=
1_{R} (b a)
=
b a
\end{align*}
as desired.}% End solution.



% Gallian 7e Exercise 14.16
\begin{enumerate}[resume, label=(\alph*)]
\item\label{itm : LQ01Bb} Let $R$ be a ring; and let $I_{1}, I_{2}$ be ideals of $R$. Recall that
\begin{align*}
I_{1} + I_{2}
&=
\{a_{1} + a_{2} \st a_{i} \in I_{i}\}
&
I_{1} I_{2}
&=
\left\{\sum_{j = 1}^{n} a_{1, j} a_{2, j} \st n \in \integersPositive; \forall j, a_{i, j} \in I_{i}\right\}
\end{align*}
are ideals. (In particular, note that $I_{1} I_{2}$ comprises all finite sums of terms of the form $a_{1} a_{2}$ with $a_{i} \in I_{i}$.) Prove that if $R$ is a commutative ring with a multiplicative identity, and if $I_{1} + I_{2} = R$, then $I_{1} \cap I_{2} = I_{1} I_{2}$.
\end{enumerate}

\spaceSolution{4in}{% Begin solution.
The inclusion $I_{1} I_{2} \subseteq I_{1} \cap I_{2}$ holds for arbitrary rings, by definition of an ideal (namely, closure under addition and ``strong closure'' under multiplication by ring elements). To show the reverse inclusion, $I_{1} \cap I_{2} \subseteq I_{1} I_{2}$, let $a \in I_{1} \cap I_{2}$. By hypothesis, $I_{1} + I_{2} = R$, and there exists a multiplicative identity element $1_{R} \in R$; so in particular, there exist $e_{1} \in I_{1}$ and $e_{2} \in I_{2}$ such that $e_{1} + e_{2} = 1_{R}$. Then
\begin{align*}
a
=
a 1_{R}
=
a (e_{1} + e_{2})
=
a e_{1} + a e_{2}
=
e_{1} a + a e_{2}
\end{align*}
where in the last equality we have used the hypothesis that $R$ is commutative. The last expression is a sum of elements, each with the form (element of $I_{1}$) times (element of $I_{2}$). Thus $a \in I_{1} I_{2}$, as desired.}% End solution.