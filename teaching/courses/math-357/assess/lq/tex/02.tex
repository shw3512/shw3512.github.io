%
% LQ02 : 2024--01--29 (M)
%

\noindent{}Consider $\R[x, y]$, the polynomial ring in two indeterminates $x, y$ whose ring of coefficients is the field $\R$ of real numbers. Let $f, g \in \R[x, y]$ be the polynomials
\begin{align*}
f(x, y)
&=
x^{2} y - x y - x y^{3}
&
g(x, y)
&=
x^{2} - x y - 2 y^{2}
\end{align*}

\begin{enumerate}[label=(\alph*)]
\item For each polynomial, state its (total) degree and its number of homogeneous components.
\end{enumerate}

\spaceSolution{2in}{% Begin solution.
By definition, the (total) degree of a nonzero polynomial is the maximum of the (total) degrees of its nonzero monomial terms. Let ``\# h.c.'' denote ``number of homogeneous components''. We have
\begin{center}
\begin{tabular}{l|c c}
		&	deg	&	\# h.c.	\\
\hline
$f$	&	$4$	&	$3$	\\
$g$	&	$2$	&	$1$
\end{tabular}
\end{center}
In particular, each nonzero monomial term of $f$ is a different homogenous component, because each term has a different (total) degree; and $g$ has a single nonzero homogeneous component, and is therefore homogeneous.}% End solution.



\begin{enumerate}[resume, label=(\alph*)]
\item\label{itm : LQ02b} Consider the following statement: ``If a polynomial is homogeneous, then the zeros of the induced function are well defined on lines through the origin.'' Use the polynomials $f$ and $g$ to explain this statement. \fontHint{What is $\{\lambda (x_{0}, y_{0}) \st \lambda \in \R\}$?}
\end{enumerate}

\spaceSolution{3in}{% Begin solution.
As noted above, the polynomial $g$ is homogeneous, whereas $f$ is not. We observe that $(2,1)$ is a zero of $g$:
\begin{align*}
g(2, 1)
=
4 - 2 - 2
=
0
\end{align*}
Let $\ell_{(2,1)}$ denote the line in $\R^{2}$ passing through the origin and the point $(2,1)$. Then
\begin{align*}
\ell_{(2,1)}
=
\{\lambda (2,1) \st \lambda \in \R\}
\end{align*}
In particular, for every point $P \in \ell_{(2,1)}$, there exists a $\lambda \in \R$ such that $P = \lambda (2,1)$. Evaluating $g$ at $P$, we find
\begin{align*}
g(P)
=
g(2 \lambda, \lambda)
=
4 \lambda^{2} - 2 \lambda^{2} - 2 \lambda^{2}
=
\lambda^{2} (4 - 2 - 2)
=
0
\end{align*}
Because $P \in \ell_{(2,1)}$ was arbitrary, we conclude that the polynomial $g$ evaluates to $0$ on the line $\ell_{(2,1)}$. That is, if a homogeneous polynomial evaluates to zero at a nonzero point $P$, then it evaluates to zero at any point on the line through $P$ and the origin.}% End solution.



\begin{enumerate}[resume, label=(\alph*)]
\item Make a conjecture.
\end{enumerate}

\spaceSolution{1in}{% Begin solution.
We observe that $(2,1)$ is also a zero of the polynomial $f$:
\begin{align*}
f(2,1)
=
4 - 2 - 2
=
0
\end{align*}
The point $(-2, -1) = -1 (2, 1)$ is on the line through $(2, 1)$ and the origin. We compute
\begin{align*}
f(-1 (2,1))
=
f(-2, -1)
=
-4 - 2 - 2
=
-8
\neq
0
\end{align*}
Perhaps we could strengthen the statement in part \ref{itm : LQ02b}? Conjecture: Let $R$ be an integral domain, let $n \in \integersPositive$, and let $t_{i}$ be indeterminates. A polynomial $f \in R[t_{1}, \ldots, t_{n}]$ is homogeneous if and only if the zeros of the induced function $f : R^{n} \rightarrow R$ are well defined on lines through the origin. (Do we need the ring $R$ of coefficients to be an integral domain? to be a field? to be infinite?)

One might also consider fractions of homogeneous polynomials. Conjecture: Let $R$, $n$, and $t_{i}$ be as above, and let $f, g \in R[t_{1}, \ldots, t_{n}]$ be nonzero homogeneous polynomials of the same degree. Let $\zeros(g)$ denote the set of zeros of $g$:
\begin{align*}
\zeros(g)
=
\{a \in R^{n} \st g(a) = 0\}
\end{align*}
Then $f / g$ is well defined as a function $R^{n} - \zeros(g) \rightarrow R$.}% End solution.