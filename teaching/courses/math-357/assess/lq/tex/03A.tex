%
% LQ03A : 2024--02--05 (M)
%

\noindent{}Let $R$ be an integral domain, and let $t$ be an indeterminate. Consider the polynomial ring $R[t]$.

\begin{enumerate}[label=(\alph*)]
\item\label{itm : LQ03A Iso} Prove that $(R[t])^{\times} \iso R^{\times}$. That is, we may view the units of $R[t]$ to be exactly the units of $R$. \fontHint{$\deg(p q) = \ldots$}
\end{enumerate}

\spaceSolution{3in}{% Begin solution.
For the isomorphism (of $R$ as a subring of $R[t]$ more generally), identify each $r \in R$ with the constant polynomial $r \in R[t]$. This identification allows us to view the given isomorphism as an equality $(R[t])^{\times} = R^{\times}$. We prove set containment in both directions.% Begin footnote.
\footnote{The isomorphism as groups follows from the fact that multiplication on $R[t]$ is defined, in part, using the multiplication on $R$.}% End footnote.

($(R[t])^{\times} \supseteq R^{\times}$) Immediate. (Why?)

($(R[t])^{\times} \subseteq R^{\times}$) We have seen that if $R$ is an integral domain, then for all $p, q \in R[t]$,
\begin{align}
\deg(p q)
=
\deg p + \deg q%
\label{eq : LQ03A Degree Relation}
\end{align}
Let $p \in (R[t])^{\times}$. By definition of unit, there exists a $q \in R[t]$ such that
\begin{align*}
p q
=
1
\end{align*}
Applying the degree function to this equation and using \eqref{eq : LQ03A Degree Relation}, we have
\begin{align}
0
=
\deg 1
=
\deg(p q)
=
\deg p + \deg q%
\label{eq : LQ03A Degree Units}
\end{align}
Because $p, q \in (R[t])^{\times}$, it follows that $p, q \neq 0$, and therefore $\deg p, \deg q \geq 0$. Hence \eqref{eq : LQ03A Degree Units} implies that $\deg p, \deg q = 0$; that is, $p$ and $q$ are constant polynomials; that is, $p, q \in R$. Thus $(R[t])^{\times} \subseteq R^{\times}$, as desired.}% End solution.



\begin{enumerate}[resume, label=(\alph*)]
\item Now let $R$ be a commutative ring with a $1 \neq 0$. Give an example to show that the isomorphism in part \ref{itm : LQ03A Iso} can fail.
\end{enumerate}

\spaceSolution{3in}{% Begin solution.
Let $R = \Z / 2 \Z$, and let $f = 2 t + 1 \in R[t]$. We compute
\begin{align*}
f^{2}
=
(2 t + 1)^{2}
=
4 t^{2} + 4 t + 1
\equiv
1
\end{align*}
because $4 \equiv 0$ in the ring $\Z / 4 \Z$ of coefficients. Thus $f \in (R[t])^{\times}$. However, $f$ does not correspond to a unit in $R$, whose two units $1, 3$ correspond to the constant polynomials (with the same constant terms) in $R[t]$.

In fact, for each $n \in \integersNonnegative$, we may define
\begin{align*}
f_{n}
&=
2 t^{n} + 1
\end{align*}
An analogous computation to the one above shows that $f_{n} \in (R[t])^{\times}$. Thus $(R[t])^{\times}$ has infinitely many elements, whereas $R^{\times} = \{1, 3\}$ has only two. Thus they cannot even be isomorphic as sets.}% End solution.