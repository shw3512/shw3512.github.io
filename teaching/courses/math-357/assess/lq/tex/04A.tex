%
% LQ04A : 2024--02--19 (M)
%

\noindent{}Let $\Q$ denote the field of rational numbers; given a prime $p \in \integersPositive$, let $\F_{p} \isomorphism \Z / (p)$ denote the finite field with $p$ elements; and let $t$ be an indeterminate. For each of the quotient rings below, characterize its algebraic structure as ``field'', ``integral domain but not field'', or ``ring but not integral domain''. Justify your characterization.
\begin{align*}
R_{1}
&=
\F_{2}[t] / (t^{4} + t^{2} + 1)
&
R_{2}
&=
\Q[t] / (t^{3} + t^{2} - t + 1)
&
R_{3}
&=
\Q[t] / (3 t^{3} + 4 t^{2} + 2 t - 4)
\end{align*}
\fontHint{If you feel inclined to do a lot of computation, then I invite you to first check with me.}

\spaceSolution{5in}{% Begin solution.
We analyze each quotient ring in turn.

$R_{1}$ : Ring but not integral domain. Let $f_{1} = t^{4} + t^{2} + 1 \in \F_{2}[t]$. If $f_{1}$ is reducible, then it has a factor of degree $1$ or degree $2$. The polynomial $f_{1} \in \F_{2}[t]$ has a factor of degree $1$ if and only if the function $f_{1} : \F_{2} \rightarrow \F_{2}$ has a zero, which direct computation shows is not the case. There are only four polynomials in $\F_{2}[t]$ with degree $2$ (why?), and three of them factor into linear factors, which we checked for in the previous case. Thus it remains only to check whether
\begin{align*}
f_{1}
=
(t^{2} + t + 1)^{2}
\end{align*}
which direct computation shows is a valid equation. Thus $f_{1} \in \F_{2}[t]$ is reducible, so $\F_{2}[t] / (f_{1})$ is a ring but not an integral domain.

$R_{2}$ : Field. Let $f_{2} = t^{3} + t^{2} - t + 1 \in \Z[t]$. If we apply the reduction homomorphism corresponding to the ideal $(3) \ideal \Z$ to $f_{2}$, then we get the polynomial $\overline{f}_{2} \in \F_{3}[t]$, which we may express with the same coefficients as $f_{2}$ (viewed in $\F_{3}$, rather than in $\Z$ or $\Q$). Because $\deg \overline{f}_{2} = 3$, the polynomial $\overline{f}_{2} \in \F_{3}[t]$ is reducible if and only if the function $\overline{f}_{2} : \F_{3} \rightarrow \F_{3}$ has a zero. Direct computation shows that for all $\alpha \in \F_{3}$, $\overline{f}_{2}(\alpha) \neq 0$, so $\overline{f}_{2}$ is irreducible. Because $f_{2}$ is nonconstant and monic, and $(3) \ideal \Z$ is proper, this implies that $f_{2} \in \Z[t]$ is irreducible. Thus Gau\ss{}'s lemma implies that $f_{2} \in \Q[t]$ is irreducible. Hence $\Q[t] / (f_{2})$ is a field.

Note that we cannot apply the Eisenstein--Sch\"{o}nemann criterion directly to $f_{2}$ (why not?), nor may we (correctly) argue that $f_{2}$ must have a zero because $\deg f_{2}$ is odd (why not?).

$R_{3}$ : Ring but not integral domain. Let $f_{3} = 3 t^{3} + 4 t^{2} + 2 t - 4 \in \Z[t]$. Because $\deg f_{3} = 3$, the polynomial $f_{3} \in \Q[t]$ is reducible if and only if the function $f_{3} : \Q \rightarrow \Q$ has a zero. If $\frac{a}{b} \in \Q$ is a zero of $f_{3}$, and if $\gcd(a, b) = 1$, then we have seen that, in $\Z$, $a \divides {-4}$ (the constant term of $f_{3}$) and $b \divides 3$ (the leading coefficient of $f_{3}$). We have finitely many possibilities for $\frac{a}{b}$---$24$ in this case (why?)---but we may reduce our work significantly if we view $f_{3} \in \R[t]$ and note that
\begin{align*}
f_{3}(0) &= -4 < 0
&
f_{3}(1) &= 5 > 0
\end{align*}
Because the function induced by a polynomial is continuous, the intermediate value theorem implies that $f_{3}$ has a zero (in $\R$) on the interval $(0,1)$. The only values of $\frac{a}{b}$ in this interval, and consistent with the divisibility requirements on $a$ and $b$, are $\frac{1}{3}$ and $\frac{2}{3}$. Checking these possibilities, we find $f_{3}(\frac{2}{3}) = 0$. Hence $f_{3}$ is reducible in $\Q[t]$, so $\Q[t] / (f_{3})$ is a ring but not an integral domain.

Note that we cannot apply the Eisenstein--Sch\"{o}nemann criterion directly to $f_{3}$ (why not?).}% End solution.