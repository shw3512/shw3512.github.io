%
% LQ05 : 2024--03--25 (M)
%

\noindent{}Let $p = t^{3} - 3 t - 1 \in \Q[t]$.
\begin{enumerate}[label=(\alph*)]
\item\label{itm : LQ05a} Prove that $p$ is irreducible.
\item\label{itm : LQ05b} Prove that $p$ has three distinct zeros in $\R$. (You need not compute them.)
\item\label{itm : LQ05c} Let $\alpha \in \R$ be a zero of $p$. Prove that $\sqrt{2} \notin \Q(\alpha)$.
\end{enumerate}

\spaceSolution{4in}{% Begin solution.
Part \ref{itm : LQ05a}: Because $p$ is a degree-$3$ polynomial over $\Q$, a field, $p$ is irreducible in $\Q[t]$ if and only if it has no zeros in $\Q$.% Begin footnote.
\footnote{See DF3e, Proposition 9.10, p 308.} % End footnote.
We may show that $p$ has no zeros in $\Q$ in various ways.
\begin{enumerate}
\item By the rational roots test,% Begin footnote.
\footnote{See DF3e, Proposition 9.11, p 308.} % End footnote.
if $\frac{r}{s} \in \Q$ is a zero of $p$ and in lowest terms, then $r$ divides the constant term of $p$ and $s$ divides the leading coefficient of $p$. This implies that if $\alpha \in \Q$ is a zero of $p$, then $\alpha = \pm{}1$. Evaluating the function $p$ at these two values, we find
\begin{align*}
p(-1)
&=
1
&
p(1)
&=
-3
\end{align*}
so $p$ has no zeros in $\Q$, and hence is irreducible in $\Q[t]$.
\item Viewing $p \in \Z[t]$ and reducing its coefficients modulo $2$, we get
\begin{align*}
\overline{p}
=
t^{3} + t + 1
\end{align*}
Evaluating $\overline{p}$ at the elements of the (finite) field $\Z / (2)$, we find
\begin{align*}
\overline{p}(0)
&=
1
&
\overline{p}(1)
&=
3
\equiv
1
\end{align*}
so $p$ has no zeros in $\Z / (2)$, and hence is irreducible in $(\Z / (2))[t]$, hence in $\Z[t]$,% Begin footnote.
\footnote{See DF3e, Proposition 9.12, p 309.} % End footnote.
hence (by Gau\ss{}'s lemma% Begin footnote.
\footnote{See DF3e, Proposition 9.5, p 303.}% End footnote.
) in $\Q[t]$.
\item Evaluating $p$ at $t + 1$, we get
\begin{align*}
p(t + 1)
&=
t^{3} + 3 t^{2} - 3
\end{align*}
which by the Eisenstein--Sch\"{o}nemann criterion% Begin footnote.
\footnote{See DF3e, Proposition 9.13, p 309.} % End footnote.
with $p = 3$ is irreducible in $\Z[t]$ and hence (by Gau\ss{}'s lemma) in $\Q[t]$.
\end{enumerate}

Part \ref{itm : LQ05b}: Evaluating $p$ at a few small values of $t \in \Z$, we find% Begin footnote.
\footnote{In settings where we have access to graphing applications, we might use these to guide our search.}% End footnote.
\begin{center}
\begin{tabular}{l|r r r r r}
\hline\hline
$t$		&	$-2$	&	$-1$	&	$0$	&	$1$	&	$2$	\\
\hline
$p(t)$	&	$-3$	&	$1$	&	$-1$	&	$-3$	&	$1$	\\
\hline
\end{tabular}
\end{center}
In particular, $p(t)$ changes sign three times as $t$ increases over the interval $[-2, 2]$. If we view the polynomial $p \in \Q[t]$ as a function $\R \rightarrow \R$, then the intermediate value theorem implies that $p$ has three zeros in $\R$.

Part \ref{itm : LQ05c}: To begin, note that
\begin{enumerate}
\item The minimal polynomial of $\alpha$ over $\Q$ is $p$, because $p$ is monic, irreducible, and $p(\alpha) = 0$.% Begin footnote.
\footnote{See DF3e, Proposition 13.9, p 520, and the subsequent discussion.} % End footnote.
Thus
\begin{align*}
[\Q(\alpha) : \Q]
=
\deg p
=
3
\end{align*}
\item The minimal polynomial of $\sqrt{2}$ over $\Q$ is $m_{\sqrt{2}, \Q} = t^{2} - 2$. Thus
\begin{align*}
[\Q(\sqrt{2}) : \Q]
=
\deg m_{\sqrt{2}, \Q}
=
2
\end{align*}
\end{enumerate}
Let $K : \Q$ be a field extension such that $\sqrt{2} \in K$. Then $\Q(\sqrt{2})$ is an intermediate field of $K : \Q$, so we get a tower of field extensions:
\begin{align*}
K : \Q(\sqrt{2}) : \Q
\end{align*}
By the tower law, the degrees of these extensions satsify% Begin footnote.
\footnote{See DF3e, Theorem 13.14, p 523.}% End footnote.
\begin{align*}
[K : \Q]
=
[K : \Q(\sqrt{2})] [\Q(\sqrt{2}) : \Q]
\end{align*}
In particular, $[\Q(\sqrt{2}) : \Q] = 2$ divides $[K : \Q]$. That is, if $\sqrt{2} \in K$, then $2 \divides [K : \Q]$. Equivalently, if $2 \notDivides [K : \Q]$, then $\sqrt{2} \notin K$. Because $[\Q(\alpha) : \Q] = 3$ is not divisible by $2$, we conclude that $\sqrt{2} \notin \Q(\alpha)$.}% End solution.