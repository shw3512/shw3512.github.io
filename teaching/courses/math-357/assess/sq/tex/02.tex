%
% SQ02 : 2024--01--12 (F)
%

\hspace{-0.25in}You join your friends for a meal. You mention that you're studying ring theory.

\begin{enumerate}[label=(\alph*)]
\item One (particularly opinionated) friend advises, ``You don't need to study rings. Just work with abelian groups and define multiplication to always be zero.'' What do you respond? In particular, is what your friend proposes a ring?
\end{enumerate}

\spaceSolution{1.25in}{It is straightforward to check that the proposed structure indeed satisfies the ring axioms. Such rings are sometimes called \fontDefWord{trivial rings}, as the multiplication operation does not add any new results beyond those of abelian groups. I thank my friend for the unsolicited advice, which led to this perhaps surprising conclusion, and proceed peacefully with my meal.}



\begin{enumerate}[resume, label=(\alph*)]
\item Another friend muses aloud, ``Could the structure just proposed---be it a ring or something else---have a multiplicative identity?'' What do you respond?
\end{enumerate}

\spaceSolution{1.25in}{Yes, it could, but only if $1 = 0$, in which case we necessarily have the zero ring. (What contradiction arises if we assume a multiplicative identity $1 \neq 0$?) I invite both friends to join me in my study of ring theory before gracefully turning our mealtime conversation to other topics.}