%
% SQ03 : 2024--01--22 (M)
%

\begin{enumerate}[label=(\alph*)]
\item Let $(R, +, \times)$ be a commutative (!) ring. Define what it means for a subset $I \subseteq R$ to be an ideal. Define what it means for an ideal to be principal.
\end{enumerate}

\spaceSolution{1.5in}{% Begin solution.
A subset $I \subseteq R$ is an \fontDefWord{ideal} of $R$ if the following two conditions are satisfied:
\begin{enumerate}[label=(\roman*)]
\item (subring) $(I, +|_{I}, \times|_{I})$ is a subring of $(R, +, \times)$.
\item (strongly closed under $\times_{R}$) For all $r \in R$, for all $a \in I$, $r a \in I$.
\end{enumerate}
An ideal $I$ is \fontDefWord{principal} if it is generated by a single element; that is, there exists an $a \in R$ such that $I = (a) = \{\sum_{i = 1}^{n} r_{i} a_{i} \st n \in \integersPositive; \forall i, r_{i} \in R, a_{i} \in I\}$. For commutative rings, this is equivalent to saying that there exist an $a \in I$ such that for all $b \in I$, there exists an $r \in R$ such that $b = r a$.% Begin footnote.
\footnote{Can you prove this? On which ring axioms does this rely?} % End footnote.
That is, for commutative rings, all elements of a principal ideal are $R$-multiples of a generator.}% End solution.



\begin{enumerate}[label=(\alph*)]
\setcounter{enumi}{1}
\item Are the following ideals principal? Answer both; briefly justify at least one.
\begin{align*}
\text{$(2, t)$ as an ideal of $\integers[t]$}
\hspace{1.25in}
\text{$(2, t)$ as an ideal of $\rationals[t]$}
\end{align*}
\end{enumerate}

\spaceSolution{1in}{% Begin solution.
The ideal $(2, t)$ in the ring $\integers[t]$ is not principal. We prove this in our responses to Expositional Homework 01.% Begin footnote.
\footnote{Alternatively, see Dummit \&{} Foote, 3e, p 252, Example (3).} % End footnote.
The ideal $(2, t)$ in the ring $\rationals[t]$ is principal. We can see this as follows: $2 \in (2, t)$; therefore by strong closure, $1 = \frac{1}{2} \times 2 \in (2, t)$; therefore again by strong closure, for all $r \in \rationals[t]$, $r \times 1 \in (2, t)$. That is, $(2, t) = (1)$, which is all of $\rationals[t]$.}% End solution.