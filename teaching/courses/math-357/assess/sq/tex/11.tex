%
% SQ11 : 2024--03--20 (W)
%

\begin{enumerate}[label=(\alph*)]
\item Define the minimal polynomial of a field element. Briefly but clearly introduce any hypotheses and auxiliary objects that you use in your definition.
\end{enumerate}

\spaceSolution{1.25in}{% Begin solution.
Let $K : K_{0}$ be a field extension, and let $\alpha \in K$ be algebraic over $K_{0}$. The \fontDefWord{minimal polynomial} of $\alpha$ over $K_{0}$ is the (unique) monic, irreducible polynomial $p \in K_{0}[t]$ such that $p(\alpha) = 0$.% Begin footnote.
\footnote{We may view the ring (field) operations $+$ and $\times$ used to evaluate $p$ at $\alpha$ as being those of $K$.} % End footnote.
We will denote this minimal polynomial by $m_{\alpha, K_{0}}$.

Note that the hypothesis that $\alpha$ be algebraic over $K_{0}$ is important. If $\alpha$ is transcendental (not algebraic) over $K_{0}$, then by definition there is no polynomial $p \in K_{0}[t]$ such that $p(\alpha) = 0$, let alone a monic, irreducible one.}% End solution.

\begin{enumerate}[resume, label=(\alph*)]
\item Let $\alpha = \sqrt{-\sqrt{2}} \in \C$. State the minimal polynomial of $\alpha$ over $\Q$, over $\R$, and over $\C$.
\end{enumerate}

\spaceSolution{1.25in}{% Begin solution.
We may manipulate the defining equation for $\alpha$ in an attempt to get a ``polynomial'' relation involving powers of $\alpha$ and coefficients in the desired base field. If we are successful, then we may replace $\alpha$ with $t$ to get a polynomial with $\alpha$ as a zero. (We still must certify that the polynomial is monic and irreducible.) Playing this game, we get candidates
\begin{align*}
\alpha
&=
\sqrt{-\sqrt{2}}
&
&\Leftrightarrow
&
\alpha - \sqrt{-\sqrt{2}}
&=
0
&
&\rightsquigarrow
&
m_{\alpha, \C}
&=
t - \sqrt{-\sqrt{2}}
\\
\alpha^{2}
&=
-\sqrt{2}
&
&\Leftrightarrow
&
\alpha^{2} + \sqrt{2}
&=
0
&
&\rightsquigarrow
&
m_{\alpha, \R}
&=
t^{2} + \sqrt{2}
\\
\alpha^{4}
&=
2
&
&\Leftrightarrow
&
\alpha^{4} - 2
&=
0
&
&\rightsquigarrow
&
m_{\alpha, \Q}
&=
t^{4} - 2
\end{align*}
All these polynomials are monic. As for irreducibility, note that
\begin{enumerate}
\item $m_{\alpha, \Q}$ is irreducible in $\Q[t]$ by the Eisenstein--Sch\"{o}nemann criterion with $p = 2$ and Gau\ss{}'s lemma.% Begin footnote.
\footnote{See DF3e, Proposition 9.13, p 309 and Proposition 9.5, p 303, respectively.}% End footnote.
\item $m_{\alpha, \R}$ is irreducible in $\R[t]$ because, if it factored, it would do so into two degree-1 factors, which would imply that $\alpha = \sqrt{-2}$ is in $\R$, a contradiction.% Begin footnote.
\footnote{See DF3e, Proposition 9.10, p 308.}% End footnote.
\item $m_{\alpha, \C}$ is irreducible in $\C[t]$ because all degree-1 polynomials over a field are irreducible.% Begin footnote.
\footnote{Note that the hypothesis that the ring of coefficients is a field is important. For example, the degree-1 polynomial $2 t - 2$ is reducible in $\Z[t]$ (why?).}% End footnote.
\end{enumerate}}% End solution.