\begin{center}
{\Large Syllabus}\\
{\scriptsize (last updated : \Year--\Month--\Day)}
\end{center}



\begin{quote}
The information contained in this syllabus, other than the absence policy, is subject to change with reasonable advance notice.
\end{quote}





%
%
%		Course information
%
%

\section{Course information}



\subsection{Course logistics}

\begin{tabular}{r c l}
Class			&	:	&	MWF 09h00--09h50, Herman Brown Hall (HBH) 427	\\
Office hours	&	:	&	M 13h30--15h30, HBH 330	\\
Recitation		&	:	&	F 15h00--15h50, HBH 427
\end{tabular}



\subsection{Instructor information}

\begin{tabular}{r c l}
Instructor	&	:	&	Stephen Wolff	\\
Office	&	:	&	HBH 330	\\
E-mail	&	:	&	\href{mailto:Stephen.Wolff@rice.edu?subject=[Math\%20357]}{Stephen.Wolff@rice.edu}
\end{tabular}
\hspace{0.25in}
\begin{tabular}{r c l}
TA		&	:	&	Zihao Liu	\\
%Office	&	:	&	TBD	\\% (!!!) uncomment, update
E-mail	&	:	&	\href{mailto:zl152@rice.edu?subject=[Math\%20357]}{zl152@rice.edu}	\\
			&		&% (!!!) delete this line if office information added above
\end{tabular}



\subsection{Prerequisites}

Math 356.



\subsection{Textbooks}

The content of this course draws heavily from the following book:
\begin{itemize}
	\item ``Abstract algebra'', 3rd edition, by David S. Dummit and Richard M. Foote
\end{itemize}
Course Reserves at Fondren Library will include a physical copy of this book, along with other relevant books that you may wish to explore.

%Course Reserves at Fondren Library will include a physical copy of this book, along with other relevant books that you may wish to explore, including:
%\begin{itemize}
%	\item ``Algebra'', 2nd edition, by Michael Artin
%	\item ``Contemporary abstract algebra'', any edition, by Joseph A. Gallian
%	\item ``Galois theory'', 3rd edition, by Ian Stewart
%\end{itemize}
%Previous instances of this course have used the following book, an e-book version of which is freely available (with optional donation) at \href{http://homepage.math.uiowa.edu/~goodman/algebrabook.dir/download.htm}{the author's website}:% Begin footnote.
%\footnote{\href{http://homepage.math.uiowa.edu/~goodman/algebrabook.dir/download.htm}{http://homepage.math.uiowa.edu/$\sim$goodman/algebrabook.dir/download.htm}}% End footnote.
%\begin{itemize}
%	\item ``Algebra: abstract and concrete'', edition 2.6, by Frederick M. Goodman
%	\begin{itemize}
%		\item Free e-book download with optional donation available at
%		\begin{quote}
%			{\small }
%		\end{quote}
%	\end{itemize}
%\end{itemize}



%\subsection{Other resources}



\subsection{Absence policy}

Class attendance is strongly encouraged. You are responsible for all material and announcements presented in class.





%
%
%		Course objectives
%
%

\newpage

\section{Course objectives and expected learning outcomes}

We will wrestle with the following topics (``$c.s$'' denotes chapter $c$, section $s$ from Dummit and Foote, 3rd edition):
\begin{center}
\begin{tabular}{l p{4.8in}}
\hline\hline
Sections		&	Topics									\\
\hline
7.1--6; 8.1--3	&	\textbf{Rings (review).} Definitions, EDs, PIDs, UFDs.	\\
9.1--5			&	\textbf{Polynomial rings.} Examples, irreducibility criteria.	\\
10.1--3			&	\textbf{Modules.} Morphisms, direct sums, free modules.	\\
18.1--3; 19.1	&	\textbf{Representation theory.} Reducibility, Schur's lemma, character theory.	\\
13.1--2, 4--6	&	\textbf{Field theory.} Field extensions, splitting fields, algebraic closure, separability, finite fields.	\\
14.1--4, 7		&	\textbf{Galois theory.} Galois groups, Galois extensions, fundamental theorem, finite fields.	\\
%12.1--3			&	Modules over PIDs (atp)		\\
%9.6				&	Gr\"{o}bner bases (atp)			\\
\hline
\end{tabular}
\end{center}
We may explore special topics as time permits.

My top goals for you in this course are the following:
\begin{itemize}
    \item Over the semester, you gain comfort and confidence contributing to your mathematical community by seeing, doing, and sharing mathematics.
    \item By the end of the semester, you understand and can apply the concepts we engage.
\end{itemize}





%
%
%		Grading policy
%
%

\section{Grading policy}

We will promote and assess the above goals using the following tools:
\begin{enumerate}
	\item \textbf{Homework.} Homework comes in two flavors: practice homework (PH) and expositional homework (EH). Practice homework comprises selected exercises, is due each Wednesday, and is graded 0--2 for reasonable completion. Expositional homework comprises specific exercises that we will discuss as a class, is due on a rolling basis (roughly every other week), and is graded 0--4 for accuracy and exposition. Homework may be typed or handwritten; in either case, it must be legible. Both kinds of homework are due on Gradescope by 9h00 on the day they are due. In case of technical issues, you may submit a physical copy in class by the same deadline. Late homework, of either kind, will not be accepted. Your lowest PH score will be dropped.
	\item \textbf{Quizzes.} Expect an in-class quiz every day. No external resources are allowed. Quizzes come in two flavors: short quiz (SQ) and long quiz (LQ). Short quizzes take about two minutes, focus on concepts, and are graded 0--2. You may not re-quiz short quizzes. Your lowest three short-quiz grades will be dropped. Long quizzes take about ten minutes, focus on exercises, and are graded 0--4. You may re-quiz long quizzes at office hours or recitation throughout the semester. Only your highest grade on each long-quiz thread will count toward your course grade. No long-quiz grades will be dropped. All quizzes will start solo. For some quizzes, I will explicitly invite you to discuss with your colleagues before you submit.
	\item \textbf{Exams.} We will have two midterm exams (ME) and one final exam (FE). All exams will be in-classroom, with no external resources allowed. The final exam will have two sections, with content corresponding to that of the two midterm exams. For each section of the final exam, if your score on that section is higher than your score on the corresponding midterm exam, then your final-exam score for that section will replace the corresponding midterm-exam score.
	
	The exams are tentatively (!) scheduled as follows:
	\begin{itemize}
	    \item Midterm exam 1: Monday 26 February
	    \item Midterm exam 2: Friday 12 April
	    \item Final exam: To be scheduled by the Registrar
	\end{itemize}
	Remarks.
	\begin{enumerate}
	    \item If an exam conflicts with a holiday you observe, then please let me know.
	    \item It is the policy of the Mathematics Department that no final exam may be given early to accommodate student travel plans. If you make travel plans that turn out to conflict with the scheduled exam, then it is your responsibility to either reschedule your travel plans or take a zero on the exam.
	\end{enumerate}
\end{enumerate}

\noindent{}\textbf{Course grade.}
Your course grade is allocated among the above tools as follows:
\begin{center}
{} \hfill{} PH : 10\% \hfill{} EH : 10\% \hfill{} SQ : 5\% \hfill{} LQ : 15\% \hfill{} ME : 15\% each \hfill{} FE : 30\% \hfill{} {}
\end{center}




%
%
%		Honor code
%
%

\section{Honor Code}

As a student at Rice University, you pledge to uphold the Rice Honor Code, which you can find in the \href{https://honor.rice.edu/honor-system-handbook/}{Honor System Handbook}.% Begin footnote.
\footnote{\href{https://honor.rice.edu/honor-system-handbook/}{https://honor.rice.edu/honor-system-handbook/}}% End footnote.

In this course,
\begin{itemize}
	\item On homework, all resources are allowed. In particular, you are strongly encouraged to work with one another. The purpose of homework is to help you to learn, share, practice, internalize concepts, and discover questions.
	\item On quizzes and exams, no external resources are allowed, unless the instructor explicitly states otherwise. The purpose of quizzes and exams is to help you to see what you can do so far, and identify what you want to work on.
\end{itemize}




%
%
%		Conduct
%
%

\section{Statement of conduct}

The Department of Mathematics supports an inclusive learning environment where diversity and individual differences are understood, respected, and recognized as a source of strength. Racism, discrimination, harassment, and bullying will not be tolerated. We expect all participants in mathematics courses---students and faculty alike---to treat one another with courtesy and respect, and to adhere to the \href{https://mathweb.rice.edu/department-statement-collegiality-respect-and-sensitivity}{Department of Mathematics' standards of collegiality, respect, and sensitivity},% Begin footnote.
\footnote{\label{ftnt : Math Dept Statement}\href{https://mathweb.rice.edu/department-statement-collegiality-respect-and-sensitivity}{https://mathweb.rice.edu/department-statement-collegiality-respect-and-sensitivity}} % End footnote.
as well as the \href{https://sjp.rice.edu/code-of-student-conduct#StandardOfConduct}{Rice Student Code of Conduct}.% Begin footnote.
\footnote{\href{https://sjp.rice.edu/code-of-student-conduct\#StandardOfConduct}{https://sjp.rice.edu/code-of-student-conduct\#StandardOfConduct}}% End footnote.

If you think you have experienced or witnessed unprofessional or antagonistic behavior, then we urge you to seek advice and support. Any member of the faculty or staff whom you trust is an appropriate contact person. The Ombudsperson% Begin footnote.
\footnote{The Ombudsperson's contact information is included in the Department of Mathematics' Statement on Collegiality, Respect, and Sensitivity, at the URL given in Footnote \ref{ftnt : Math Dept Statement}.} % End footnote.
is also available as an intermediate, informal opinion. When bringing concerns forward about discrimination or harassment, we will strive to treat these concerns with discretion, respecting the privacy of individuals insofar as possible, but in some cases policy or law may require that department members contact an appropriate university authority.




%
%
%		Disabilities
%
%

\section{Disabilities accommodations}

If you have a documented disability that may affect academic performance, then please (1) make sure this documentation is on file with the \href{https://drc.rice.edu/}{Rice Disability Resource Center}% Begin footnote.
\footnote{\href{https://drc.rice.edu/}{https://drc.rice.edu/}} % End footnote.
(\href{mailto:adarice@rice.edu}{adarice@rice.edu}; 713-348-5841; Allen Center, Room 111), in order to determine accommodations; and (2) contact me during the first two weeks of class to discuss your accommodations. All such discussions will remain as confidential as possible.

%Any student with a documented disability who needs academic accommodations is encouraged to contact both the course instructor (\href{mailto:Stephen.Wolff@rice.edu?subject=[Math\%20357]}{Stephen.Wolff@rice.edu}) and the \href{https://drc.rice.edu/}{Rice Disability Resource Center} (\href{mailto:adarice@rice.edu}{adarice@rice.edu}; 713-348-5841; Allen Center, Room 111).




%
%
%		Mental health
%
%

\section{Mental health resources}

The wellbeing and mental health of students is important. If you are having trouble, please reach out to the \href{https://wellbeing.rice.edu/}{Wellbeing and Counseling Center}% Begin footnote.
\footnote{\href{https://wellbeing.rice.edu/}{https://wellbeing.rice.edu/}} % End footnote.
(713-348-3311, available 24/7; located in the Gibbs Wellness Center). Rice University provides cost-free mental health services through the Wellbeing and Counseling Center to help you manage personal challenges that threaten your personal or academic well-being. If you believe you are experiencing unusual amounts of stress, sadness, or anxiety, the Student Wellbeing Office or the Rice Counseling Center may be able to assist you.




%
%
%		Religious accommodations
%
%

\section{Religious accommodations}

Every reasonable effort will be made to allow members of the university community to observe their religious holidays without jeopardizing the fulfillment of their academic obligations. Absence from classes or examinations for religious reasons does not relieve students from responsibility for any part of the course work required during the period of absence. It is the obligation of students to provide faculty with reasonable notice of the dates of religious holidays on which they will be absent. 




%
%
%		Title IX
%
%

\section{Title IX statement}

Rice University cares about your wellbeing and safety. Rice encourages any student who has experienced an incident of harassment; pregnancy or gender discrimination; or relationship, sexual, or other forms interpersonal violence to seek support through the SAFE Office. Please be aware, when seeking support on campus, that most employees (including myself, as an instructor) are required by Title IX to disclose all incidents of non-consensual interpersonal behaviors to Title IX professionals on campus, who can act to support students and meet their needs. For more information, please visit \href{https://safe.rice.edu/}{safe.rice.edu} or e-mail \href{mailto:titleixsupport@rice.edu}{titleixsupport@rice.edu}.