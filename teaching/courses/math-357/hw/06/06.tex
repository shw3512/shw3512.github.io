\documentclass[oneside, english, 11pt]{article}
\usepackage{geometry}
\geometry{letterpaper}
\geometry{left=1in, top=1in, right=1in, bottom=1in, nohead}
\setlength\parindent{0.25in}



\usepackage{amsmath}
\usepackage{amsthm}
\usepackage{enumitem}



% Sets fonts. Euler for math | Palatino for rm | Helvetica for ss | Courier for tt
% http://www.tug.org/mactex/fonts/LaTeX_Preamble-Font_Choices.html

\renewcommand{\rmdefault}{ppl}% rm
\linespread{1.05}% Palatino needs more leading
\usepackage[scaled]{helvet}% ss
\usepackage{courier}% tt
%\usepackage{euler}% math
\usepackage{eulervm}% math --- a better implementation of the euler package (not in gwTeX)
\normalfont
\usepackage[T1]{fontenc}



\newcommand{\fontDefWord}[1]{\textbf{#1}}
\newcommand{\fontField}[1]{\mathbf{#1}}
\newcommand{\fontHint}[1]{\emph{Hint:} #1}



\newcommand{\C}{\complexes}
\DeclareMathOperator{\characteristic}{char}
\newcommand{\complexes}{\fontField{C}}
\newcommand{\divides}{\,|\,}
\DeclareMathOperator{\Endomorphisms}{End}
\newcommand{\F}{\fontField{F}}
\DeclareMathOperator{\generalLinear}{GL}
\newcommand{\GL}{\generalLinear}
\DeclareMathOperator{\Hom}{Hom}
\DeclareMathOperator{\identityMap}{id}
\newcommand{\integers}{\fontField{Z}}
\newcommand{\integersPositive}{\integers_{> 0}}
\newcommand{\iso}{\isomorphism}
\newcommand{\isomorphism}{\cong}
\newcommand{\notDivides}{{}\hspace{-1mm}{\not\hspace{-.15em}\divides{}}\,}
\newcommand{\order}{\#}
\newcommand{\Q}{\rationals}
\newcommand{\rationals}{\fontField{Q}}
\newcommand{\st}{\, | \,}
\newcommand{\Z}{\integers}



\newtheorem{theorem}{Theorem}
\newtheorem{lemma}[theorem]{Lemma}



\title{Math 357\\Expositional homework 06}
\author{}
\date{Assigned: 2024--03--29 (F)\\\hspace{9.7mm}Due: 2024--04--12 (F)}



% Document begins here.

\begin{document}

\maketitle

The goal of this homework is to strengthen our understanding of field theory, through proof and example.

\paragraph{Proofs}

\begin{enumerate}[label=(\alph*)]
\item\label{itm : EH06a} Prove Proposition 13.12:% Begin footnote.
\footnote{See DF3e, p 521.} % End footnote.
Let $K : K_{0}$ be a field extension, and let $\alpha \in K$. Then $\alpha$ is algebraic over $K_{0}$ if and only if $[K_{0}(\alpha) : K_{0}] < \infty$.
\item\label{itm : EH06b} Prove Theorem 13.25% Begin footnote.
\footnote{See DF3e, p 536. If you are up for it, then you can also prove that any two splitting fields for $f$ are isomorphic; see Corollary 13.28, p 542. This will take a little work, but the theory and proof are both accessible and rewarding.} % End footnote.
on the existence of splitting fields: Let $K_{0}$ be a field, and let $f \in K_{0}[t]$. Then there exists an extension field $K : K_{0}$ such that $f$ splits completely in $K[t]$.
\item\label{itm : EH06c} Let $K : K_{0}$ be a field extension such that $[K : K_{0}] < \infty$. Prove that $K : K_{0}$ is normal if and only if for all irreducible polynomials $f \in K_{0}[t]$, if there exists an $\alpha \in K$ such that $f(\alpha) = 0$, then $f$ splits completely in $K[t]$.% Begin footnote.
\footnote{See DF3e, Exercise 13.4.5, p 545.} % End footnote.
(One may take this as the definition of a normal field extension, in which case one can prove the definition we gave in class as a proposition.)
\item\label{itm : EH06d} Let $K_{0}$ be a field, and let $f_{1}, f_{2} \in K_{0}[t]$. Prove that the formal derivative $D_{t}$ of a polynomial in $K_{0}[t]$ satisfies the following relations (as does the derivative operator from calculus):% Begin footnote.
\footnote{See DF3e, Exercise 13.5.1, p 551.}% End footnote.
\begin{align*}
D_{t}(f_{1} + f_{2})
&=
D_{t} f_{1} + D_{t} f_{2}
&
D_{t}(f_{1} f_{2})
&=
(D_{t} f_{1}) \cdot f_{2} + f_{1} \cdot (D_{t} f_{2})
\end{align*}
\end{enumerate}

\paragraph{Examples}

\begin{enumerate}[resume, label=(\alph*)]
\item\label{itm : EH06e} Determine the degree over $\Q$ of $2 + \sqrt{3}$ and $1 + \sqrt[3]{2} + \sqrt[3]{4}$.% Begin footnote.
\footnote{See DF3e, Exercise 13.2.4, p 530.}% End footnote.
\item\label{itm : EH06f} Let $K : K_{0}$ be a field extension of finite degree $n$, and let $\alpha \in K$.% Begin footnote.
\footnote{See DF3e, Exercises 13.2.19(a) and 20, p 531.}% End footnote.
\begin{enumerate}[label=(\roman*)]
\item\label{itm : EH06f1} Prove that the map
\begin{align*}
T_{\alpha}
:
K
&\rightarrow
K
\\
\beta
&\mapsto
\alpha \beta
\end{align*}
which is (left) multiplication by $\alpha$, is a $K_{0}$-linear transformation of $K$.
\end{enumerate}
Let $n \in \integersPositive$, let $M$ be an $n \times n$ matrix, let $I$ be the $n \times n$ identity matrix, and let $t$ be an indeterminate. The \fontDefWord{characteristic polynomial} of $M$ is $\det(t I - M) = (-1)^{n} \det(M - t I)$.
\begin{enumerate}[resume, label=(\roman*)]
\item\label{itm : EH06f2} Let $\mathcal{B}$ be a $K_{0}$-basis of $K$, and let $M_{\mathcal{B}}(T_{\alpha})$ be the matrix of $T_{\alpha}$ with respect to $\mathcal{B}$. Prove that $\alpha$ is a zero of the characteristic polynomial of $M_{\mathcal{B}}(T_{\alpha})$.
\item\label{itm : EH06f3} Use this technique to find monic polynomials in $\Q[t]$ of degree $3$ satisfied by $\sqrt[3]{2}$ and by $1 + \sqrt[3]{2} + \sqrt[3]{4}$.
\end{enumerate}
\item\label{itm : EH06g} Let $p \in \integersPositive$ be prime, let $\fontField{F}_{p} = \Z / (p)$ (a finite field of order $p$), let $a \in \fontField{F}_{p}$ be nonzero, and let $f = t^{p} - t + a \in \fontField{F}_{p}[t]$. Prove that $f$ is irreducible in $\fontField{F}_{p}[t]$ and separable.% Begin footnote.
\footnote{See DF3e, Exercise 13.5.5, p 551.}% End footnote.
\end{enumerate}



\subsection*{Exercise \ref{itm : EH06a}}

See DF3e, pp 521--2.



\subsection*{Exercise \ref{itm : EH06b}}

For existence, see DF3e, p 536. For uniqueness (which we were not required to prove on this homework), see DF3e, pp 541--2.



\subsection*{Exercise \ref{itm : EH06c}}

($\Rightarrow$) Let $K : K_{0}$ be normal. Then by definition, there exists a set $S \subseteq K_{0}[t]$ such that $K$ is a splitting field of $S$. By hypothesis, $[K : K_{0}] < \infty$, so there exists a finite subset $S' \subseteq S$ such that $K$ is a splitting field of $S'$. Let $f_{0} = \prod_{g \in S'} g$. (Note that this construction uses that $S'$ is finite.)

Let $f \in K_{0}[t]$ such that there exists an $\alpha \in K$ such that $f(\alpha) = 0_{K}$. Let $\beta$ be a zero of $f$, potentially in some extension field over $K$. (For example, view $f \in K[t]$, let $\tilde{K}_{f}$ be a splitting field for $f$ over $K$, and take $\beta \in \tilde{K}_{f}$.) By Theorem 13.8, there exists an isomorphism $\sigma_{0} : K_{0}(\alpha) \rightarrow K_{0}(\beta)$ such that $\sigma_{0}(\alpha) = \beta$ and $\sigma_{0}|_{K_{0}} = \identityMap_{K_{0}}$. Note that $K(\alpha)$ is a splitting field of $f_{0}$ over $K_{0}(\alpha)$, and $K(\beta)$ is a splitting field of $f_{0}$ over $K_{0}(\beta)$. Thus by Theorem 13.27, $\sigma_{0}$ extends to an isomorphism $\sigma : K(\alpha) \rightarrow K(\beta)$ such that $\sigma|_{K_{0}(\alpha)} = \sigma_{0}$. By hypothesis, $\alpha \in K$, so $K(\alpha) = K$. We conclude that $K(\beta) \isomorphism K(\alpha) = K$, so $\beta \in K$ as well.

($\Leftarrow$) By Theorem 13.17, $[K : K_{0}] < \infty$ if and only if $K$ is generated by a finite number, say $n$, of elements $\alpha_{i} \in K$, each of which is algebraic over $K_{0}$. Because $\alpha_{i}$ is algebraic over $K_{0}$, each $\alpha_{i}$ has a (unique) minimal polynomial $m_{\alpha_{i}, K_{0}} \in K_{0}[t]$, which by definition is irreducible. By definition, each $m_{\alpha_{i}, K_{0}}$ has a zero in $K$, namely $\alpha_{i}$; so by hypothesis, each $m_{\alpha_{i}, K_{0}}$ splits completely in $K$.

Let $f_{0} = \prod_{i = 1}^{n} m_{\alpha_{i}, K_{0}}$. (Again, note that this construction uses that the indexing set of the product is finite.) Our argument in the preceding paragraph shows that $f$ splits completely in $K$, and that $K$ is generated by the zeros of $f$. Thus $K : K_{0}$ is normal (take $S = \{f_{0}\}$).



\subsection*{Exercise \ref{itm : EH06d}}

This is a straightforward if tedious bookkeeping exercise.

Let $n_{i} = \deg f_{i}$, let $n = \max(n_{1}, n_{2})$, let $N = n_{1} + n_{2}$, and let
\begin{align*}
f_{i}
&=
\sum_{j = 0}^{n_{i}} a_{i, j} t^{j}
&
&\text{so}
&
D_{t} f_{i}
&=
\sum_{j = 1}^{n_{i}} j \cdot a_{i, j} t^{j - 1}
\end{align*}
Let us adopt the convention that if the index $j > n_{i}$, then $a_{i, j} = 0$. Then
\begin{align*}
D_{t}(f_{1} + f_{2})
&=
D_{t}\left(\sum_{j = 0}^{n} (a_{1, j} + a_{2, j}) t^{j}\right)
&
&\text{by definition of $+$ in a polynomial ring}
\\
&=
\sum_{j = 1}^{n} j \cdot (a_{1, j} + a_{2, j}) t^{j - 1}
&
&\text{by definition of formal derivative}
\\
&=
\sum_{j = 1}^{n} j \cdot a_{1, j} t^{j - 1} + \sum_{j = 1}^{n} j \cdot a_{2, j} t^{j - 1}
&
&\text{by field axioms}
\\
&=
D_{t} f_{1} + D_{t} f_{2}
&
&\text{by definition of formal derivative}
\end{align*}
Similarly,
\begin{align*}
D_{t}(f_{1} f_{2})
&=
D_{t}\left(\sum_{j = 0}^{N} \sum_{k = 0}^{j} a_{1, k} a_{2, j - k} t^{j}\right)
\\
&=
\sum_{j = 1}^{N} \sum_{k = 0}^{j} j \cdot a_{1, k} a_{2, j - k} t^{j - 1}
\\
&=
\sum_{j = 1}^{N} \sum_{k = 0}^{j} (k + j - k) \cdot a_{1, k} a_{2, j - k} t^{j - 1}
\\
&=
\sum_{j = 1}^{N} \sum_{k = 0}^{j} k \cdot a_{1, k} a_{2, j - k} t^{j - 1} + \sum_{j = 1}^{N} \sum_{k = 0}^{j} (j - k) \cdot a_{1, k} a_{2, j - k} t^{j - 1}
\\
&=
D_{t} f_{1} \cdot f_{2} + f_{1} \cdot D_{t} f_{2}
\end{align*}



\subsection*{Exercise \ref{itm : EH06e}}

Let $\alpha = 2 + \sqrt{3}$. Then
\begin{align*}\alpha - 2
&=
\sqrt{3}
&
&\Rightarrow
&
(\alpha - 2)^{2}
&=
3
&
&\Leftrightarrow
&
\alpha^{2} - 4 \alpha + 1
&=
0
\end{align*}
so $\alpha$ is a zero of the polynomial
\begin{align*}
f
=
t^{2} - 4 t + 1
\end{align*}
Note that $f$ is irreducible (why?% Begin footnote.
\footnote{One can see that $f$ is irreducible in various ways. One way is to reduce the coefficients mod $2$ and show that the image of $f$ has no zeros in $\Z / (2)$. Another way is to apply the rational zeros test to $f$ directly.}% End footnote.
) and monic, so it is the minimal polynomial of $\alpha$ over $\Q$. Hence
\begin{align*}
\deg_{\Q} \alpha
=
\deg m_{\alpha, \Q}
=
\deg f
=
2
\end{align*}

Now let $\alpha = 1 + \sqrt[3]{2} + \sqrt[3]{4}$. Then
\begin{align*}
(\alpha - 1)^{3}
&=
(\sqrt[3]{2} + \sqrt[3]{2}^{2})^{3}
=
2 + 3 \sqrt[3]{2}^{4} + 3 \sqrt[3]{2}^{5} + 4
=
6 (1 + \sqrt[3]{2} + \sqrt[3]{2}^{2})
=
6 \alpha
&
&\Leftrightarrow
&
(\alpha - 1)^{3} - 6 \alpha
&=
0
\end{align*}
so $\alpha$ is a zero of the polynomial
\begin{align*}
f
=
t^{3} - 3 t^{2} - 3 t - 1
\end{align*}
Note that $f$ is irreducible (why?% Begin footnote.
\footnote{The same approaches we took for $\alpha = 2 + \sqrt{3}$ above also work here.}% End footnote.
) and monic, so it is the minimal polynomial of $\alpha$ over $\Q$. Hence
\begin{align*}
\deg_{\Q} \alpha
=
\deg m_{\alpha, \Q}
=
\deg f
=
3
\end{align*}



\subsection*{Exercise \ref{itm : EH06f}}

Part \ref{itm : EH06f1}: Let $\beta_{1}, \beta_{2} \in K$, and let $a \in K_{0}$. Then
\begin{align*}
T_{\alpha}(a \beta_{1} + \beta_{2})
&=
\alpha (a \beta_{1} + \beta_{2})
&
&\text{by definition of $T_{\alpha}$}
\\
&=
a \alpha \beta_{1} + \alpha \beta_{2}
&
&\text{by the axioms of a field}
\\
&=
a T_{\alpha}(\beta_{1}) + T_{\alpha}(\beta_{2})
&
&\text{by definition of $T_{\alpha}$}
\end{align*}
so $T_{\alpha}$ is $K_{0}$-linear.



Part \ref{itm : EH06f2}: Let $M$ denote $M_{\mathcal{B}}(T_{\alpha})$; let $p = \det(t I - M)$ denote the characteristic polynomial of $M$; and let $\identityMap_{K}$ denote the identity map on $K$, so $I = M_{\mathcal{B}}(\identityMap_{K})$. The intuition is that the matrix $\alpha I - M$ that appears in the expression for $p(\alpha)$ represents the map $\alpha \identityMap_{K} - T_{\alpha}$, and for all $\beta \in K$,
\begin{align*}
(\alpha \identityMap_{K} - T_{\alpha})(\beta)
=
\alpha \identityMap_{K}(\beta) - T_{\alpha}(\beta)
=
\alpha \beta - \alpha \beta
=
0
\end{align*}
We may be tempted to apply a result from linear algebra% Begin footnote.
\footnote{See DF3e, Proposition 12.12, p 473.} % End footnote.
that says $\det(\alpha I - M) = 0$ if and only if $\alpha$ is an eigenvalue of the corresponding linear map. However, we must be careful to correctly apply any theory from linear algebra. In our current setting, we view $K$ as a $K_{0}$-vector space. In particular, $\alpha \in K$ corresponds to a vector, not a scalar. As such, $\alpha$ cannot be an eigenvalue. As illustrated in our response to part \ref{itm : EH06f3} below, the matrix $\alpha I - M$ is not the zero matrix if (and only if) $\alpha \notin K_{0}$. Instead, we could view the characteristic polynomial $p \in K_{0}[t]$ as being an element of $K[t]$, or we could view the $n \times n$ matrix $\alpha I - M$ as operating on $K^{n}$ and find an eigenvector there with eigenvalue $\alpha$.% (!!!)



Part \ref{itm : EH06f3}: Let
\begin{align*}
\alpha
&=
\sqrt[3]{2}
&
\beta
&=
1 + \sqrt[3]{2} + \sqrt[3]{4}
=
1 + \alpha + \alpha^{2}
\end{align*}
Note that $\alpha, \beta \in \Q(\alpha)$; and $m_{\alpha, \Q} = t^{3} - 2$ is the minimal polynomial for $\alpha$ over $\Q$ (why?), so $[\Q(\alpha) : \Q] = \deg m_{\alpha, \Q} = 3$. Therefore, one $\Q$-basis for $\Q(\alpha)$ is $\mathcal{B} = (1, \alpha, \alpha^{2})$. We compute the matrix representations with respect to this basis for the multiplication by $\alpha$ (respectively, $\beta$) map to be
\begin{align*}
M_{\mathcal{B}}(T_{\alpha})
&=
\begin{pmatrix}
0	&	0	&	2	\\
1	&	0	&	0	\\
0	&	1	&	0
\end{pmatrix}
&
M_{\mathcal{B}}(T_{\beta})
&=
\begin{pmatrix}
1	&	2	&	2	\\
1	&	1	&	2	\\
1	&	1	&	1
\end{pmatrix}
\end{align*}
From these, we compute the characteristic polynomials to be
\begin{align*}
p_{T_{\alpha}}
&=
t^{3} - 2
&
p_{T_{\beta}}
&=
t^{3} - 3 t^{2} - 3 t - 1
\end{align*}
By construction, the characteristic polynomial is monic; by part \ref{itm : EH06f2}, it has the given element ($\alpha$ or $\beta$) as a zero; and (for these particular polynomials) one can verify each is irreducible. We conclude that each is the minimal polynomial of its respective element over $\Q$. (In general, the minimal polynomial will be a factor, irreducible over the base field, of the characteristic polynomial.)



\subsection*{Exercise \ref{itm : EH06g}}

Let $K_{0}$ be a perfect field, and let $f \in K_{0}[t]$. We have seen that if $f$ is irreducible, then $f$ is separable.% Begin footnote.
\footnote{See DF3e, p 549.} % End footnote.
In particular, finite fields (for example, $\F_{p}$) are perfect. Thus for this exercise, it suffices to prove that $f$ is irreducible.% Begin footnote.
\footnote{That said, we can give a quick proof that $f$ is separable using the formal derivative. Recall (see DF3e, Proposition 13.33, p 547.) that a polynomial $f$ is separable if and only if $\gcd(f, D_{t} f) = 1$. Using that $\characteristic \F_{p} = p$, we compute
\begin{align*}
D_{t} f
=
D_{t} (t^{p} - t + a)
=
(p \cdot 1_{\F_{p}}) t^{p - 1} - 1 + 0
=
-1
\end{align*}
Therefore $\gcd(f, D_{t} f) = 1$, so $f$ is separable.}% End footnote.

Lemma. Let $K : \F_{p}$, and let $\alpha \in K$ such that $f(\alpha) = 0_{K}$. Then $K$ contains $n$ distinct zeros of $f$.

Proof of lemma. Note that $\characteristic K = \characteristic \F_{p} = p$.% Begin footnote.
\footnote{This follows immediately in both the subfield ($\F_{p} \subseteq K$) and injective field homomorphism ($\F_{p} \rightarrow K$) view of a field extension $K : \F_{p}$.} % End footnote.
By hypothesis,
\begin{align*}
0_{K}
=
f(\alpha)
=
\alpha^{p} - \alpha + a
\end{align*}
Note that, for all $x, y \in K$,
\begin{align*}
(x + y)^{p}
=
x^{p} + {p \choose 1} \cdot x^{p - 1} y + \ldots + {p \choose p - 1} \cdot x y^{p - 1} + y^{p}
=
x^{p} + y^{p}
\end{align*}
where for the final equality, we observe that for $1 \leq m \leq p - 1$, ${p \choose m}$ is divisible by $p$ in $\Z$; because $\characteristic K = p$, all ``middle'' terms are zero in $K$. Using this, we compute
\begin{align*}
f(\alpha + 1_{K})
=
(\alpha + 1_{K})^{p} - (\alpha + 1_{K}) + a
=
\alpha^{p} + 1_{K} - \alpha - 1_{K} + a
=
\alpha^{p} - \alpha + a
=
f(\alpha)
=
0
\end{align*}
We conclude that if an extension field $K : \F_{p}$ contains one zero $\alpha$ of $f$, then it contains all $n$ zeros of $f$. More precisely, these zeros are
\begin{align*}
\alpha, \alpha + 1_{K}, \ldots, \alpha + (p - 1) \cdot 1_{K}
\end{align*}
Because $\characteristic K = p$, these zeros are distinct,% Begin footnote.
\footnote{More precisely, suppose that there were $m, n \in \{0, \ldots, p - 1\}$ such that $m \neq n$ and $\alpha + m \cdot 1_{K} = \alpha + n \cdot 1_{K}$ in $K$. Without loss of generality, assume $n > m$. Then $0 = (n - m) \cdot 1_{K}$ with $n - m < p$, contradicting $\characteristic K = p$.} % End footnote.
as desired.

Note that the lemma also shows, directly, that $f$ is separable.

Corollary. $f$ contains no zeros in $\F_{p}$.

Proof. Suppose for the sake of contradiction that there exists an $\alpha \in \F_{p}$ such that $f(\alpha) = 0$. Then the lemma implies that
\begin{align*}
f
=
\prod_{\beta \in \F_{p}} (t - \beta)
\end{align*}
In particular, $a = f(0) = 0$, contradicting the hypothesis that $a \neq 0$.

We now use the lemma to show that $f$ is irreducible (and hence separable). Let
\begin{align*}
f
=
\prod_{i = 1}^{n} f_{i}
\end{align*}
be a factorization of $f$ into irreducible factors $f_{i} \in \F_{p}[t]$. Because $f$ is monic, without loss of generality, we may assume that each $f_{i}$ is monic.% Begin footnote.
\footnote{Because $\F_{p}$ is a field, by definition of polynomial multiplication, the product of the leading coefficients of all the $f_{i}$ equals the leading coefficient of $f$, which is $1_{\F_{p}}$. Factoring out the leading coefficient from each $f_{i}$ leaves each $f_{i}$ monic, and the product of these leading coefficients equals $1_{\F_{p}}$ and hence does not change the product.} % End footnote.
We wish to show that $n = 1$.

Claim: The degree of each factor $f_{i}$ of $f$ is equal. To see this, let $K$ be a splitting field for $f$ over $\F_{p}$. For each $i \in \{1, \ldots, n\}$, let $\alpha_{i} \in K$ be a zero of $f_{i}$. Because each $f_{i}$ is irreducible and monic, $f_{i}$ is the minimal polynomial of $\alpha_{i}$ over $\F_{p}$, denote it $m_{\alpha_{i}, \F_{p}}$. Because $\alpha_{i}$ is a zero of $f_{i}$, it is also a zero of $f$. Hence by the lemma,
\begin{align*}
\alpha_{i}
\in
\{\alpha_{1}, \alpha_{1} + 1_{K}, \ldots, \alpha_{1} + (p - 1) \cdot 1_{K}\}
\subseteq
K
\end{align*}
That is, for each $i \in \{1, \ldots, n\}$, there exists a $m_{i} \in \{0, \ldots, p - 1\} \subseteq \Z$ such that
\begin{align*}
\alpha_{i}
=
\alpha_{1} + m_{i} \cdot 1_{K}
\end{align*}
It follows that, for each $i \in \{1, \ldots, n\}$,
\begin{align*}
f_{1}(t - m_{i} \cdot 1_{K})
\end{align*}
is an irreducible, monic polynomial in $\F_{p}[t]$ with $\alpha_{i}$ as a zero. That is,
\begin{align*}
f_{1}(t - m_{i} \cdot 1_{K})
=
m_{\alpha_{i}, \F_{p}}(t)
=
f_{i}(t)
\end{align*}
so
\begin{align*}
\deg f_{i}(t)
=
\deg f_{1}(t - m_{i} \cdot 1_{K})
=
\deg f_{1}(t)
\end{align*}
proving the claim.

Let $d = \deg f_{i}$ denote the common degree of each (irreducible) factor $f_{i}$ of $f$. Because $\F_{p}$ is a field (and hence an integral domain),
\begin{align*}
p
=
\deg f
=
\deg \prod_{i = 1}^{n} f_{i}
=
\sum_{i = 1}^{n} \deg f_{i}
=
n d
\end{align*}
Because $p$ is prime, $p \divides n$ or $p \divides d$. If $p \divides n$, then we must have $n = p$ and $d = 1$. However, $d = 1$ implies that each $\alpha_{i} \in \F_{p}$, contradicting the corollary above. Therefore we must have $n = 1$ and $d = p$. That is, $f$ is irreducible in $\F_{p}[t]$, as desired.



%Suppose $f$ is reducible, say
%\begin{align}
%f
%=
%g_{1} g_{2}%
%\label{eq : EH06g f reducible}
%\end{align}
%for $g_{i} \in \F_{p}[t]$ with $\deg g_{i} \geq 1$. Because $\F_{p}$ is a field and hence an integral domain,
%\begin{align}
%p
%=
%\deg f
%=
%\deg(g_{1} g_{2})
%=
%\deg g_{1} + \deg g_{2}%
%\label{eq : EH06g sum degree gi}
%\end{align}
%This, along with each $\deg g_{i} \geq 1$, implies that
%\begin{align}
%1 \leq \deg g_{i} < \deg f = p%
%\label{eq : EH06g degree gi}
%\end{align}
%Taking the formal derivative of equation \eqref{eq : EH06g f reducible} and using the product law, we get
%\begin{align*}
%-1
%=
%D_{t} f
%=
%D_{t}(g_{1} g_{2})
%=
%D_{t} g_{1} \cdot g_{2} + g_{1} \cdot D_{t} g_{2}
%\end{align*}
%From the conditions on $\deg g_{i}$ in \eqref{eq : EH06g sum degree gi} and \eqref{eq : EH06g degree gi} above, it follows that if $p > 2$, then at least one of $D_{t} g_{1}$ and $D_{t} g_{2}$ has degree greater than or equal to $1$, and therefore at least one of $D_{t} g_{1} \cdot g_{2}$ and $g_{1} \cdot D_{t} g_{2}$ has degree greater than or equal to $2$. (Show terms in sum on right cannot cancel to leave constant only. (!!!))
%
%It remains to consider the case $p = 2$. In this case, the only nonzero element $a \in \F_{p}$ is $a = 1$, so $f = t^{2} + t + 1$, which is a degree-2 polynomial whose associated function has no zeros in $\F_{2}$, hence $f$ is irreducible.
\end{document}