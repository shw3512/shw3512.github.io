\documentclass[oneside, english, 11pt]{article}
\usepackage{geometry}
\geometry{letterpaper}
\geometry{left=1in, top=1in, right=1in, bottom=1in, nohead}
\setlength\parindent{0.25in}



\usepackage{amsmath}
\usepackage{amssymb}
\usepackage{enumitem}
\usepackage{ifthen}
%\usepackage{mathtools}

\newcommand{\showSolutions}{1}% Show solutions?
% Allowed values:
% - 0 = Do not show
% - 1 = Show

% Display or do not display solutions.
\ifthenelse{\equal{\showSolutions}{1}}{%
  \newcommand{\spaceSolution}[2]{#2}
}%
{%
  \newcommand{\spaceSolution}[2]{\vspace{#1}}
}%



% Sets fonts. Euler for math | Palatino for rm | Helvetica for ss | Courier for tt
% http://www.tug.org/mactex/fonts/LaTeX_Preamble-Font_Choices.html

\renewcommand{\rmdefault}{ppl}% rm
\linespread{1.05}% Palatino needs more leading
\usepackage[scaled]{helvet}% ss
\usepackage{courier}% tt
%\usepackage{euler}% math
\usepackage{eulervm}% math --- a better implementation of the euler package (not in gwTeX)
\normalfont
\usepackage[T1]{fontenc}



\newcommand{\fontDefWord}[1]{\textbf{#1}}
\newcommand{\fontHint}[1]{\emph{Hint:} #1}



\newcommand{\divides}{\,|\,}
\newcommand{\F}{\mathbf{F}}
\DeclareMathOperator{\Frac}{Frac}
\newcommand{\idealeq}{\trianglelefteq}
\newcommand{\integers}{\mathbf{Z}}
\newcommand{\integersPositive}{\integers_{> 0}}
\DeclareMathOperator{\LC}{LC}
\newcommand{\notDivides}{{}\hspace{-1mm}\not\hspace{-.15em}\divides{}}
\newcommand{\order}{\#}
\newcommand{\Q}{\rationals}
\newcommand{\rationals}{\mathbf{Q}}
\DeclareMathOperator{\reverse}{rev}
\newcommand{\st}{\, | \,}
\newcommand{\Z}{\integers}



\title{Math 357\\Expositional homework 03}
\author{}
\date{Assigned: 2024--02--07 (W)\\\hspace{0.37in}Due: 2024--02--19 (M)}



% Document begins here.

\begin{document}

\maketitle

The goal of this homework is to practice and extend our irreducibility--detector toolkit. The exercises are adapted from Dummit \&{} Foote, 3e, Section 9.4.

Let $t$ be an indeterminate; let $\Z$ denote the ring of integers; and for $p \in \Z$ prime, let $\F_{p} = \Z / (p)$ (a finite field with $p$ elements).

\begin{enumerate}[label=(\alph*)]
\item\label{itm : EH03a} Prove the Eisenstein--Sch\"{o}nemann criterion for $\Z$, as we stated it in class: Let $p \in \Z$ be prime; let $f = a_{n} t^{n} + \ldots + a_{0} \in \Z[t]$, with $n = \deg f \geq 1$; and let $p \notDivides a_{n}$, $p \divides a_{n - 1}, \ldots, a_{0}$, and $p^{2} \notDivides a_{0}$. Then $f$ is irreducible in $\Q[t]$. Moreover, if $\gcd(a_{n}, \ldots, a_{0}) = 1$, then $f$ is irreducible in $\Z[t]$.

\item\label{itm : EH03b} For each of the following polynomials, determine whether it is irreducible or reducible in the indicated polynomial ring. If it is reducible, then give its factorization into irreducibles.
\begin{align*}
f(t)
&=
t^{6} + 30 t^{5} - 15 t^{3} + 6 t - 120 \in \Z[t]
\\
g(t)
&=
t^{3} + t + 1 \in \Z[t]
\\
h(t)
&=
t^{3} + t + 1 \in \F_{3}[t]
\end{align*}

\item\label{itm : EH03c} Let $n \in \integersPositive$, and consider the polynomial
\begin{align*}
f_{n}(t)
=
1 + \prod_{i = 1}^{n} (t - i)
\in
\Z[t]
\end{align*}
Show that $f_{n}$ is irreducible for all $n \neq 4$.

\item\label{itm : EH03d} Let $p \in \Z$ be prime, and consider the cyclotomic polynomial% Begin footnote.
\footnote{Note that we can view $\Phi_{p}$ as the quotient of $t^{p} - 1$ when (evenly) divided by $t - 1$; that is, $\Phi_{p}(t) = \frac{t^{p} - 1}{t - 1}$.}% End footnote.
\begin{align*}
\Phi_{p}(t)
=
t^{p - 1} + \ldots + 1
\in
\Z[t]
\end{align*}
Show that $\Phi_{p}$ is irreducible. Explain the technique you use. \fontHint{See p 310.}

\item\label{itm : EH03e} Let $F$ be a field, let $f \in F[t]$, and let $n = \deg f$. The \fontDefWord{reverse} of $f$ is the polynomial $t^{n} f(t^{-1})$. Justify why this construction gives a valid polynomial in $F[t]$ (even though $t^{-1}$ is not an element of $F[t]$). Give an example of a polynomial and its reverse that clearly illustrates why this name is apt for this construction. Prove that if $f(0) \neq 0$, then $f$ is irreducible if and only if its reverse is irreducible.
\end{enumerate}



\spaceSolution{4in}{% Begin solution.
\subsection*{Solutions}

\subsubsection*{Exercise \ref{itm : EH03a}}

Suppose for the sake of contradiction that $f$ is reducible in $\Q[t]$. Then by Gauss's lemma, $f$ is reducible in $\Z[t]$, say
\begin{align}
f
=
f_{1} f_{2}%
\label{eq : EH03a Factorization}
\end{align}
with $\deg f_{i} \geq 1$.% Begin footnote.
\footnote{See DF3e, Proposition 9.5, pp 303--4.} % End footnote.
Given a prime $p \in \Z$, let
\begin{align*}
\varphi_{p}
:
\Z[t]
\rightarrow
(\Z / (p))[t]
\end{align*}
be the reduction homomorphism associated to the prime ideal $(p)$ in $\Z$, which sends a polynomial $g \in \Z[t]$ to the polynomial $\overline{g} = \varphi_{p}(g) \in (\Z / (p))[t]$ whose coefficients are those of $g$ reduced modulo $p$. Using (i) the fact that $\varphi_{p}$ is a ring homomorphism and (ii) the hypotheses on the coefficients of $f$, we get
\begin{align*}
\overline{f}_{1} \overline{f}_{2}
=
\varphi_{p}(f_{1}) \varphi_{p}(f_{2})
=
\varphi_{p}(f_{1} f_{2})
=
\varphi_{p}(f)
=
\overline{a}_{n} t^{n}
\end{align*}
with $\overline{a}_{n} \neq 0$ in $\Z / (p)$. Thus $\overline{f}_{1}$ and $\overline{f}_{2}$ divide $\overline{a}_{n} t^{n}$ in $(\Z / (p))[t]$. Because $(p)$ is a prime (in fact, maximal) ideal in $\Z$, $\Z / (p)$ is an integral domain (in fact, a field), so $(\Z / (p))[t]$ is an integral domain. Therefore, the constant term of both $\overline{f}_{1}, \overline{f}_{2} \in (\Z / (p))[t]$ is $0$, which is true if and only if the constant term of both $f_{1}, f_{2} \in \Z[t]$ is divisible by $p$. This implies that the constant term of $f = f_{1} f_{2}$ is divisible by $p^{2}$, a contradiction. We conclude that $f$ is irreducible in $\Q[t]$.

Further suppose that the greatest common divisor of the coefficients of $f$ is $1$. Recall Corollary 9.6 (p 304):
\begin{quote}
Let $R$ be a unique factorization domain, let $F = \Frac R$, let $f \in R[t]$, and suppose that the greatest common divisor of the coefficients of $f$ is $1$. Then $f$ is irreducible in $R[t]$ if and only if $f$ is irreducible in $F[t]$.
\end{quote}
We have just shown that $f$ is irreducible in $\Q[t] = (\Frac \Z)[t]$. Because $\Z$ is a euclidean domain, it is a unique factorization domain, and hence $\Z[t]$ is a unique factorization domain.% Begin footnote.
\footnote{See DF3e, Corollary 9.8, p 305.} % End footnote.
Hence Corollary 9.6 applies, and $f$ is irreducible in $\Z[t]$. 



\subsubsection*{Exercise \ref{itm : EH03b}}

We analyze each polynomial in turn.

$f = t^{6} + 30 t^{5} - 15 t^{3} + 6 t - 120 \in \Z[t]$ is irreducible by the Eisenstein--Sch\"{o}nemann criterion with the prime $p = 3$. In particular, $3$ does not divide the leading coefficient of $f$, $3$ divides all other coefficients of $f$, and $3^{2}$ does not divide the constant term of $f$.

$g = t^{3} + t + 1 \in \Z[t]$ is irreducible. We give two proofs. Both use the fact that a polynomial of degree $2$ or $3$ over a field is reducible if and only if its associated function has a zero.% Begin footnote.
\footnote{See DF3e, Proposition 9.10, p 308.}% End footnote.
\begin{enumerate}
\item Method 1: Reduction modulo a proper ideal. Consider the maximal ideal $(2) \idealeq \Z$ and the associated reduction homomorphism $\varphi_{2} : \Z[t] \rightarrow (\Z / (2))[t]$. The polynomial
\begin{align*}
\varphi_{2}(g)
=
t^{3} + t + 1
\in
(\Z / (2))[t]
\end{align*}
has no zeros in the field $\Z / (2)$, so $\varphi_{2}(g)$ is irreducible, and therefore $g$ is irreducible.% Begin footnote.
\footnote{The last implication uses DF3e Proposition 9.12, p 309.}% End footnote.

Note that the reduction part of this argument requires only a proper, not necessarily maximal, ideal. However, the equivalence of reducibility and zeros for degree-2 and degree-3 polynomials requires that the ring of coefficients be a field, which (for a quotient ring) requires that the ideal by which we quotient be maximal.

\item Method 2: Analysis in $\Q[t]$. View $g \in \Q[t]$. Because $\deg g = 3$, the polynomial $g$ is reducible if and only if the induced function $g : \Q \rightarrow \Q$ has a zero. By the rational zeros test, if $\frac{a}{b} \in \Q$ is a zero of $g$ with $\gcd(a, b) = 1$ (that is, $\frac{a}{b}$ is in lowest terms), then $a$ divides the constant term of $g$ and $b$ divides the leading coefficient of $g$; that is, $a \divides 1$ and $b \divides 1$, so $\frac{a}{b} = \pm{}1$. We compute
\begin{align*}
g(-1)
&=
-1
&
g(1)
&=
3
\end{align*}
We conclude that $g$ is irreducible in $\Q[t]$. Because the greatest common divisor of the coefficients of $g$ is $1$, if $g$ is irreducible in $\Q[t]$, then $g$ is irreducible in $\Z[t]$, as desired.
\end{enumerate}

$h =  t^{3} + t + 1 \in \F_{3}[t]$ is reducible. Because $\deg h = 3$, the polynomial $h$ is reducible if and only if the induced function $h$ has a zero. We compute that $h(1) = 0$, so $h$ is reducible.



\subsubsection*{Exercise \ref{itm : EH03c}}

% Provide an overview of the logic in this proof. (!!!)

Suppose for the sake of contradiction that $f_{n}$ is reducible in $\Z[t]$, say
\begin{align*}
f_{n}
=
g_{1} g_{2}
\end{align*}
Without loss of generality, let $\deg g_{1} \leq \deg g_{2}$. By definition of $f_{n}$, $\LC(f_{n}) = 1$, so the greatest common divisor of the coefficients of $f_{n}$ equals $1$, and therefore each $g_{i}$ has degree at least $1$.

Note that
\begin{align*}
g_{1} g_{2} - 1
=
f_{n} - 1
=
\prod_{i = 1}^{n} (t - i)
\end{align*}
This implies that for each $\alpha \in \{1, \ldots, n\}$,
\begin{align*}
(g_{1} g_{2} - 1)(\alpha)
&=
0
&
&\Leftrightarrow
&
g_{1}(\alpha) g_{2}(\alpha)
&=
1
&
&\Leftrightarrow
&
g_{1}(\alpha), g_{2}(\alpha)
&=
\pm{}1
\end{align*}
where in the final statement, for each $\alpha$, $g_{1}(\alpha)$ and $g_{2}(\alpha)$ have the same sign. Denote
\begin{align*}
S_{1}
&=
\{\alpha \in \{1, \ldots, n\} \st g_{i}(\alpha) = 1\}
&
S_{-1}
&=
\{\alpha \in \{1, \ldots, n\} \st g_{i}(\alpha) = -1\}
\end{align*}
Note that $S_{1} \cup S_{-1} = \{1, \ldots, n\}$, so at least one of $S_{1}, S_{-1}$ has order greater than or equal to $\lceil{}\frac{n}{2}\rceil{}$. Denote this set by $S_{M}$, where $M = \pm{1}$.

By hypothesis, $f_{n} = g_{1} g_{2}$. Because $\Z$ is an integral domain, $\Z[t]$ is an integral domain, so
\begin{align*}
n
=
\deg f_{n}
=
\deg(g_{1} g_{2})
=
\deg g_{1} + \deg g_{2}
\end{align*}
Thus at least one of the $g_{i}$ has degree less than or equal to $\lfloor{}\frac{n}{2}\rfloor{}$. Earlier we arranged that $\deg g_{1} \leq \deg g_{2}$, so we have $\deg g_{1} \leq \lfloor{}\frac{n}{2}\rfloor{}$.

Suppose $\deg g_{1} < \frac{n}{2}$. Then the polynomial $g_{1} - M$, which we may view in $\Q[t]$, has degree
\begin{align*}
\deg(g_{1} - M)
=
\deg g_{1}
<
\frac{n}{2}
\end{align*}
and at least $\order(S_{M}) \geq \lceil{}\frac{n}{2}\rceil{} \geq \frac{n}{2}$ zeros, a contradiction. When $n$ is odd, necessarily $\deg g_{1} < \frac{n}{2}$. Thus this contradiction shows that if $n$ is odd, then $f_{n}$ is irreducible.

This contradiction also shows that if $n$ is even, then we must have $\deg g_{1} = \deg g_{2} = \frac{n}{2}$ and $\order(S_{1}) = \order(S_{2}) = \frac{n}{2}$. In this case, the set $S_{1}$ contains $\frac{n}{2}$ distinct zeros for the degree-$\frac{n}{2}$ polynomial $g_{i} - 1$, for $i \in \{1, 2\}$. (The analogous statement holds for the set $S_{-1}$ and the polynomial $g_{i} + 1$.) Therefore, for $i \in \{1, 2\}$,
\begin{align*}
g_{i}
=
-1 + \prod_{\alpha \in S_{1}} (t - \alpha)
=
1 + \prod_{\alpha \in S_{-1}} (t - \alpha)
\end{align*}
In particular, $g_{1} = g_{2}$; denote the common polynomial by $g$. Thus
\begin{align*}
g^{2}
=
f_{n}
=
1 + \prod_{i = 1}^{n} (t - i)
\end{align*}
That is, if the polynomial $f_{n}$ is reducible in $\Z[t]$, then it is a square. If we can show that the corresponding function $f_{n}$ evaluates to negative values, then we will have a contradiction.

View the polynomials in $\Q[t]$, and consider evaluating their functions at $t = n - \frac{1}{2}$:
\begin{align}
\left(g\left(n - \frac{1}{2}\right)\right)^{2}
=
g^{2}\left(n - \frac{1}{2}\right)
=
f_{n}\left(n - \frac{1}{2}\right)
=
1 + \left(-\frac{1}{2}\right) \prod_{i = 1}^{n - 1} \left(n - \frac{1}{2} - i\right)%
\label{eq : EH03c fn as square}
\end{align}
If $n > 4$, then the product in this expression satisfies
\begin{align*}
\prod_{i = 1}^{n - 1} \left(n - \frac{1}{2} - i\right)
=
\left(\frac{1}{2}\right) \left(\frac{3}{2}\right) \left(\frac{5}{2}\right) \left(\frac{7}{2}\right) \cdots \left(n - \frac{1}{2} - 1\right)
>
\frac{105}{16}
>
2
\end{align*}
where we obtain the penultimate inequality by keeping only the first four factors and noting that any subsequent factors are greater than or equal to $1$. Substituting this result into equation \eqref{eq : EH03c fn as square}, it follows that if $n > 4$, then
\begin{align*}
\left(g\left(n - \frac{1}{2}\right)\right)^{2}
<
1 - \frac{1}{2} (2)
=
0
\end{align*}
a contradiction (recall that we are viewing $g$ as a function from $\Q$ to $\Q$).

It remains to address the cases $n = 2$ and $n = 4$.

When $n = 2$,
\begin{align*}
f_{2}(t)
=
t^{2} - 3 t + 3
\end{align*}
which is irreducible by the Eisenstein--Sch\"{o}nemann criterion with $p = 3$.

When $n = 4$,
\begin{align*}
f_{4}(t)
=
t^{4} - 10 t^{3} + 35 t^{2} - 50 t + 25
=
(t^{2} - 5 t + 5)^{2}
\end{align*}
so $f_{4}$ is reducible.



\subsubsection*{Exercise \ref{itm : EH03d}}

Note that we cannot apply the Eisenstein--Sch\"{o}nemann criterion directly to $\Phi_{p}(t)$. However, we compute
\begin{align*}
\Phi_{p}(t + 1)
&=
\frac{(t + 1)^{p} - 1}{(t + 1) - 1}
\\
&=
t^{p - 1} + p t^{p - 2} + \ldots + \frac{p (p - 1)}{2} t + p
\end{align*}
We note two aspects. First, the constant term in the numerator, before dividing by $t$ in the denominator, is zero. Second, the coefficients of the final polynomial are ${p \choose i} = \frac{p (p - 1) \cdots (p - i + 1)}{i!}$, for $i \in \{0, \ldots, p - 1\}$; in particular, all coefficients except the leading coefficient (corresponding to $i = 0$) are divisible by $p$. Thus we may apply the Eisenstein--Sch\"{o}nemann criterion with the prime $p$ to $\Phi_{p}(t + 1)$ to conclude that $\Phi_{p}(t + 1)$, and hence $\Phi_{p}(t)$, is irreducible.



\subsubsection*{Exercise \ref{itm : EH03e}}

Let
\begin{align*}
f
=
a_{n} t^{n} + \ldots + a_{0}
\end{align*}
By definition, its reverse is
\begin{align*}
\reverse f
=
t^{n} f(t^{-1})
=
t^{n} (a_{n} t^{-n} + \ldots + a_{0})
=
a_{n} + \ldots + a_{0} t^{n}
\end{align*}
Because each monomial term in $f(t^{-1})$ has the form $a_{i} t^{-i}$ for some $i \in \{0, \ldots, n\}$, it follows that each monomial term in $t^{n} f(t^{-1})$ has the form $a_{i} t^{n - i}$, with $n - i \geq 0$. Thus this construction gives a polynomial in $F[t]$.

As an example, consider the polynomial
\begin{align*}
f
=
t^{3} + 2 t + 3
\end{align*}
in $\Q[t]$. Then
\begin{align*}
\reverse f
=
t^{3} f(t^{-1})
=
1 + 2 t^{2} + 3 t^{3}
=
3 t^{3} + 2 t^{2} + 1
\end{align*}
This construction ``reverses'' the order of the coefficients of $f$ relative to the powers of $t$.

Note that this construction can product a polynomial of strictly smaller degree than the input. More precisely, $\deg \reverse f < \deg f$ if and only if the constant term of $f$ is $0$---that is, if and only if $f(0) \neq 0$. For example, if $f = t$, then $\reverse f = 1$.

Let $f \in F[t]$, and suppose that $f = g_{1} g_{2}$ for some $g_{i} \in F[t]$. Then a bookkeeping exercise with the definition of multiplication of polynomials shows that
\begin{align*}
(\reverse g_{1}) (\reverse g_{2})
=
\reverse f
\end{align*}
Because the reverse of the reverse of a polynomial is the original polynomial, the reverse implication also holds.

Further suppose that $f(0) \neq 0$. Then because $F$ is a field (and hence an integral domain), each $g_{i}$ in the factorization $f = g_{1} g_{2}$ satisfies $g_{i}(0) \neq 0$. In particular, if $g_{i}(0) \neq 0$ and $\deg g_{i} \geq 1$, then $\deg(\reverse g_{i}) \geq 1$. It follows that such an $f$ is reducible if and only if $\reverse f$ is reducible.}% End solution.

\end{document}