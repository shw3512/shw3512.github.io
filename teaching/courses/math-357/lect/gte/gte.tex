\documentclass[oneside, english, 11pt]{article}
\usepackage{geometry}
\geometry{letterpaper}
\geometry{left=1in, top=1in, right=1in, bottom=1in, nohead}
\setlength\parindent{0.25in}



\usepackage{amsmath}
%\usepackage{amsthm}
%\usepackage{enumitem}



% Sets fonts. Euler for math | Palatino for rm | Helvetica for ss | Courier for tt
% http://www.tug.org/mactex/fonts/LaTeX_Preamble-Font_Choices.html

\renewcommand{\rmdefault}{ppl}% rm
\linespread{1.05}% Palatino needs more leading
\usepackage[scaled]{helvet}% ss
\usepackage{courier}% tt
%\usepackage{euler}% math
\usepackage{eulervm}% math --- a better implementation of the euler package (not in gwTeX)
\normalfont
\usepackage[T1]{fontenc}



%\newcommand{\fontDefWord}[1]{\textbf{#1}}
\newcommand{\fontField}[1]{\mathbf{#1}}
%\newcommand{\fontHint}[1]{\emph{Hint:} #1}



\newcommand{\Aut}{\automorphisms}
\DeclareMathOperator{\automorphisms}{Aut}
%\newcommand{\C}{\complexes}
%\DeclareMathOperator{\characteristic}{char}
%\newcommand{\complexes}{\fontField{C}}
%\newcommand{\divides}{\,|\,}
%\DeclareMathOperator{\Endomorphisms}{End}
%\newcommand{\F}{\fontField{F}}
\newcommand{\fixedField}{\mathcal{F}}
\newcommand{\Gal}{\galoisGroup}
\DeclareMathOperator{\galoisGroup}{Gal}
%\DeclareMathOperator{\generalLinear}{GL}
%\newcommand{\GL}{\generalLinear}
%\DeclareMathOperator{\Hom}{Hom}
\newcommand{\id}{\identityMap}
\DeclareMathOperator{\identityMap}{id}
%\newcommand{\integers}{\fontField{Z}}
%\newcommand{\integersPositive}{\integers_{> 0}}
%\newcommand{\iso}{\isomorphism}
%\newcommand{\isomorphism}{\cong}
%\newcommand{\notDivides}{{}\hspace{-1mm}{\not\hspace{-.15em}\divides{}}\,}
%\newcommand{\order}{\#}
\newcommand{\Q}{\rationals}
\newcommand{\rationals}{\fontField{Q}}
\newcommand{\st}{{\, | \,}}
\newcommand{\transpose}{^\mathrm{t}}
%\newcommand{\Z}{\integers}



\title{Math 357\\Galois theory example}
%\author{}
%\date{}



% Document begins here.

\begin{document}

\maketitle

In these notes we use galois theory to analyze algebraic objects---zeros, field extensions, and groups---arising from the polynomial $f = t^{3} - 2 \in \Q[t]$. We start with a summary of the story. We invite the reader to fill in details and justify the assertions before proceeding to following sections, in which we explore the story and logic together.



\section{Overview}

Let
\begin{align*}
f
=
t^{3} - 2
\in
\Q[t]
\end{align*}
A splitting field for $f$ is $K = \Q(\alpha, \zeta_{3})$, where $\alpha$ is a zero of $f$, and $\zeta_{3}$ is a zero of $g = t^{2} + t + 1 \in \Q[t]$. A basis for $K$ as a $\Q$-vector space is
\begin{align*}
\mathcal{B}
=
(1, \alpha, \alpha^{2}, \zeta_{3}, \zeta_{3} \alpha, \zeta_{3} \alpha^{2})
\end{align*}
The field extension $K : \Q$ is galois, with galois group
\begin{align*}
\Gal(K : \Q)
=
\langle{}\sigma, \tau \st \sigma^{3}, \tau^{2}, (\sigma \tau)^{2}\rangle{}
\end{align*}
where the automorphisms $\sigma, \tau : K \rightarrow K$ are defined by
\begin{align}
\sigma
:{}
&\alpha \mapsto \zeta_{3} \alpha
&
\tau
:{}
&\alpha \mapsto \alpha
\label{eq : generators of galois group}
\\
&\zeta_{3} \mapsto \zeta_{3}
&
&\zeta_{3} \mapsto \zeta_{3}^{2} = -(1 + \zeta_{3})
\nonumber
\end{align}
One can use the galois correspondence between subfields and subgroups to enumerate and match all of each. In particular, the subfield $\Q(\zeta_{3} \alpha)$ corresponds to the subgroup $\langle{}\sigma^{2} \tau\rangle{}$. That is,
\begin{align*}
\fixedField(\langle{}\sigma^{2} \tau\rangle{})
&=
\Q(\zeta_{3} \alpha)
&
\Aut(K : \Q(\zeta_{3} \alpha))
&
=
\langle{}\sigma^{2} \tau\rangle{}
\end{align*}



%\section{Constructing field extensions}
%
%We can construct the field $K$ in steps. In the ``top-down'' view, we adjoin one element after another. In the ``bottom-up'' view, we form one quotient ring after another. Recall that the ``top-down'' view requires that we have or assert the existence of a field that contains all the elements we need. In our present example, a splitting field for $f$ over $\Q$, or an algebraic closure of $\Q$, will do. In what follows, however, let's explore the ``bottom-up'' view.
%
%The polynomial $f$ is irreducible in $\Q[t]$---for example, by the Eisenstein--Sch\"{o}nemann criterion with $p = 2$, or by the rational zeros test. Therefore the principal ideal ${f}$ is maximal in $\Q[t]$, so the quotient ring $\Q[t] / (f)$ is a field, denote it $K_{1}$. Consider the maps
%\begin{align*}
%\Q
%\xrightarrow{\iota_{0}}
%\Q[t]
%\xrightarrow{\pi_{0}}
%\Q[t] / (f)
%=
%K_{1}
%\xrightarrow{\iota_{1}}
%K_{1}[t]
%\end{align*}
%where $\iota_{0}$ and $\iota_{2}$ are inclusion maps, and $\pi_{0}$ is the quotient map defined by $h \mapsto h + (f)$.



\section{Subfield--subgroup correspondence}

Claim: $\fixedField(\langle{}\sigma^{2} \tau\rangle{}) = \Q(\zeta_{3} \alpha)$.

To prove this claim, we show set containment in both directions. Recall that $\langle{}\sigma^{2} \tau\rangle{} = \{\id_{K}, \sigma^{2} \tau\}$, and the identity map $\id_{K}$ fixes all of $K$, and hence, in particular, $\Q \subseteq K$. Thus for the fixed-field computations here, it suffices to analyze $\sigma^{2} \tau$. (More generally, we need to analyze a set of generators of the subgroup.) Also recall that, by definition, $\zeta_{3}$ is a zero of $g = t^{2} + t + 1$, hence also a zero of $(t - 1) g = t^{3} - 1$. Explicitly writing out what these zeros mean, we get
\begin{align*}
\zeta_{3}^{2}
&=
-(1 + \zeta_{3})
&
\zeta_{3}^{3}
&=
1
\end{align*}

($\supseteq$) Because $\sigma, \tau \in \Gal(K : \Q) = \Aut(K : \Q)$, they are field homomorphisms. Using this fact and the defining images of $\sigma$ and $\tau$ specified in equation \eqref{eq : generators of galois group}, we compute
\begin{align*}
\sigma^{2} \tau(\zeta_{3} \alpha)
&=
\sigma(\sigma(\tau(\zeta_{3} \alpha)))
\\
&=
\sigma(\sigma(\tau(\zeta_{3}) \tau(\alpha)))
\\
&=
\sigma(\sigma(\zeta_{3}^{2} \cdot \alpha))
\\
&=
\sigma(\sigma(\zeta_{3})^{2} \sigma(\alpha))
\\
&=
\sigma(\zeta_{3}^{2} \cdot \zeta_{3} \alpha)
=
\sigma(\zeta_{3}^{3} \alpha)
=
\sigma(\alpha)
\\
&=
\zeta_{3} \alpha
\end{align*}
Alternatively, we can use the same hypotheses to compute where $\sigma^{2} \tau$ sends each generator of $K = \Q(\alpha, \zeta_{3})$:
\begin{align}
\sigma^{2} \tau(\alpha)
&=
\zeta_{3}^{2} \alpha
&
\sigma^{2} \tau(\zeta_{3})
&=
\zeta_{3}^{2}%
\label{eq : s2t images}
\end{align}
to conclude that
\begin{align*}
\sigma^{2} \tau(\zeta_{3} \alpha)
=
\sigma^{2} \tau(\zeta_{3}) \cdot \sigma^{2} \tau(\alpha)
=
\zeta_{3}^{2} \cdot \zeta_{3}^{2} \alpha
=
\zeta_{3}^{4} \alpha
=
\zeta_{3} \alpha
\end{align*}
This shows that $\sigma^{2} \tau$ fixes the element $\zeta_{3} \alpha \in K$. Because $\sigma^{2} \tau \in \Gal(K : \Q)$, $\sigma^{2} \tau$ also fixes the base field $\Q$. Hence $\sigma^{2} \tau$ fixes $\Q(\zeta_{3} \alpha)$. That is,
\begin{align}
\Q(\zeta_{3} \alpha)
\subseteq
\fixedField(\langle{}\sigma^{2} \tau\rangle{})%
\label{eq : subfield in fixed field}
\end{align}

($\subseteq$) To show the reverse inclusion, let $\beta \in \fixedField(\langle{}\sigma^{2} \tau\rangle{}) \subseteq K$. Because $\mathcal{B} = (1, \alpha, \alpha^{2}, \zeta_{3}, \zeta_{3} \alpha, \zeta_{3} \alpha^{2})$ is a $\Q$-basis for $K$, there exist unique $a_{i, j} \in \Q$ such that
\begin{align}
\beta
&=
\underset{j \in \{0, 1, 2\}}{\sum_{i \in \{0, 1\},}} a_{i, j} \zeta_{3}^{i} \alpha^{j}%
\nonumber
\\
&=
a_{0, 0} 1 + a_{0, 1} \alpha + a_{0, 2} \alpha^{2} + a_{1, 0} \zeta_{3} + a_{1, 1} \zeta_{3} \alpha + a_{1, 2} \zeta_{3} \alpha^{2}%
\label{eq : beta as linear combination}
\end{align}
Because $\sigma^{2} \tau \in \Gal(K : \Q) = \Aut(K : \Q)$, it is a field homomorphism that fixes all elements of $\Q$. In particular, it fixes each $a_{i, j}$. Using these facts and our computation of $\sigma^{2} \tau$ on generators of $K : \Q$ in equation \eqref{eq : s2t images}, we compute
\begin{align}
&\sigma^{2} \tau(\beta)
=
\sigma^{2} \tau\left(\underset{j \in \{0, 1, 2\}}{\sum_{i \in \{0, 1\},}} a_{i, j} \zeta_{3}^{i} \alpha^{j}\right)
=
\underset{j \in \{0, 1, 2\}}{\sum_{i \in \{0, 1\},}} a_{i, j} \cdot (\sigma^{2} \tau(\zeta_{3}))^{i} \cdot (\sigma^{2} \tau(\alpha))^{j}%
\nonumber
\\
&=
a_{0, 0} \cdot 1 + a_{0, 1} \cdot \sigma^{2} \tau(\alpha) + a_{0, 2} \cdot (\sigma^{2} \tau(\alpha))^{2}%
\nonumber
\\
&\hspace{10mm} + a_{1, 0} \cdot \sigma^{2} \tau(\zeta_{3}) + a_{1, 1} \cdot \sigma^{2} \tau(\zeta_{3}) \cdot \sigma^{2} \tau (\alpha) + a_{1, 2} \cdot \sigma^{2} \tau(\zeta_{3}) \cdot (\sigma^{2} \tau(\alpha))^{2}%
\nonumber
\\
&=
a_{0, 0} 1 + a_{0, 1} \zeta_{3}^{2} \alpha + a_{0, 2} \zeta_{3} \alpha^{2} + a_{1, 0} \zeta_{3}^{2} + a_{1, 1} \zeta_{3} \alpha + a_{1, 2} \alpha^{2}%
\nonumber
\\
&=
a_{0, 0} 1 - a_{0, 1} (1 + \zeta_{3}) \alpha + a_{0, 2} \zeta_{3} \alpha^{2} - a_{1, 0} (1 + \zeta_{3}) + a_{1, 1} \zeta_{3} \alpha + a_{1, 2} \alpha^{2}%
\nonumber
\\
&=
(a_{0, 0} - a_{1, 0}) 1 - a_{0, 1} \alpha + a_{1, 2} \alpha^{2} - a_{1, 0} \zeta_{3} + (a_{1, 1} - a_{0, 1}) \zeta_{3} \alpha + a_{0, 2} \zeta_{3} \alpha^{2}%
\label{eq : s2t beta as linear combination}
\end{align}
By hypothesis, $\beta \in \fixedField(\langle{}\sigma^{2} \tau\rangle{})$, so
\begin{align*}
\sigma^{2} \tau(\beta)
=
\beta
\end{align*}
That is, the right sides of equations \eqref{eq : beta as linear combination} and \eqref{eq : s2t beta as linear combination} are equal. Because $\mathcal{B}$ is a $\Q$-basis (so, in particular, it is linear independent over $\Q$), for each basis element, the corresponding coefficients in these equations are equal. This gives a system of six linear equations in $\Q$:
\begin{align*}
1
&:
a_{0, 0}
=
a_{0, 0} - a_{1, 0}
&
\zeta_{3}
&:
a_{1, 0}
=
-a_{1, 0}
\\
\alpha
&:
a_{0, 1}
=
-a_{0, 1}
&
\zeta_{3} \alpha
&:
a_{1, 1}
=
a_{1, 1} - a_{0, 1}
\\
\alpha^{2}
&:
a_{0, 2}
=
a_{1, 2}
&
\zeta_{3} \alpha^{2}
&:
a_{1, 2}
=
a_{0, 2}
\end{align*}
This system of equations implies
\begin{align*}
a_{0,1}
&=
0
&
a_{1, 0}
&=
0
&
a_{1, 2}
&=
a_{0, 2}
&
a_{0, 0}, a_{1, 1}
&\in
\Q
\end{align*}
(The final expression simply states that $a_{0, 0}$ and $a_{1, 1}$ are free parameters.) That is, if $\beta \in \fixedField(\langle{}\sigma^{2} \tau\rangle{})$, then $\beta$ has the form
\begin{align*}
\beta
=
a_{0, 0} 1 + a_{1, 1} \zeta_{3} \alpha + a_{0, 2} (\alpha^{2} + \zeta_{3} \alpha^{2})
\end{align*}
for some $a_{0, 0}, a_{1, 1}, a_{0, 2} \in \Q$.% Begin footnote.
\footnote{If you like to think in matrices, then we can reinterpret our work here in that language. Having chosen a basis $\mathcal{B}$ for the $\Q$-vector space $K$, we get a matrix $M_{\mathcal{B}}(\sigma^{2} \tau)$ for the linear transformation $\sigma^{2} \tau : K \rightarrow K$. For our basis $\mathcal{B} = (1, \alpha, \alpha^{2}, \zeta_{3}, \zeta_{3} \alpha, \zeta_{3} \alpha^{2})$, we get
\begin{align*}
M_{\mathcal{B}}(\sigma^{2} \tau)
=
\begin{pmatrix}
1	&		&		&	-1	&		&		\\
	&	-1	&		&		&		&		\\
	&		&		&		&		&	1	\\
	&		&		&	-1	&		&		\\
	&	-1	&		&		&	1	&		\\
	&		&	1	&		&		&	
\end{pmatrix}
\end{align*}
Recall that the columns of this matrix are the coefficients of the linear combination with respect to the chosen basis $\mathcal{B}$ of each basis vector in $\mathcal{B}$; that is, $M_{\mathcal{B}}(\sigma^{2} \tau(\zeta_{3}^{i} \alpha^{j}))$. Given an arbitrary $\beta \in K$, its unique $\Q$-linear combination with respect to the basis $\mathcal{B}$ given in equation \eqref{eq : beta as linear combination} writes as the $6 \times 1$ matrix (column vector)
\begin{align*}
M_{\mathcal{B}}(\beta)
=
\begin{pmatrix}
a_{0, 0}	&	a_{0, 1}	&	a_{0, 2}	&	a_{1, 0}	&	a_{1, 1}	&	a_{1, 2}
\end{pmatrix}%
\transpose
\end{align*}
Multiplying $M_{\mathcal{B}}(\sigma^{2} \tau)$ by $M_{\mathcal{B}}(\beta)$ gives a $6 \times 1$ matrix, equal to $M_{\mathcal{B}}(\sigma^{2} \tau(\beta))$, whose entries are the coefficients in equation \eqref{eq : s2t beta as linear combination}. Setting this matrix equal to $M_{\mathcal{B}}(\beta)$ gives the six linear equations we listed above.

Note that, in this matrix view, solving
\begin{align*}
M_{\mathcal{B}}(\sigma^{2} \tau) M_{\mathcal{B}}(\beta)
=
M_{\mathcal{B}}(\beta)
\end{align*}
is equivalent to solving
\begin{align*}
(I - M_{\mathcal{B}}(\sigma^{2} \tau)) M_{\mathcal{B}}(\beta)
=
0
\end{align*}
where $I$ denotes the $6 \times 6$ identity matrix. That is, to compute the fixed field of an element of the galois group, we can compute the eigenspace associated to the eigenvalue $1$ for that element.}% End footnote.

The first two terms are in $\Q(\zeta_{3} \alpha)$. What about the last term? Using the relation $\zeta_{3}^{2} = -(1 + \zeta_{3})$, we compute
\begin{align*}
\alpha^{2} + \zeta_{3} \alpha^{2}
=
(1 + \zeta_{3}) \alpha^{2}
=
-\zeta_{3}^{2} \alpha^{2}
=
-(\zeta_{3} \alpha)^{2}
\in
\Q(\zeta_{3} \alpha)
\end{align*}
Remember, $\Q(\zeta_{3} \alpha)$ denotes the field generated by $\zeta_{3} \alpha$ over $\Q$. It contains $\Q$ and $\zeta_{3} \alpha$, and it is closed under the field operations. Thus, in particular, it also contains $(-1) (\zeta_{3} \alpha)^{2}$, as we asserted above. The same logic shows that $\beta \in \Q(\zeta_{3} \alpha)$. Because $\beta \in \fixedField(\langle{}\sigma^{2} \tau\rangle{})$ was arbitrary, we conclude that $\fixedField(\langle{}\sigma^{2} \tau\rangle{}) \subseteq \Q(\zeta_{3} \alpha)$. Combining this with equation \eqref{eq : subfield in fixed field}, we conclude that $\fixedField(\langle{}\sigma^{2} \tau\rangle{}) = \Q(\zeta_{3} \alpha)$, as desired.



\end{document}